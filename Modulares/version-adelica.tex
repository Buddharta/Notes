\documentclass[letterpaper]{report}
%\usepackage{pst-node}
\usepackage{tikz-cd} 
\usepackage{amsmath}
\usepackage{float}
\usepackage{amsfonts}
\usepackage{amssymb}
\usepackage[spanish,activeacute]{babel}
\usepackage{amscd}
\usepackage{fancyhdr}
\usepackage{graphicx}
\usepackage{color}
\usepackage{transparent}
\usepackage{makeidx}
\usepackage{afterpage}

\makeindex

\newtheorem{teorema}{Teorema}[chapter]
\newtheorem{prop}[teorema]{Proposici\'on}
\newtheorem{cor}[teorema]{Corolario}
\newtheorem{lema}[teorema]{Lema}
\newtheorem{obs}{Observaci\'on}
\newtheorem{def.}{Definici\'on}[chapter]
\newtheorem{afir}{Afirmaci\'on}


\renewcommand{\figurename}{Figura}
\renewcommand{\chaptername}{\Large  \sc Cap\'{\i}tulo}
\renewcommand{\indexname}{\'{I}ndice anal\'{\i}tico}
\renewcommand{\bibname}{Bibliograf\'{\i}a}
\newcommand{\dem}{{\sc Demostraci\'on. }}

\newcommand{\rac}{\ensuremath{ \mathbb Q }}
\newcommand{\nat}{\ensuremath{ \mathbb N }}
\newcommand{\dbz}{\ensuremath{ \mathbb Z }}
\newcommand{\co}{\ensuremath{\mathbb C }}
\newcommand{\hc}{\ensuremath{\widehat{\mathbb C} }}
\newcommand{\con}{\ensuremath{\mathbb{C}^{n}}}
\newcommand{\hil}{\ensuremath{\mathcal H }}
\newcommand{\re}{\ensuremath{\mathbb R }}
\newcommand{\cp}{\ensuremath{\mathbb{CP}}}
\newcommand{\rp}{\ensuremath{\mathbb{RP}}}
\newcommand{\sph}{\ensuremath{\mathbb{S}}}

\newcommand{\Oan}{\ensuremath{\mathcal{O}}}
\newcommand{\Ad}{\ensuremath{\mathbb{A}}}
\newcommand{\specKn}{\ensuremath{Spec\,K[x_1,\cdots ,x_n]}}
\newcommand{\specLn}{\ensuremath{Spec\,L[x_1,\cdots ,x_n]}}
\newcommand{\specRn}{\ensuremath{Spec\,R[x_1,\cdots ,x_n]}}
\newcommand{\specKm}{\ensuremath{Spec\,K[x_1,\cdots ,x_m]}}
\newcommand{\specQn}{\ensuremath{Spec\,\mathbb{Q}[x_1,\cdots ,x_k]}}
\newcommand{\specRen}{\ensuremath{Spec\,\mathbb{R}[x_1,\cdots ,x_k]}}
\newcommand{\specCn}{\ensuremath{Spec\,\mathbb{C}[z_1,\cdots ,z_k]}}

\newcommand{\nuc}{\ensuremath{\mathcal{N}}}
\newcommand{\hip}{\ensuremath{\mathbb H}}
\newcommand{\hd}{\ensuremath{\mathbf{H}_{\delta}}}  

\newcommand{\bg}{\ensuremath{\overline \Gamma}}
\newcommand{\ga}{\ensuremath{\Gamma}}
\newcommand{\fb}{\ensuremath{\overline f}}
\newcommand{\la}{\ensuremath{\lambda}}
\newcommand{\La}{\ensuremath{\Lambda}}
\newcommand{\bt}{\ensuremath{\overline T}}
\newcommand{\li}{\ensuremath{\mathbb{L}}}
\newcommand{\ord}{\ensuremath{\mathbb{O}}}

\newcommand{\pslz}{\ensuremath{PSL(2,\mathbb Z) }}
\newcommand{\pslr}{\ensuremath{PSL(2,\mathbb R) }}
\newcommand{\pslc}{\ensuremath{PSL(2,\mathbb C) }}
\newcommand{\qed}{\ensuremath{\hspace*{0em plus 1fill}\blacksquare}}


\begin{document}

\begin{titlepage}
\begin{center}
%\vspace*{-3cm}
%\makebox{\includegraphics[height=3cm]{unam.jpg}} 
%\vspace*{1cm}

%\LARGE\textbf{UNIVERSIDAD NACIONAL AUT'ONOMA DE MEXICO}
%\vspace*{0.3cm}

%\large PROGRAMA DE MAESTR'IA Y DOCTORADO EN CIENCIAS MATEM'ATICAS Y DE LA ESPECIALIZACI'ON EN ESTAD'ISTICA APLICADA
%\vspace*{2cm}

\LARGE\textbf{VERSIÓN ADÉLICA}
\vspace*{0.5cm}

%\large NOTAS PARA EVALUACIÓN
%\vspace*{2.5cm}

%\small \textbf{PRESENTA:}
%\vspace*{0.2cm}

\small CARLOS EDUARDO MART'INEZ AGUILAR
%\vspace*{0.2cm}

%\small DIRECTOR: DR. ADOLFO GUILLOT SANTIAGO
%\vspace*{0.2cm}

%\small \textbf{INSTITUTO DE MATEM'ATICAS UNAM}
%\vspace*{2.5cm}

%\small CIUDAD UNIVERSITARIA, ENERO DE 2019

%\tableofcontents
 
\end{center}
\end{titlepage} 

\chapter{Adeles}
\begin{def.}
Una valuación en un campo $K$ es una función $\vert\cdot\vert:K\rightarrow\re^{+}$ que safisface los siguientes axiomas

\begin{itemize}
\item[i)] $\vert a\vert=0$ si y sólo si $a=0$.
\item[ii)] $\vert a\,b\vert = \vert a\vert\,\vert b\vert$ para todo $\lbrace a,b\rbrace\subset K$
\item[iii)] Existe $C\in\re^{+}$ tal que $\vert 1+a \vert\leq C$ para todo $a\in K$
\end{itemize}
\end{def.}

\noindent Observamos que por la segunda propiedad $\vert 1\vert=\vert 1\vert\,\vert 1\vert$, por lo tanto $\vert 1\vert=1$ además si $w\in K$ es raiz de la unidad $w^n=1$, entonces $\vert w\vert=1$.\\

La valuación trivial se define por $\chi_{K\setminus\lbrace 0\rbrace}:K\rightarrow\re^{+}$
\begin{equation}
	\chi_{K\setminus\lbrace 0\rbrace}(x)=
	    \begin{cases}
      		0, & \text{if}\ x=0 \\
      		1, & \text{en otro caso}.
    	\end{cases}
\end{equation}
\noindent excluimos la valuación trivial de nuestro estudio. Así definimos dos tipos distintos de valuaciones: decimos que una valuacion es no arquimideana si 
\begin{equation}
	\vert a+b\vert\leq\max\lbrace\vert a\vert,\vert b\vert\rbrace.
\end{equation}
\noindent y por lo tanto diremos que una valuacíon es arquimideana en otro caso (observamos que la condición de no arquimidenidad es equivalente a que $C=1$).\\

Las valuaciones definen una norma en el campo $K$ y por lo tanto una distancia dada por $\vert x-y\vert$, así diremos que dos valuaciones son equivalentes si definen la misma topología en $K$, más precisamente, dos valuaciones $\nu_1,\nu_2$ son equivalentes si existen \hbox{$\lbrace c_1,c_2\rbrace\subset\re^{+}$} tales que
$$c_1\,\nu_1\leq \nu_2	\hspace{1cm}	c_2\,\nu_2\leq\nu_1$$
\noindent a las clases de equivalecia bajo esta relación les llamaremos \textit{lugares}.

\begin{teorema}[Ostrowski]
En $\rac$ sólo existen dos tipos distintos de lugares, el lugar del valor absoluto usual en $\rac$, $\vert\cdot\vert$ y los lugares no arquimideamos correspondientes a las valuaciones $p$-ádicas $\vert\cdot\vert_p$ para $p$ primo definidas por
\begin{equation}
	\vert u\,p^n\vert_p:=p^{-n},\hspace{0.3cm}\text{donde}\hspace{0.5cm}u\in\rac,\,u=\frac{a}{b},\,(a,p)=(b,p)=1.
\end{equation}
\noindent A estas valuaciones también les llamamos \textbf{finitas}. 
\end{teorema}

\begin{def.}
Un campo global es un campo $K$ dentro de las siguientes dos categorias distintas
\begin{itemize}
	\item $K$ es una extensión finita de $\rac$ (también llamdo campo numérico)
	\item $K$ es una estensión finita y separable de $\mathbb{F}(\tau)$, con $\mathbb{F}$ un campo finito y $\tau$ es tracendental sobre $\mathbb{F}$, es decir $\mathbb{F}(\tau)\cong Frac(\mathbb{F}[X])$.
\end{itemize}
\end{def.}

\noindent\textit{Nota.-} Dada una valuación $\nu$ en un campo $K$, denotamos por $K_{\nu}$ al campo correspondiente a la completación métrica de $K$.

\begin{def.}[Anillo de valuación]
Decimos que un dominio entero $R$ es un anillo de valuación si para su campo de fracciones $F=Frac(R)$ sucede que para todo $x\in F$ sucede que $x\in R$ o $x^{-1}\in R$.\\

Para $K$ un campo y $\nu$ una valuación no arquimideana, el anillo de valuación de la completación $K_{\nu}$ es'ta dado por
\begin{equation}
	\Oan_{\nu}:=\lbrace x\in K_{\nu}\mid \vert x\vert\leq 1\rbrace.
\end{equation}
\end{def.}

En general para una valuación $\nu$ denotaremos por $\Oan_{\nu}$ a su anillo de valuación, en caso de las valuaciones p-ádicas escribimos $\Oan_{p}$.

\begin{def.}[Anillo de adeles]
Sea $K$ un campo global, definimos el anillo de adeles de $K$ como el subanillo de $\prod_{\nu} K_{\nu}$ definido por el producto topológico restringido
\begin{equation}
	\Ad_{K}:=\prod_{\nu\,\text{valuación}}(K_{\nu},\Oan_{\nu})\subset\prod_{\nu} K_{\nu}
\end{equation}
\noindent definido como el conjunto de $(a_\nu)$ tales que $a_{\nu}\in\Oan_{\nu}$ salvo un número finito de lugares. La estructura de anillo se define entrada a entrada
$$(x\,y)_{\nu}:=x_{\nu}\,y_{\nu}\hspace{0.5cm}(x+y)_{\nu}:=x_{\nu}+y_{\nu}$$
\noindent a un elemento invertible de $\Ad_{K}$ le llamaremos idele.
\end{def.}
Observamos que existe un encaje de $K$ en $\Ad_{K}$ por medio de $\varphi:K\rightarrow\Ad_{K}$ $\varphi(x)=(x,x,x,\cdots)$
\subsection{Espacios adélicos y variedades algebraicas}

\noindent Sea $V$ una variedad algebraica definida sobre un campo $K$ y sea $L$ una extensión de $K$, denotamos por $V_L$ a los puntos de $L$  que son racionales sobre $K$, es decir los ceros de polinomios con coeficientes en $K$ con coordenadas en $L$. Sabemos que $V$ admite una cubierta finita de abiertos de Zariski, es decir que $V$ se puede cubrir con un número finito de variedades afines definidas sobre $K$, así

$$V=\bigcup_i\varphi_i(V_i),$$  
\noindent donde $V_i$ es una variedad afín $V_i=\specKn/I$ y $\varphi_i$ es un isomorfimo entre $V_i$ y un abierto de $V$. Así tenemos que

$$V_L=\bigcup_i\varphi_i(V_{i,L}),$$
\noindent en particular para $\nu$ una valuacion en $K$ y $K_{\nu}$ con la $\nu$ topología, tenemos que 

$$V_{K_{\nu}}=\bigcup_i\varphi_i(V_{i,K_{\nu}})$$
\noindent es localmente compacto.Por lo tanto  definimos

\begin{equation}
	[V,\varphi_i,V_i]_{\Oan_{\nu}}=\bigcup_i\varphi_i(V_{i,\Oan_{\nu}}),
\end{equation} 
\noindent donde $V_{i,\Oan_{\nu}}$ es el subconjunto compacto de $V_{K_{\nu}}$ con coordenadas en $\Oan_{\nu}$ y así $[V,\varphi_i,V_i]_{\Oan_{\nu}}$ es un subconjunto compacto de $V_{K_{\nu}}$, con esto definimos el espacio adélico asociado a $V$ como el producto topológico restringido
\begin{equation}
	V_{\Ad_{K}}:=\prod_{\nu\,\text{valuación}}(V_{K_{\nu}},[V,\varphi_i,V_i]_{\Oan_{\nu}})\subset\prod_{\nu}V_{K_{\nu}}.
\end{equation}
Cuando $V$ es \textbf{completo}, se puede demostrar que $V_{\Ad_{K}}$ y $V_{K_{\nu}}$ son compactos y por lo tanto $[V,\varphi_i,V_i]_{\Oan_{\nu}}=V_{\Oan_{\nu}}$

\begin{teorema}\label{morf_adel}
Sean $V=\bigcup_i \varphi_i(V_i)$ y $W=\bigcup_j \psi_j(W_j)$ variedades algebraicas definidas sobre $K$ y sea $F:V\rightarrow W$ un morfismo de variedades algebraicas definido sobre $K$, entonces existe un conjunto finito de valuaciones tal que $F$ mapea $[V,\varphi_i,V_i]_{\Oan_{\nu}}$ en $[W,\psi_j,W_j]_{\Oan_{\nu}}$. 
\end{teorema}
\noindent\dem Para demostrar esto es suficiente considerar el caso cuando $V$ es afín ya que podemos continuar el argumento a una cubierta de variedades afines. Para cada $j$ sea \hbox{$F_j=\psi^{-1}_j\circ F:V\rightarrow W_j$}, ahora $F$ es representable por funciones racionales $R_{j\,k}$ en coordenadas de $V$, donde $1\leq k\leq dim(W_j)$. Sea $\mathfrak{a}_j\subset K[X]$ el ideal consistente de todos los polinomios $A(x)$ tales que

$$A(x)\,R_{j\,k}(x)=Q_{k}(x)\in K[x],$$
\noindent para todo $k$ con $x$ punto genérico de $V$ sobre $K$, entonces es claro que $F_j$ está definido en $x_1\in V$ si y sólo si no es un cero de $\mathfrak{a}_j$. Como $F$ está bien definido en todo $V$, entonces por lo menos un $F_j$ está definido en un $x_1$ cualquiera, lo que significa que que no existe un cero común de $\sum_j\,\mathfrak{a}_j$, es decir

$$(1)=\sum_j\,\mathfrak{a}_j.$$
\noindent Por lo tanto podemos escribir $1=\sum_j\,A_j$ con 
$$A_j(x)R_{j\,k}(x)=Q_{j\,k}(x)\in K[x].$$
\noindent Sea $S$ el conjunto de todas las $\nu$ tal que algún coeficiente de $\lbrace A_j,Q_{j\,k}\rbrace$ no sea $\nu$ entero. Para cualquier $x_1\in V_{\Oan_{\nu}}$,  fuera de $S$, entonces $A_j(x_1)$ es una $\nu$-unidad para algún $j=j_1$ y asi $R_{j_1\,k}(x_1)=Q_{j_1\,k}(x_1)/A_{j_1}(x_1)$ está en $\Oan_{\nu}$ para toda $k$.\qed 
\begin{cor}
Si $F:V\rightarrow W$ es un isomorfismo, entonces 

$$F([V,\varphi_i,V_i]_{\Oan_{\nu}})=[W,\psi,W_j]_{\Oan_{\nu}}$$
\noindent para casi toda $\nu$.
\end{cor}

Con esto notamos que la definición de $[V,\varphi_i,V_i]_{\Oan_{\nu}}$ es casi intrínseca y que si aplicamos el resultado con la identidad en $V$ obtenemos que la definición de $V_{\Ad_{K}}$ es independiente de la la cubierta afín (coordenadas). Además de lo anterior, obtenemos que para cada morfismo $F:V\rightarrow W$ determina una función continua \hbox{$F_{\Ad_{K}}:V_{\Ad_{K}}\rightarrow W_{\Ad_{K}}$}.   

De lo anterior es posible definir un functor 
\begin{equation}
	\mathcal{A}:{\mathcal{V}ar}_K\rightarrow\Ad_{K}\,Spc
\end{equation}
Entre la categoría de variedades algebraicas sobre $K$ y la categoría de espacios adélicos. Es claro que si $F:V\rightarrow W$ y $G:W\rightarrow X$, entonces $\mathcal{A}(F):=F_{\Ad_{K}}$ es functorial, es decir \hbox{$(G\circ F)_{\Ad_{K}}=G_{\Ad_{K}}\circ F_{\Ad_{K}}$}. Si $V$ es una subvariedad de $W$, entonces el mapeo inclusión $\iota:V\rightarrow W$, entonces se puede demostrar que al aplicarle el fuctor $\mathcal{A}(\iota)=\iota_{\Ad_{K}}$ es un encaje cerrado. Además sucede que \hbox{$(V\times W)_{\Ad_{K}}\cong V_{\Ad_{K}}\times W_{\Ad_{K}}$}, por lo tanto el functor $\mathcal{A}$ tiene varias propiedades interesante para su estudio, por ejemplo  existe la siguiente condición que asegura la suprayectividad de \hbox{$F_{\Ad_{K}}:V_{\Ad_{K}}\rightarrow W_{\Ad_{K}}$}

\begin{teorema}\label{supra}
Sea $F:V\rightarrow W$ un morfismo de variedades algebraicas definidas sobre $K$. Si sucede que para cada $p\in W$ existe $\phi_p:W\rightarrow V$ morfismo racional sobre $K$ tal que $F\circ\phi_p=id_W$ (i.e $\phi_p$ es una sección local de $F$), entonces \hbox{$F_{\Ad_{K}}:V_{\Ad_{K}}\rightarrow W_{\Ad_{K}}$} es suprayectiva. 
\end{teorema}   
\noindent\dem Para cada $p\in W$ sea $\phi_p$ la función mencionada en la declaración del teorema y $D(\phi_p)$ el abierto de $W$ donde esta definida. Así tenemos una cubierta abierta de $W$  dada por $\lbrace D(\phi_p)\rbrace_{p\in W}$, los cuales por la racionalidad de $\phi_p$ son isomorfos a variedades afines. Así podemos escribir $W$ como una unión finita (Noeterianidad) $W=\bigcup_j\,\psi_j(W_j)$ con $W_j$ variedad afin tal que en cada $\psi_j(W_j)$ hay una sección global $\psi_j$. Así definimos \hbox{$G_j=\phi_j\circ\psi_j:W_j\rightarrow V$} el cual es un morfismo que cumple $F\circ G_j=\psi_j$, así por el teorema \ref{morf_adel} existe un subconjunto finito $S$ de valuaciones tal que 

$$G_j(W_{j,\Oan_{\nu}})\subset[V,\varphi_i,V_i]_{\Oan_{\nu}}\hspace{0.2cm}\forall\,j$$
siempre y cuando $\nu$ esté fuera de $S$. Sea $p=(p_{\nu})\in W_{\Ad_{K}}$ con $p_{\nu}\in\psi_{j(\nu)}(W_{j(\nu)})$, por definición de $W_{\Ad_{K}}$ exite un conjunto finito $S\subset T$ tal que $p_{\nu}\in\psi_{j(\nu)}(W_{j(\nu)})$ para $\nu$ fuera de $T$, así definimos $q=q_{\nu}=\psi_{j(\nu)}(p_{\nu})$, así si $\nu$ no está en $T$, entonces 

$$q_{\nu}\in\psi_{j(\nu)}(W_{\nu\,\Oan_{\nu}})\subset[V,\varphi_i,V_i]_{\Oan_{\nu}},$$
\noindent así $q_{\nu}$ está bien definido y $F(q_{\nu})=p_{\nu}$, es decir $F_{\Ad_{K}}(q)=p$ \qed 
\begin{obs}
Este teorema es inmediatamente aplicable al caso de acciones de grupos algebraicos. Sea $X=H/G$ donde $G$ es un grupo algebraico y $H$ es una variedad algebraica (ambas sobre $K$), entonces el teorema \ref{supra} implica que en muchos casos al aplicar el functor $\mathcal{A}$ a $\pi:H\rightarrow X$, obtendemos una función suprayectiva $\pi_{\Ad_{K}}:H_{\Ad_{K}}\rightarrow H_{\Ad_{K}}/G_{\Ad_{K}}$. 
\end{obs}

\chapter{Restricciones de Weil}
\noindent Sea $L/K$ una extensión separable de grado $d$ sobre $K$ y sean $\Ad_{K}$, $\Ad_L$ sus respectivos anillos de adeles. Toda valuación $\omega$ en $L$ determina una valuación $\nu$ en $K$ por medio de restricción, esto lo denotamos por $\omega/\nu$. Como la extensión de $K$ es de grado $d$, hay a lo más $d$ valuaciones $\omega$ tal que $\omega/\nu$. Además podemos identificar $K_{\nu}$ como la cerradura de $K$ en $L_{\omega}$, para valuaciones discretas sucede que \hbox{$\Oan_{\nu}=K_{\nu}\cap\Oan_{\omega}$}, también es claro que el mapeo \hbox{$(a_{\nu})\mapsto(a_{\omega})$} con $\omega/\nu$ es una inyección $\Ad_K\hookrightarrow\Ad_L$. 

Supongmos ahora que $L/K$ es normal con grupo de Galois $Gal(L/K)=\ga$, entonces la acción de $\ga$ en $L$ es continua en la $\nu$-topología, como $L$ es denso en 

$$\prod^m_{i=1}L_{\omega_i}\hspace{0.2cm}\text{donde}\,\,\,\omega_i/\nu,$$

\noindent podemos extender continuamente la acción a este producto y por lo tanto podemos extender la acción continuamente a todo $\Ad_{L}$. Notamos que $\Ad_K$ son los puntos invariantes de $\Ad_{L}$ bajo la acción de $\ga$. Además si una variedad algebraica $V$ está definida sobre $K$, también está definida sobre $L$ y por lo tanto $V_{\Ad_K}$ se encaja canonicamente en $V_{\Ad_{L}}$. Si $L$ es normal sobre $K$, el grupo de Galois $\ga$ actúa de forma natural en $V_{\Ad_{L}}$ y claramente $V_{\Ad_K}$ es el conjunto de puntos invarintes bajo la acción de $\ga$.

Ahora nuestro propósito para definir la reducción de escalares, es encontrar una variedad $W$ sobre $K$ tal que para dada una variedad $V$ sobre $L$ tengamos que $V_{\Ad_L}\cong W_{\Ad_K}$ de forma canónica, para hacer esto usaremos una contrucción de tipo algebraico-geométrica.

Sean $V$ y $W$ variedades definidas sobre $L$ y $K$ respectivamente ($L/K$ no necesariamente normal). Sea $\varphi:W\rightarrow V$ un mapeo sobre $L$ y sea $\Sigma:=\lbrace\sigma_1,\cdots,\sigma_d\rbrace$, el conjunto de encajes de $L$ en $\overline{K}$ (cerradura algebraica), con esto podemos definir $\varphi^{\sigma_i}:W\rightarrow V^{\sigma_i}$ para toda $i\in\lbrace 1,\cdots,d\rbrace$, donde $V^{\sigma_i}$ es la imagen de $V$ bajo $\sigma_i$, vista como subconjunto de $V_L$, así podemos definir

$$
	(\varphi^{\sigma_1},\cdots,\varphi^{\sigma_d}):W\rightarrow V^{\sigma_1}\times\cdots\times V^{\sigma_d}
		\hspace{0.3cm}
			w\mapsto(\varphi^{\sigma_i}(w))_{i\in\lbrace 1,\cdots,d\rbrace}.
$$
Si este mapeo es un encaje llamamos al par $(W,\varphi)$ como \textit{la variedad obtenida de $V$ por restricción de $L$ a $K$} y lo denoratemos por $(W,\varphi)=Res_{L/K}(V)$ o $W=Res_{L/K}(V)$.

Demostramos que este espacio es \textbf{único} debido a que la restricción tiene la siguiente propiedad universal:

Sea $X$ una variedad algebraica sobre $K$ y sea $f:X\rightarrow V$ un morfimos definido sobre $L$, entonces existe un único $\psi:X\rightarrow Res_{L/K}(V)$ definido sobre $K$ tal que $f=\varphi\circ\phi$, de hecho

\begin{equation}
	\begin{tikzcd}
    	X \arrow{d}{\psi} \arrow{r}{f} & V \\
    	Res_{L/K}(V) \arrow{ur}{\varphi} & {}
	\end{tikzcd}
		\hspace{0.3cm}\phi=(\varphi^{\sigma_1},\cdots,\varphi^{\sigma_{d-1}})^{-1}\circ(f^{\sigma_1},\cdots,f^{\sigma_{d}}),
\end{equation}
\noindent por lo tanto  $\phi$ esta definida sobre $K$ y es única. La \textbf{existencia} se sigue del siguiente teorema

\begin{teorema}
Sean $V$ y $Res_{L/K}(V)=(W,\varphi)$ como antes, sea también $V'$ definido sobre $L$. Si $V'$ es una subvaridad algebraica o un abierto de Zariski de $V$, entonces existe $Res_{L/K}(V')=(W',\varphi')$. Además si $V_1$ y $V_2$ tienen restricciones $(W_1,\varphi_1)$ y $(W_2,\varphi)$ respectivamente, entonces

$$(W_1\times W_2,\,\varphi_1\times\varphi_2)=Res_{L/K}(V_1\times V_2)$$  
\end{teorema}
Si $V$ tiene estructuras adicionales como la de grupo, entonces el morfimo $\varphi$ y $Res(V)_{L/K}$ preservan dicha estructura. Por ejemplo sea $V=G_m:=L^*$ el grupo multiplicativo de unidades de $L$ uno dimensional, entonces $W=Res(V)_{L/K}$ es un grupo de dimensión $d=[L:K]$ definido sobre $K$. La multiplicación esta definida sobre $K$ como la multiplicación en $L*$ vista como una tranformación $K$-lineal.  

\chapter{Superficies modulares de Hilbert y espacios adélicos}
\noindent Sea $K$, una extensión finita de $\rac$, denotamos por $S_f$ al conjunto de lugares finitos de $K$, respectivamente $S_{\infty}$ será el cunjunto de lugares infinitos de $K$, denotamos $\Ad_K$ al anillo de adeles de $K$ y por $\Ad_K^*$ al grupo de ideles de $K$. El grupo de ideles contiene al grupo de unidades $K^*$, definimos el \textit{idele class group} de $K$ como 
\begin{equation}
	\mathcal{C}l(\Ad_K):=\Ad_K^*/K^*.
\end{equation}
Resulta que podemos recuperar el \textit{class group} clásico de elemento absolutamente positivos
$$
	\mathcal{C}l(\Ad_K)\,\Big(\prod_{p\in S_f}\Oan_p\Big)
	\Big(\prod_{\nu\in S_{\inf}}\re^{+}\Big)\cong
	\mathcal{C}l^{+}(K).
$$
Sea $G=Res_{K/\rac}(GL(2,K))$ la restricción sobre $\rac$ del grupo matrices invertibles $2\times2$ con coeficientes en $K$, así definimos 

$$G(\re)=\prod_{\nu\in S_{\infty}}GL(2,\re),$$
además definimos el morfismo de grupos $h_0:\co^*\rightarrow G(\re)$ definido por

\begin{equation}
	x+iy\mapsto
	\bigg(
		\begin{pmatrix}
			x & -y\\
			y & x
		\end{pmatrix},
		\cdots,
		\begin{pmatrix}
			x & -y\\
			y & x
		\end{pmatrix}		
	\bigg).
\end{equation}
\noindent El centralizador de $h_0$ en $G(\re)$ es
$$
	K_{\infty}=
	\Bigg\lbrace
		\bigg(
		\begin{pmatrix}
			x_1 & -y_1\\
			y_1 & x_1
		\end{pmatrix},
		\cdots,
		\begin{pmatrix}
			x_n & -y_n\\
			y_n & x_n
		\end{pmatrix}
		\bigg)\,
		\bigg\vert\,
		\lbrace x_i,y_i\rbrace\subset\re\,\forall i\in\lbrace 1,\cdots,n\rbrace		
	\Bigg\rbrace.
$$
\noindent Notamos que $K_{\infty}$ es un subgrupo conexo de $G(\re)$ isomorfo al tangente del toro $n$-dimensional. Más aún el cociente $G(\re)/K_{\infty}$ es una variedad real de dimensión $2n$ con $2^n$ componentes conexas permutadas por 
$$
	\pi_0(G(\re))=G(\re)/G(\re)^0\cong(\dbz/2\,\dbz)^n.
$$
Podemos darle a $G(\re)/K_{\infty}$ una estructura de variedad compleja por medio de la siguiente identificación
$$
G(\re)/K_{\infty}\rightarrow(\co\setminus\re)^n\hspace{0.3cm}g\mapsto(g_1(i),\cdots,g_n(i)).
$$
\noindent La acción de $\epsilon_j\in\pi_0(G(\re))$ es la conjugación compleja en la coordenada $j$-ésima, donde
$$
	\epsilon_j=
			\bigg(
			Id_2,
			\cdots,
			\begin{pmatrix}
				-1 & 0\\
				0 & 1
			\end{pmatrix},
			\cdots,
			Id_2
			\bigg).
$$
Denotamos por $\Ad^{fin}_K$ al subanillo de $\Ad_K$ de adeles finitos, es decir que sólo se consideran las valuaciones no arquimideanas en el producto reducido. Así tenemos que
$$
G(\Ad_K)/K_{\infty}=(G(\re)/K_{\infty})\times G(\Ad^{fin}_K).
$$
\noindent Este cociente también tiene una estuctura compleja debido a la acción izquierda de $G_{fin}=G(\Ad^{fin}_K)$ en $G(\Ad)/K_{\infty}$.

El grupo $G(\Ad_K)$ es un grupo topológico con la topología que determina que el subbgrupo
$$
	\prod_{\nu\in S_f}GL(2,\Oan_{\nu})\times G(\re)^0,
$$
sea abierto. Ahora consideramos un subgrupo compacto $C_{fin}$ de $G_{fin}$ y considermos el cociente
$$
	X_{C_{fin}}:=G(\rac)\backslash G(\Ad_K)/K_{\infty}C_{fin}.
$$
Para anallizar este cociente utilizamos el mapeo $G(\Ad_K)\rightarrow\Ad^*_{K}$ definido por el determinante en cada entrada, esto induce el mapeo
$$
	G(\rac)\backslash G(\Ad_K)/K_{\infty}C_{fin}\rightarrow\Ad^*_K/K^*\,det(G(\re)^0C_{fin}).
$$
Donde escogemos $\lbrace g_1,\cdots,g_m\rbrace\subset G_{fin}$ tal que $\lbrace det(g_i)\rbrace$ forma un conjunto completo de representantes de $\Ad^*_K/K^*\,det(G(\re)^0C_{fin})$, entonces
\begin{teorema}
\begin{equation}
G(\Ad_K)=\bigcup_{j=1}^m G(\rac)g_jG(\re)^0K_{fin}
\end{equation}
\end{teorema}
\begin{teorema}
Podemos identificar:
\begin{equation}
	G(\rac)\backslash G(\Ad_K)/K_{\infty}C_{fin}=\bigcup^m_{j=1}\ga_j\backslash\hip^n
\end{equation}
donde $\ga_j=g_j\big(G(\re)^0K_{fin}\big)g_j^{-1}\cap G(\rac)$ 
\end{teorema}
\dem Como sabemoss por el teorema anterior
\begin{align}
	X_{C_{fin}}:&=G(\rac)\backslash G(\Ad_K)/K_{\infty}C_{fin}\\
				&=G(\rac)\bigg\backslash\bigcup_{j=1}^mG(\rac)g_jG(\re)^0C_{fin}/K_{\infty}C_{fin}\\
				&=\bigcup^m_{j=1}g_j\big(G(\re)^0C_{fin}\big)g_j^{-1}\cap G(\rac)\backslash(G(\re)^0/K_{\infty})\\
				&=\bigcup^m_{j=1}\ga_j\backslash\hip^n.
\end{align}\qed\\
Si hacemos esto en el caso de $C_0=\prod_{\nu\in S_{fin}}GL(2,\Oan_{\nu})$
\begin{cor}
$G(\rac)\backslash G(\Ad_K)/K_{\infty}C_0$ se puede identificar con
$$
\bigcup_{\mathfrak{a}}\ga(\mathfrak{a}\oplus\Oan_K)\backslash\hip^n,
$$
\noindent donde $\mathfrak{a}$ corre sobre todos los representantes en $\mathcal{C}l^{+}(K)$.
\end{cor}
\dem Sabemos que los componentes de $G(\rac)\backslash G(\Ad_K)/K_{\infty}C_0$ están en correspondencia uno a uno con $\Ad_K^*/K^*det(G(\re)^0C_o)\cong\mathcal{C}l^{+}(K)$ (igual que el caso real) y el resultado se sigue del teorema anterior.\qed 

Esto explica el por qué consideramos todas las superficies $\ga(\mathfrak{a}\oplus\Oan)/\hip^2$, en gran parte consideramos propiedades geométricas de $G(\rac)\backslash G(\Ad_K)/K_{\infty}C_0$, sin embargo al considerar las propiedades aritméticas de este espacio es imperativo hablar de adeles.

Sean $C_1$ y $C_2$ subgrupos abiertos y compactos de $G(\Ad_K^{fin})$ y sea $g\in G(\Ad_K^{fin})$ tal que $g^{-1}C_1g\subset C_2$, entonces existe un morfismo \textbf{natural}
\begin{equation}
	I_{C_1\,C_2}:X_{C_1}\rightarrow X_{C_2},
\end{equation} 
dado por la multiplicación a la derecha por $g$ que cumple
\begin{itemize}
	\item[i)] $I_{C_3\,C_2}(h)I_{C_2\,C_2}(g)=I_{C_3\,C_1}(gh)$ para toda $g,h$
	\item[ii)] $I_{C_1\,C_1}=Id_{X_{C_1}}$
	\item[iii)] Si $C_1\subset C_2$ es un subgrupo normal entonces $C_2/C_1$ actúa en $X_{C_1}$ y $I_{C_1\,C_2}(1)$ induce un isomorfismo

		$$X_{C_{1}}/(C_2/C_1)\cong X_{C_{2}}.$$
\end{itemize}

\begin{thebibliography}{99}
\bibitem{stein}{\sc Stein, W.,} {\it Algebraic Number Theory a Computaional Approach}
\bibitem{weil}{\sc Weil, A.,} {\it Adeles and Algebraic Groups,1982 Birkhauser}
\bibitem{vandergeer}{\sc Van Der Geer, G.,}{\it Hilbert Modular Surfaces, Springer-Verlag}
\end{thebibliography}
\end{document}
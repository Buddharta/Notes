\input exfeti.sty

% 20 de oct. 1989
\newbox\Ancha
\def\gros#1{{\setbox\Ancha=\hbox{$#1$}
   \kern-.025em\copy\Ancha\kern-\wd\Ancha
   \kern.05em\copy\Ancha\kern-\wd\Ancha
   \kern-.025em\raise.0433em\box\Ancha}}
%\def\gros#1{\vec #1}
\font\titulo=cmbxti10 scaled 5000
\font\biggfnt=cmr10 scaled\magstep 3

\font\bigfnt=cmr10 scaled\magstep 1
\def\ni{\noindent}
\def\esp{\par \vskip .2 in }
\baselineskip 18pt \nopagenumbers

\def\ni{\noindent}
\baselineskip 15pt \magnification 1200
\def\Par{\par\vskip .15in}
\def\esp{\Par \vskip .2 in }
%----------------------------------empieza--------------------------------------

 \centerline{\bigggfnt F\'{\i}sica Moderna III}\Par
\centerline{{\bf Profesor:}  {\it Rodolfo P. Mart\'{\i}nez y Romero}}\Par

{\baselineskip 8pt \centerline{\it Examen 1}\par
\centerline{\small \hoy}
\Par

%-----------------------------------figura---------------------------------------------

%\newbox\z
\gdef\fetifig#1#2#3{\setbox1=\vbox{
%%Empieza figura
\let\picnaturalsize=N
\def\picsize{#3}
\def\picfilename{#1}
%If you do not have the picture file add:
%\let\nopictures=Y
%to the beginning of the file.
\ifx\nopictures Y\else{\ifx\epsfloaded Y\else\input epsf \fi
\global\let\epsfloaded=Y \hskip 2cm{\ifx\picnaturalsize
N\epsfxsize \picsize\fi \epsfbox{\picfilename}}}\fi
%%Termina figura
 \setbox\z=\hbox{}
\copy\z}
 \setbox2=\vbox{\vskip
4pt\splittopskip=\baselineskip\hsize 5cm {\hskip -60pt
{\titulo#2}}} \centerline{$\hskip -3cm\vcenter{\box1}\hskip
-4.8cm\vcenter{\box2}\hfill$} }
%\input epsf.sty
\vskip -125pt 
\fetifig{unamN1.eps}{\titulo}{2.5cm}


\vskip 30pt

\libro
\ni  En este examen se estudiar\'a la colisi\'on de protones  que inciden sobre otros  protones.\Par
\ni 1).  Calcular el impulso m\'{\i}nimo del prot\'on   necesario para producir   la reacci\'on $P +P\to P+P+P+P,$ sobre protones en reposo (Fig. 1). La masa del prot\'on es $m_p =$ 938 Mev.

\input epsf

%%Begin InstantTeX Picture
\let\picnaturalsize=N
\def\picsize{9cm}
\def\picfilename{ModIII_II_1_2002.eps}
%If you do not have the picture file add:
%\let\nopictures=Y
%to the beginning of the file.
\ifx\nopictures Y\else{\ifx\epsfloaded Y\else\input epsf \fi
\global\let\epsfloaded=Y
\centerline{\ifx\picnaturalsize N\epsfxsize \picsize\fi \epsfbox{\picfilename}}}\fi
%%End InstantTeX Picture
\centerline{\small Fig. 1}\Par


\ni 2). El mismo problema que arriba, s\'olo que ahora el prot\'on blanco se mueve con hacia el prot\'on incidente con un impulso de 150 Mev/c, (Fig. 2). {\it Nota: Este problema es independiente del anterior.}\Par





%%Begin InstantTeX Picture
\let\picnaturalsize=N
\def\picsize{9cm}
\def\picfilename{ModIII_II_1_2002(2).eps}
%If you do not have the picture file add:
%\let\nopictures=Y
%to the beginning of the file.
\ifx\nopictures Y\else{\ifx\epsfloaded Y\else\input epsf \fi
\global\let\epsfloaded=Y
\centerline{\ifx\picnaturalsize N\epsfxsize \picsize\fi \epsfbox{\picfilename}}}\fi
%%End InstantTeX Picture
\centerline{\small Fig. 2}\Par


\ni 3). (Un regalito). Probar las desigualdades siguientes: 
$$\matrix{s\geq (m_1 + m_2)^2\quad \hbox{y}\quad
s\geq (m_3 + m_4)^2\cr
          t\leq (m_1 - m_3)^2\quad \hbox{y}\quad
 t\leq (m_2 - m_4)^2\cr
          u\leq (m_1 - m_4)^2\quad \hbox{y}\quad
u\leq (m_2 - m_3)^2\cr}
$$


 \vfill
\eject
\end









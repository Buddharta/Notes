\input exfeti.sty

% 20 de oct. 1989
\newbox\Ancha
\def\gros#1{{\setbox\Ancha=\hbox{$#1$}
   \kern-.025em\copy\Ancha\kern-\wd\Ancha
   \kern.05em\copy\Ancha\kern-\wd\Ancha
   \kern-.025em\raise.0433em\box\Ancha}}
%\def\gros#1{\vec #1}
\font\titulo=cmbxti10 scaled 5000
\font\biggfnt=cmr10 scaled\magstep 3

\font\bigfnt=cmr10 scaled\magstep 1
\def\ni{\noindent}
\def\esp{\par \vskip .2 in }
\baselineskip 18pt \nopagenumbers

\def\ni{\noindent}
\baselineskip 15pt \magnification 1100
\def\Par{\par\vskip .15in}
\def\esp{\Par \vskip .2 in }
%----------------------------------empieza--------------------------------------

 \centerline{\biggfnt Funciones
Especiales}\par\centerline{\slll y}\par
 \centerline{\biggfnt Transformadas Integrales}\Par

{\baselineskip 8pt \centerline{\it Examen 5}\par
\centerline{\small \hoy}
\Par

%-----------------------------------figura---------------------------------------------

%\newbox\z
\gdef\fetifig#1#2#3{\setbox1=\vbox{
%%Empieza figura
\let\picnaturalsize=N
\def\picsize{#3}
\def\picfilename{#1}
%If you do not have the picture file add:
%\let\nopictures=Y
%to the beginning of the file.
\ifx\nopictures Y\else{\ifx\epsfloaded Y\else\input epsf \fi
\global\let\epsfloaded=Y \hskip 2cm{\ifx\picnaturalsize
N\epsfxsize \picsize\fi \epsfbox{\picfilename}}}\fi
%%Termina figura
 \setbox\z=\hbox{}
\copy\z}
 \setbox2=\vbox{\vskip
4pt\splittopskip=\baselineskip\hsize 5cm {\hskip -60pt
{\titulo#2}}} \centerline{$\hskip -3cm\vcenter{\box1}\hskip
-4.8cm\vcenter{\box2}\hfill$} }
%\input epsf.sty
\vskip -150pt 
\fetifig{unamN4.eps}{\titulo}{2.5cm}


\vskip 70pt
\libro

\ni 1) En este problema pueden escoger para resolver el inciso a) o el b). En N dimensiones $ (l_1,\dots,l_N),$ el jacobiano en coordenadas polares $ (L, \phi, \theta_1\dots\theta_{N-2})$  se escribe 
$d^Nl =\break L^{N-1} dL\,d\phi\, \hbox{sen} \theta_1d\theta_1 \,\hbox{sen}^2\theta_2d\theta_2\dots\hbox{sen}^{N-2} \theta_{N-2}d\theta_{N-2},$ en donde $L = \sqrt{l_1^2 + \dots + l_N^2}$,  $ 0\leq L\leq \infty$, $0<\phi<2\pi$ y $0<\theta_i<\pi$. \par
a) Prueben que la integral $I_N = \int d^Nl F(L^2) $ sobre todo el espacio para una funci\'on  $F(L^2)$ que solamente depende del m\'odulo de $L^2$, da

$$ I_N= {\pi^{N/2} \over \Gamma(N/2)} \int_0^\infty L^{N-1}F(L^2)$$

b) En estas integrales es usual que  $F(x) = (x + a^2)^{-A}$, en donde $x=L^2$. Prueben entonces que $I_N$ es

$$ I_N=\pi^{N/2} {\Gamma(A-N/2)\over \Gamma(A)} {1\over (a^2)^{A-N/2}}  $$

\ni {\it Nota} Este tipo de integrales se usan con frecuencia en teor\'{\i}a del campo en la tecnica llamada {\it regularizaci\'on dimensional,} en donde N se convierte en una variable continua (incluso imaginaria).

\ni 2) La aproximaci\'on de {\it Bloch--Gruneissen} para el c\'alculo de la resistencia el\'ectrica $\rho$ en un metal monovalente es 

 $$\rho= C{T^5\over \Theta^6} \int_0^{\Theta/T}{x^5\over(e^x-1)\,(1-e^{-x})}dx.$$

\ni donde $C$ es una constante y $\Theta$ es la llamada temperatura de Debye, la cual depende del material. Demostrar que en el l\'{\i}mite cuando $T\to0$, $\rho$ vale
$$  \rho = 5!\zeta(5) C{T^5\over \Theta^6}.$$

\ni 3) Probar que la $\zeta(s)$ de Riemann se puede poner como  

$$   \zeta(s)=-{(-s)!\over 2\pi i} \int_{\cal C} {(-z)^{s-1} \over e^z -1} \,dz,$$

\ni en donde $ \cal C$ es la trayectoria de la figura 



\input epsf.sty
%%Begin InstantTeX Picture
\let\picnaturalsize=N
\def\picsize{5cm}
\def\picfilename{FETI00_1.eps}
%If you do not have the picture file add:
%\let\nopictures=Y
%to the beginning of the file.
\ifx\nopictures Y\else{\ifx\epsfloaded Y\else\input epsf \fi
\global\let\epsfloaded=Y
\centerline{\ifx\picnaturalsize N\epsfxsize \picsize\fi \epsfbox{\picfilename}}}\fi
%%End InstantTeX Picture

 
\ni 4) {\it Un regalito.} Probar que 
$$  \lim_{n\to \infty} {(2n-1)\,!!\over (2n)\,!!} n^{1/2}= \pi^{-1/2}. $$


\ni {\bf Algunas f\'ormulas \'utiles.} \Par
\ni La funci\'on {\it Gama:}
$$z! = \int_0^\infty t^z e^{-t}dt .$$
\ni La funci\'on {\it Zeta de Riemann:}
$$  \zeta(s) = \sum_{k=1}^\infty{1\over k^s}.$$
\ni La funci\'on {\it Beta:}
$$  B(z + 1, w +1) = 2 \int_0^{\pi/2}\cos^{2z + 1}\theta\,\sen^{2w+1}\theta\,d\theta= \int_0^\infty {u^z\over (1+u)^{z+w+2}}du = {z!\,w!\over (w+z+1)!}.$$
\ni  F\'ormulas de {\it duplicaci\'on de Legendre} y para factoriales de la forma$ (-z)!z!  $:
$$ \eqalign{ z\,!(z-{1\over 2} )\,! =& 2^{-2z} \pi^{1/2} (2z)\,!,\cr(-z)!z! =& {\pi z\over \hbox{\rm sen}\pi z}.}$$
\ni F\'ormula de {\it Stirling.} 

$$z! \approx \sqrt{2\pi} z^{z+1/2}e^{-z}  $$

\vfill
\eject
\end
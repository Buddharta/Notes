\input exfeti.sty
% 20 de oct. 1989
\newbox\Ancha
\def\gros#1{{\setbox\Ancha=\hbox{$#1$}
   \kern-.025em\copy\Ancha\kern-\wd\Ancha
   \kern.05em\copy\Ancha\kern-\wd\Ancha
   \kern-.025em\raise.0433em\box\Ancha}}
%\def\gros#1{\vec #1}
\font\biggfnt=cmr10 scaled\magstep 3
\font\bigfnt=cmr10 scaled\magstep 1
\def\ni{\noindent}
\def\esp{\par \vskip .2 in }
\baselineskip 18pt
\magnification 1200
\nopagenumbers

\def\ni{\noindent}  
\baselineskip 15pt    
\magnification 1200 
\def\Par{\par\vskip .15in} 
\def\esp{\Par \vskip .2 in }
 

%%Begin InstantTeX Picture
\let\picnaturalsize=N
\def\picsize{11cm}
\def\picfilename{FETI01sem_I_1.eps}
%If you do not have the picture file add:
%\let\nopictures=Y
%to the beginning of the file.
\ifx\nopictures Y\else{\ifx\epsfloaded Y\else\input epsf \fi
\global\let\epsfloaded=Y
{\ifx\picnaturalsize N\epsfxsize \picsize\fi \epsfbox{\picfilename}}}\fi
%%End InstantTeX Picture

%\centerline{\bigfnt FETI}\par

\centerline{\sl Examen 1}\hfill{\smal \hoy} \Par
\ni 1) Calcular la integral siguiente
 $$  \int_0^\infty {x^s\over(e^x-1)}dx = s!\zeta(s + 1)$$
\ni 2) Demostrar la siguiente relaci\'on  
$$ \int_{-1}^1 (1-x^2)^{1/2}x^{2n}dx = \cases{{\pi\over 2}& Si $n=0$\cr
\cr
\pi {(2n-1)!!\over (2n +2)!!}  & Si $n= 1,2,3...$}$$
\ni 3) Probar que en la expansi\'on de $ (1+x)^{-1/2} $ el coeficiente $a_n$ de la expansi\'on se puede escribir ya sea en terminos de factoriales o de de dobles factoriales como sigue
$$  a_n=(-1)^n{(2n)!\over 2^{2n} (n!)^2} = (-1)^n{2n -1)!!\over (2n)!!},\quad n= 1,2,3\dots $$

\ni 4)  Probar que para las funciones $ P^m_m(x)\equiv (2m-1)!!\,(1-x^2)^{m/2} $, llamadas en la literatura {\it Polinomios asociados de Legendre,}  se cumplen las relaciones

$$ \eqalign{&\int_{-1}^1[P_m^m(x)]^2\,dx = {2\over (2m +1) }(2m)!\cr
&\int_{-1}^1[P_m^m(x)]^2{1\over 1-x^2}\,dx = 2(2m -1)!\cr}$$


\ni {\bf Algunas f\'ormulas \'utiles.} \par
\ni La funci\'on {\it gama}
$$z! = \int_0^\infty t^z e^{-t}dt $$
\ni La funci\'on {\it zeta de Riemann.}
$$  \zeta(s) = \sum_{k=1}^\infty{1\over k^s}$$
\ni La funci\'on {\it beta}
$$  B(m + 1, n +1) = 2 \int_0^{\pi/2}\cos^{2m + 1}\theta\,\sen^{2n+1}\theta\,d\theta= {m!\,n!\over (m+n+1)!}$$

\vfill
\eject
\end
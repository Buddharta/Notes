
\input exfeti.sty

% 20 de oct. 1989
\newbox\Ancha
\def\gros#1{{\setbox\Ancha=\hbox{$#1$}
   \kern-.025em\copy\Ancha\kern-\wd\Ancha
   \kern.05em\copy\Ancha\kern-\wd\Ancha
   \kern-.025em\raise.0433em\box\Ancha}}
%\def\gros#1{\vec #1}
\font\titulo=cmbxti10 scaled 5000
\font\biggfnt=cmr10 scaled\magstep 3

\font\bigfnt=cmr10 scaled\magstep 1
\def\ni{\noindent}
\def\esp{\par \vskip .2 in }
\baselineskip 18pt \nopagenumbers

\def\ni{\noindent}
\baselineskip 15pt \magnification 1100
\def\Par{\par\vskip .15in}
\def\esp{\Par \vskip .2 in }
%----------------------------------empieza--------------------------------------

 \centerline{\biggfnt Funciones
Especiales}\par\centerline{\slll y}\Par
 \centerline{\biggfnt Transformadas Integrales}\Par

{\baselineskip 8pt \centerline{\it Examen 2}\par
\centerline{\smalll \hoy}
\Par

%-----------------------------------figura---------------------------------------------

%\newbox\z
\gdef\fetifig#1#2#3{\setbox1=\vbox{
%%Empieza figura
\let\picnaturalsize=N
\def\picsize{#3}
\def\picfilename{#1}
%If you do not have the picture file add:
%\let\nopictures=Y
%to the beginning of the file.
\ifx\nopictures Y\else{\ifx\epsfloaded Y\else\input epsf \fi
\global\let\epsfloaded=Y \hskip 2cm{\ifx\picnaturalsize
N\epsfxsize \picsize\fi \epsfbox{\picfilename}}}\fi
%%Termina figura
 \setbox\z=\hbox{}
\copy\z}
 \setbox2=\vbox{\vskip
4pt\splittopskip=\baselineskip\hsize 5cm {\hskip -60pt
{\titulo#2}}} \centerline{$\hskip -3cm\vcenter{\box1}\hskip
-4.8cm\vcenter{\box2}\hfill$} }
%\input epsf.sty
\vskip -160pt 
\fetifig{unamN4.eps}{\titulo}{2.5cm}


\vskip 70pt
\libro
\ni 1) Generalizaci\'on de los Polinomios de Hermite a \'{\i}ndice no entero. Prueben que 

$$  {\nu! \over 2\pi i}\int_c {e^{-z^2 + 2zx}\over z^{\nu + 1}}dz ,$$

\ni es solucion de la ecuaci\'on diferencial de Hermite: $H_\nu^{"} -2xH_\nu^{'} + 2\nu H_\nu . $ Aqu\'{\i} $c$ es la trayectoria, recorrida en el sentido positivo siguiente

\input epsf

%%Begin InstantTeX Picture
\let\picnaturalsize=N
\def\picsize{5cm}
\def\picfilename{FETI01sem_II_2.eps}
%If you do not have the picture file add:
%\let\nopictures=Y
%to the beginning of the file.
\ifx\nopictures Y\else{\ifx\epsfloaded Y\else\input epsf \fi
\global\let\epsfloaded=Y
\centerline{\ifx\picnaturalsize N\epsfxsize \picsize\fi \epsfbox{\picfilename}}}\fi
%%End InstantTeX Picture

%\vskip -50pt
\ni 2) Prueben que la integral el\'{\i}ptica de segunda clase $E(k^2)$ se puede escribir como:

$$ E\,(k^2)= (1-k^2)\int_0^{\pi/2} (1-k^2\hbox{\rm sen}^2\theta)^{-3/2} \, d\theta  .$$



\ni 3) Probar que  la funci\'on $B_x(a,b) = \int_0^x t^{a-1} (1-t)^{b-1}\, dt$
es igual a: 

$$ B_x(a,b) =a^{-1} x^a {}_2\hskip -1ptF_1(a, 1- b;a +1 ;x) . $$


\ni 4) Probar que 

$$ \int_{-\infty }^\infty x e^{-x^2} H_n(x) H_m(x) dx = \sqrt{\pi} \left[  2^m(m+1)!\delta_{n,m+1} + 2^{m-1}  m! \delta_{n,m-1}\right]$$






\ni {\bf F\'ormulas, f\'ormulas....} \Par

\ni Hipergeom\'etrica:
$$ {}_2\hskip -1ptF_1(a,b;c;x) =\sum_{n=0}^\infty {(a)_n (b)_n\over n!\,(c)_n}x^n. $$

\ni Hipergeom\'etrica confluente:
$${}_1\hskip -1ptF_1(a;c;x) =\sum_{n=0}^\infty {1\over n!}{(a)_n \over \,(c)_n}x^n  .$$

\ni Integral el\'{\i}ptica de 2\hskip -2pt$^{\rm nda.}$ clase
$$E\,(x) = \int_0^{\pi/2} (1-x\, \hbox{\rm sen}^2\,\theta)^{1/2}={\pi\over 2}\,{}_2\hskip -1pt F_1({1\over 2}, {-1\over 2};1;x) .$$

\ni Relaciones de recurrencia para los {\it Polinomios de Hermite}:
$$ \eqalign{H_{n+1} (x) &= 2x H_n(x) -2n H_{n-1}(x),\cr
H_n'(x)&= 2nH_{n-1}(x).} $$


\vfill \eject
\end

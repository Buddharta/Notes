\input exfeti.sty

% 20 de oct. 1989
\newbox\Ancha
\def\gros#1{{\setbox\Ancha=\hbox{$#1$}
   \kern-.025em\copy\Ancha\kern-\wd\Ancha
   \kern.05em\copy\Ancha\kern-\wd\Ancha
   \kern-.025em\raise.0433em\box\Ancha}}
%\def\gros#1{\vec #1}
\font\titulo=cmbxti10 scaled 5000
\font\biggfnt=cmr10 scaled\magstep 3

\font\bigfnt=cmr10 scaled\magstep 1
\def\ni{\noindent}
\def\esp{\par \vskip .2 in }
\baselineskip 18pt \nopagenumbers

\def\ni{\noindent}
\baselineskip 15pt \magnification 1200
\def\Par{\par\vskip .15in}
\def\esp{\Par \vskip .2 in }
%----------------------------------empieza--------------------------------------

 \centerline{\bigggfnt F\'{\i}sica Moderna III}\Par
\centerline{{\bf Profesor:}  {\it Rodolfo P. Mart\'{\i}nez y Romero}}\Par

{\baselineskip 8pt \centerline{\it Examen 2}\par
\centerline{\small \hoy}
\Par
\centerline{\it PROYECTOR DE ESP\'IN}

%-----------------------------------figura---------------------------------------------

%\newbox\z
\gdef\fetifig#1#2#3{\setbox1=\vbox{
%%Empieza figura
\let\picnaturalsize=N
\def\picsize{#3}
\def\picfilename{#1}
%If you do not have the picture file add:
%\let\nopictures=Y
%to the beginning of the file.
\ifx\nopictures Y\else{\ifx\epsfloaded Y\else\input epsf \fi
\global\let\epsfloaded=Y \hskip 2cm{\ifx\picnaturalsize
N\epsfxsize \picsize\fi \epsfbox{\picfilename}}}\fi
%%Termina figura
 \setbox\z=\hbox{}
\copy\z}
 \setbox2=\vbox{\vskip
4pt\splittopskip=\baselineskip\hsize 5cm {\hskip -60pt
{\titulo#2}}} \centerline{$\hskip -3cm\vcenter{\box1}\hskip
-4.8cm\vcenter{\box2}\hfill$} }
%\input epsf.sty
\vskip -125pt 
\fetifig{unamN1.eps}{\titulo}{2.5cm}


\vskip 40pt

%\n

\ni En este examen estudiaremos el operador llamado {\it  proyector de esp\'{\i}n }\Par
\ni 1) Definamos para un vector $a^\mu$ la cantidad $ a\hbox{\hskip- 5pt /} \equiv \gamma^\mu a_\mu$ (Que leeremos aqu\'{\i} entre nosotros  como {\it a slash}). Prueben que el producto de matrices $ a\hbox{\hskip- 5pt /} a\hbox{\hskip- 5pt /}  $ es igual a

$$  a\hbox{\hskip- 5pt /} a\hbox{\hskip- 5pt /} = a^\mu \gamma_\mu a^\nu \gamma_\nu=  \tilde{a}\cdot \tilde{a} =\tilde{a}^2.$$

\ni 2) Definamos el {\it vector de proyecci\'on del esp\'{\i}n $ s^\mu $}  con $\tilde{s}^2 = -1,$ de tal manera que en el centro de masa (CM) $s^\mu\hskip -4pt\mid_{\hbox{\smalll CM}} \equiv (0,0,0,1).$ Definamos ahora al proyector de esp\'{\i}n como
$$  P(s) \equiv \left( {1 +s\hbox{\hskip- 5pt /}\gamma_5 \over 2}\right).$$

Prueben entonces que 

$$ \eqalign{a)&\qquad  P(s)  + P(-s) = 1,\cr  b)& \qquad P(s) P(-s) =0,\cr
c)& \qquad  P^2(s)  = P(s).  \cr} $$

\ni 3) Prueben que en el Centro de Masa:

$$ P(s)\hskip -4pt \mid_{\hbox{\smalll CM}} = {1 + \sigma_z\gamma^0\over 2}. $$

\ni 4) Definamos al espinor de energ\'{\i}a positiva y  proyecci\'on de esp\'{\i}n a lo largo del eje $z,$ hacia arriba, $u(p,{1/ 2}),$ como $w^1(p),$ y al de energ\'{\i}a positiva pero esp\'{\i}n para abajo $u(p,-{1/ 2})$ como $w^2(p).$ Verifiquen que $\gamma^0 u(p,\pm s ) = u(p,\pm s).$ Entonces para los estados de energ\'{\i}a positiva,  el proyector de esp\'{\i}n $P(s)$ se puede escribir como $  P(s)\hskip -4pt \mid_{\hbox{\smalll CM}} = {1 + \sigma_z\over 2}$ y tenemos que 

$$ \eqalign{P(s) w^1 =w^1,  \quad P(-s)w^2&=w^2.\cr
P(s) w^2 =0,  \quad P(-s)w^1 &=0.\cr} $$

\ni Digan ahora qu\'e pasa para los espinores de energ\'{\i}a negativa. Como el esp\'{\i}n no es un buen n\'umero cu\'antico, en general un espinor no estar\'a en un estado de esp\'{\i}n definido, sin embargo, se puede proyectar ese estado a un estado de esp\'{\i}n definido usando el proyector $P(s)$.

\ni 
\vfill
\eject
\end









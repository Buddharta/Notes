\input exfeti.sty

% 20 de oct. 1989
\newbox\Ancha
\def\gros#1{{\setbox\Ancha=\hbox{$#1$}
   \kern-.025em\copy\Ancha\kern-\wd\Ancha
   \kern.05em\copy\Ancha\kern-\wd\Ancha
   \kern-.025em\raise.0433em\box\Ancha}}
%\def\gros#1{\vec #1}
\font\titulo=cmbxti10 scaled 5000
\font\biggfnt=cmr10 scaled\magstep 3

\font\bigfnt=cmr10 scaled\magstep 1
\def\ni{\noindent}
\def\esp{\par \vskip .2 in }
\baselineskip 20pt \nopagenumbers

\def\ni{\noindent}
\baselineskip 25pt \magnification 1200
\def\Par{\par\vskip .15in}
\def\esp{\Par \vskip .2 in }
%----------------------------------empieza--------------------------------------

 \centerline{\bigggfnt F\'{\i}sica Moderna III}\Par
\centerline{{\bf Profesor:}  {\it Rodolfo P. Mart\'{\i}nez y Romero}}\Par

{\baselineskip 8pt \centerline{\it Examen 2}\par
\centerline{\small \hoy}
\Par
\centerline{\it Vector de polarizaci\'on}

%-----------------------------------figura---------------------------------------------

%\newbox\z
\gdef\fetifig#1#2#3{\setbox1=\vbox{
%%Empieza figura
\let\picnaturalsize=N
\def\picsize{#3}
\def\picfilename{#1}
%If you do not have the picture file add:
%\let\nopictures=Y
%to the beginning of the file.
\ifx\nopictures Y\else{\ifx\epsfloaded Y\else\input epsf \fi
\global\let\epsfloaded=Y \hskip 2cm{\ifx\picnaturalsize
N\epsfxsize \picsize\fi \epsfbox{\picfilename}}}\fi
%%Termina figura
 \setbox\z=\hbox{}
\copy\z}
 \setbox2=\vbox{\vskip
4pt\splittopskip=\baselineskip\hsize 5cm {\hskip -60pt
{\titulo#2}}} \centerline{$\hskip -3cm\vcenter{\box1}\hskip
-4.8cm\vcenter{\box2}\hfill$} }
%\input epsf.sty
\vskip -135pt 
\fetifig{unamN4.eps}{\titulo}{2.5cm}


\vskip 45pt

%\n

\ni 1) Calcular la corriente para una onda plana de un electr\'on 
libre de energ\'ia positiva y de impulso ${\bf k}$
%
$$ u({\bf k}) = \sqrt{{E+m_{{}_0}\over 2E}}\pmatrix{ \chi \cr {}\cr
{{\gros\sigma}\cdot {\bf k}\over E+m_{{}_0}}\chi \cr}$$
%
en donde $\chi$ es un bi-espinor ($\hbar=c=1$).\Par

\ni 2) Calcular el conmutador con el hamiltoniano $H={\gros\alpha}\cdot 
{\bf p}+{\beta}m_{{}_0}$ del operador $ {\gros \sigma}\cdot \hat n$, 
en donde ${\hat n}$ es un vector unidad. Para simplificar los c\'alculos 
se puede tomar $\hat n$ paralelo a $ {\bf k}$.\Par
\ni 3) Misma pregunta que en 2 para el operador
$\beta {\gros \sigma}\cdot\hat n$. \Par

\ni 4) Definiremos el vector de polarizaci\'on ${\bf P}$ como:
%
$${\bf P}= {\gros \sigma}_{\|}+\beta{\gros \sigma}_{\bot}$$
en donde
%
$${\gros \sigma}_{\|}={({\gros \sigma}\cdot{\bf k}){\bf k}\over |{\bf k}|^2}$$
y
$${\gros \sigma}_{\bot}= {\gros \sigma}-{\gros \sigma}_{\|}.$$
%
Mostrar que ${\bf P}$  conmuta con el hamiltoniano $H.$\par 
%\ni 5) Calcular $ {\bf P}\cdot{\hat n}$ y ${\bf P}^2.$ 
%?`Se pueden diagonalizar simult\'aneamente $ H,{\bf P}_{{}_x},
%{\bf P}_{{}_y}$ y ${\bf P}_{{}_z}$?\par
\vfill
\eject
\end



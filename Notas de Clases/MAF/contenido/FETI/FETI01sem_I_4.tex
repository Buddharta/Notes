
\input exfeti.sty

% 20 de oct. 1989
\newbox\Ancha
\def\gros#1{{\setbox\Ancha=\hbox{$#1$}
   \kern-.025em\copy\Ancha\kern-\wd\Ancha
   \kern.05em\copy\Ancha\kern-\wd\Ancha
   \kern-.025em\raise.0433em\box\Ancha}}
%\def\gros#1{\vec #1}
\font\titulo=cmbxti10 scaled 5000
\font\biggfnt=cmr10 scaled\magstep 3

\font\bigfnt=cmr10 scaled\magstep 1
\def\ni{\noindent}
\def\esp{\par \vskip .2 in }
\baselineskip 18pt \nopagenumbers

\def\ni{\noindent}
\baselineskip 15pt \magnification 1100
\def\Par{\par\vskip .15in}
\def\esp{\Par \vskip .2 in }
%----------------------------------empieza--------------------------------------

 \centerline{\biggfnt Funciones
Especiales}\par\centerline{\slll y}\Par
 \centerline{\biggfnt Transformadas Integrales}\Par

{\baselineskip 8pt \centerline{\it Examen 4}\par
\centerline{\smalll \hoy}
\Par

%-----------------------------------figura---------------------------------------------

%\newbox\z
\gdef\fetifig#1#2#3{\setbox1=\vbox{
%%Empieza figura
\let\picnaturalsize=N
\def\picsize{#3}
\def\picfilename{#1}
%If you do not have the picture file add:
%\let\nopictures=Y
%to the beginning of the file.
\ifx\nopictures Y\else{\ifx\epsfloaded Y\else\input epsf \fi
\global\let\epsfloaded=Y \hskip 2cm{\ifx\picnaturalsize
N\epsfxsize \picsize\fi \epsfbox{\picfilename}}}\fi
%%Termina figura
 \setbox\z=\hbox{}
\copy\z}
 \setbox2=\vbox{\vskip
4pt\splittopskip=\baselineskip\hsize 5cm {\hskip -60pt
{\titulo#2}}} \centerline{$\hskip -3cm\vcenter{\box1}\hskip
-4.8cm\vcenter{\box2}\hfill$} }
%\input epsf.sty
\vskip -160pt 
\fetifig{unamN1.eps}{\titulo}{2.5cm}


\vskip 70pt
\libro
\ni 1.a) Mostrar que la expansi\'on de Fourier de $\hbox{cos}\,x$ es: 

  $$ \hbox{cos}\,ax={2a\hbox{sen}\,a\pi \over \pi}\left[{1\over 2a^2} + \sum_{m=1}^\infty{(-1)^m\hbox{cos}\,mx\over (a^2 - m^2)}\right].$$

\ni b) A partir de este resultado, prueben que,

$$  a\pi \hbox{cot}\, a\pi = 1-2\sum_{p=1}^\infty \zeta(2p)a^{2p}.$$


\ni 2) Muestren que la $\delta(x-a)$ de Dirac expandida como una serie {\it seno } de Fourier en el  medio intervalo $[0,L]$ se escribe como

$$  \delta(x-a) = {2\over L} \sum_{n=1}^\infty \hbox{\rm sen} \left( {n\pi a\over L}\right)\hbox{\rm sen} \left( {n\pi x\over L}\right).$$

\ni Notar que en realidad esta serie describe a $-\delta(x+a) + \delta(x-a)$ en el intervalo completo $[-L,L]$. 

 
\ni 3) Prueben que de la  expansi\'on de Fourier de $f(x)= x$ en el intervalo $- \pi<x<\pi$ se obtiene al integrar, el resultado,

$$  {\pi^2\over 12} = \sum_{n=1} ^\infty {(-1)^{n+1}\over n^2}.$$ 

 
\ni 4) Tenemos una cuerda detenida de los extremos en $x=0$ y $x=L$. La ecuaci\'on que satisface la amplitud de vibraci\'on $y(x,t)$ est\'a dada por la ecuaci\'on de onda:
$$ {\partial^2 y\over \partial x^2} = {1\over v^2}{\partial^2 y\over \partial t^2}, $$

\ni en donde $v$ es la velocidad de propagaci\'on de una onda que se propague en la cuerda. Supongamos que en $x=a$ $(0<a<L)$ jalamos la cuerda y la soltamos inmediatamente. Entonces la ecuaci\'on de onda debe satisfacer las condiciones, 

$$ y(x,t=0)=0\quad\hbox{y}\quad {\partial y (x,t=0)\over \partial t}  = Lv_0\delta(x-a).$$

\ni Demuestren que la soluci\'on a la ecuaci\'on de onda sujeta a las condiciones mencionadas arriba es:

$$ y(x,t)= {2v_0L\over\pi v} \sum_{n=1}^\infty{1\over n}\, \hbox{\rm sen} \left( {n\pi a\over L}\right)\hbox{\rm sen} \left( {n\pi x\over L}\right)\hbox{\rm sen} \left( {n\pi vt\over L}\right).$$

\ni Aqu\'{\i} conviene usar el resultado del problema 2. Adem\'as noten que para cada $t$ la soluci\'on buscada se debe poder poner como una serie de Fourier con coeficientes dependientes del tiempo.
\Par
\vfill
\eject
\ni {\bf F\'ormulas, f\'ormulas...} \Par
\ni Serie de Fourier en el intervalo $-\pi\leq x \leq \pi$

$$f(x) = {a_0\over 2} + \sum_{m=1}^\infty a_m\hbox{cos}\,mx \, +  \sum_{m=1}^\infty b_m\hbox{sen}\,mx,$$

\ni en donde 

$$ a_0 = {1\over \pi}\int_{-\pi}^\pi f(x)\, dx,\quad a_m = {1\over \pi} \int_{-\pi}^\pi f(x)\,\hbox{cos}\,mx\, dx,\quad  b_m = {1\over \pi}
\int_{-\pi}^\pi f(x)\,\hbox{sen}\,mx\, .$$


\ni La funci\'on {\it Zeta de Riemann:}
$$  \zeta(s) = \sum_{k=1}^\infty{1\over k^s},\qquad \zeta(2)= {\pi^2\over 6}.$$

\vfill \eject
\end

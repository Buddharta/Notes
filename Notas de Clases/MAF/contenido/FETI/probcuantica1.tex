\documentstyle[12pt]{article}
\hoffset=-1.5cm
\textwidth=17cm

\begin{document}

\section{Problema \# 1}

\bigskip 

(1) Al tiempo $t=0\;$la funcion de onda  del \'atomo de hidr\'ogeno es
\[
\Psi (\vec{r},0)=A\left( 2\psi _{100}+\psi _{210}+\sqrt{2}\psi _{211}+\sqrt{3%
}\psi _{21-1}\right) 
\]%
en la notaci\'on usual y donde $A\;$es un coeficiente por determinar.

$\;\;\;\;$(a) ?` Cu\'al es la energ\'\i a del sistema?.

\ \ \ \ (b) Encuentre $\Psi (\vec{r},t)$

\ \ \ \ (c) ?`Cu\'al es la probabilidad al tiempo $t\;$de hallar al sistema con $%
l=1,$\ $m=+1$\ ?

\ (2) Suponga que una medici\'on nos dice que el \'atomo se encuentra en un
estado de mpmento angular con  $L=1$ y $L_{x}=+1$.  Escriba la funci\'on de onda inmediatamente
despues de la medici\'on, en t\'erminos de las funciones $\psi _{nlm}.$

\subsection{\protect\bigskip SOLUCION}

Primero determinamos $A\;$por normalizaci\'on. Las $\psi _{nlm}\;$est\'an
ortonormalizadas por lo que%
\[
1=|A|^{2}\left( 4+1+2+3\right) =10|A|^{2}\Longrightarrow A=\frac{1}{\sqrt{10}%
}
\]

a) 
\begin{eqnarray*}
E &=&<\Psi |H|\Psi > \\
&=&\frac{1}{10}<\left( 2\psi _{100}+\psi _{210}+\sqrt{2}\psi _{211}+\sqrt{3}%
\psi _{21-1}\right) |\times  \\
&&\left( \left( E_{1}2\psi _{100}+E_{2}\psi _{210}+E_{2}\sqrt{2}\psi
_{211}+E_{2}\sqrt{3}\psi _{21-1}\right) \right)  \\
E &=&\frac{1}{10}\left( 2E_{1}+6E_{2}\right) =\frac{1}{5}(E_{1}+3E_{2})
\end{eqnarray*}%
donde%
\[
E_{n}=-\frac{me^{4}}{2\hbar ^{2}}\frac{1}{n^{2}}
\]%
\[
E=-\frac{me^{4}}{2\hbar ^{2}}\frac{1}{5}\left( 1+\frac{3}{4}\right) =-\frac{7%
}{40}\frac{me^{4}}{\hbar ^{2}}
\]

(b)%
\begin{eqnarray*}
\Psi (\vec{r},t) &=&e^{-i\frac{H}{\hbar }t}\Psi (\vec{r},0) \\
\Psi (\vec{r},t) &=&\frac{1}{\sqrt{10}}\left( 2e^{-i\frac{E_{1}}{\hbar }%
t}\psi _{100}+e^{-i\frac{E_{2}}{\hbar }t}\left( \psi _{210}+\sqrt{2}\psi
_{211}+\sqrt{3}\psi _{21-1}\right) \right) 
\end{eqnarray*}

(c)%
\[
P_{l=1,m=1}=\sum_{n}|<\psi _{n11}|\Psi (\vec{r},t)>|^{2}
\]%
Como $l=n-1,n-2,...0\;\;$contribuye solo \ $n=2$, por lo que 
\begin{eqnarray*}
P_{l=1,m=1} &=&|<\psi _{211}|\Psi (\vec{r},t)>|^{2} \\
P_{l=1,m=1} &=&\frac{1}{10}|e^{-i\frac{E_{2}}{\hbar }t}\sqrt{2}|^{2}=\frac{2%
}{10}=0.2
\end{eqnarray*}


(2) Como $L=1$,  entonces la nueva funci\'on de onda ser\'a 
\begin{eqnarray*}
\Phi  &=&a\psi _{211}+b\psi _{210}+c\psi _{21-1} \\
&=&R_{21}\left( aY_{11}+bY_{10}+cY_{1-1}\right) 
\end{eqnarray*}%
normalizada adecuadamente.%
\[
L_{x}=\frac{1}{2}\left( L_{+}+L_{-}\right) 
\]%
con%
\begin{eqnarray*}
L_{+}Y_{lm} &=&\sqrt{\left( l+m+1\right) \left( l-m\right) }Y_{lm+1} \\
L_{-}Y_{lm} &=&\sqrt{\left( l-m+1\right) \left( l+m\right) }Y_{lm-1}
\end{eqnarray*}%
\begin{eqnarray*}
L_{+}Y_{11} &=&0,\;L_{+}Y_{10}=\sqrt{2}Y_{11},\;\;L_{+}Y_{1-1}=\sqrt{2}%
Y_{10}\; \\
L_{-}Y_{11} &=&\sqrt{2}Y_{10},\;L_{-}Y_{10}=\sqrt{2}Y_{1-1},\;%
\;L_{-}Y_{1-1}=0\;
\end{eqnarray*}%
La\ condicion es 
\[
\frac{1}{2}\left( L_{+}+L_{-}\right) \left( aY_{11}+bY_{10}+cY_{1-1}\right)
=\left( aY_{11}+bY_{10}+cY_{1-1}\right) 
\]%
\begin{eqnarray*}
RHS &=&\frac{1}{2}\left( L_{+}+L_{-}\right) \left(
aY_{11}+bY_{10}+cY_{1-1}\right)  \\
&=&\frac{1}{2}\left( L_{+}\right) \left( aY_{11}+bY_{10}+cY_{1-1}\right)  \\
&&+\frac{1}{2}\left( L_{-}\right) \left( aY_{11}+bY_{10}+cY_{1-1}\right)  \\
&=&\frac{\sqrt{2}}{2}\left( bY_{11}+cY_{10}\right) +\frac{\sqrt{2}}{2}\left(
aY_{10}+bY_{1-1}\right)  \\
&=&\frac{1}{\sqrt{2}}\left( bY_{11}+\left( a+c\right) Y_{10}+bY_{1-1}\right) 
\end{eqnarray*}%
Comparando coeficientes de la base%
\[
\frac{b}{\sqrt{2}}=a,\;\;\;\frac{\left( a+c\right) }{\sqrt{2}}=b,\;\frac{b}{%
\sqrt{2}}=c\;\;
\]%
\[
c=a,\;\;\;b=\sqrt{2}a
\]%
La normalizacion es%
\[
1=a^{2}+b^{2}+c^{2}=4a^{2}\Longrightarrow a=\frac{1}{2}
\]%
\[
\Phi =\frac{1}{2}\left( \psi _{211}+\sqrt{2}\psi _{210}+\psi _{21-1}\right) 
\]

\end{document}

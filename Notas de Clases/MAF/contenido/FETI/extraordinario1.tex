\input extrafeti.sty

%%Begin InstantTeX Picture
\let\picnaturalsize=N
\def\picsize{11cm}
\def\picfilename{FETI.eps}
%If you do not have the picture file add:
%\let\nopictures=Y
%to the beginning of the file.
\ifx\nopictures Y\else{\ifx\epsfloaded Y\else\input epsf \fi
\global\let\epsfloaded=Y
{\ifx\picnaturalsize N\epsfxsize \picsize\fi \epsfbox{\picfilename}}}\fi
%%End InstantTeX Picture

\magnification 1000
\centerline{\medio Profesor: Rodolfo P. Mart\'{\i}nez y Romero}\par
\centerline{\it Examen extrordinario}
\centerline{\small \hoy}\par
\smal
\centerline{\subt El examen consta de 5 preguntitas simples}\par

\ni 1) En mec\'anica cu\'antica, la dispersion de part\'{\i}culas por un potencial central, est\'a relacionada con la llamada {\smali amplitud de onda dispersada,} dada por
$ f(\theta) = k \sum_{l=0}^\infty (2l + 1) \,e^{i\delta_l}\sen\,\delta_l \,P_l(\cos\theta),$
\ni donde $l$ es el valor del momento angular orbital, $\delta_l$ es el corrimiento de fase producido por el potencial central sobre la onda de part\'{\i}culas del haz incidente y $\theta$ es el \'angulo de dispersi\'on. Si la secci\'on eficaz de dispersi\'on est\'a dada por $\int f^*(\theta)f(\theta)\, d\Omega,$ mostrar que
$$ \sigma_{\rm tot} = 4\pi k^2 \sum_{l=0}^\infty (2l + 1)\, \sen^2\delta_l.$$
\ni 2) Calcular la integral
$$ G(k)= \int_{-\infty}^\infty {e^{ikz}\over z^2 - m^2}\,dz$$
\ni para los casos a)$ k>0 $y b)$k<0,$ Tomando como referencia el contorno de la figura.  (Esta integral aparece como funci\'on de Green en teor\'{\i}a del campo).
\input epsf.sty

%%Begin InstantTeX Picture
\let\picnaturalsize=N
\def\picsize{5cm}
\def\picfilename{extraordinario.eps}
%If you do not have the picture file add:
%\let\nopictures=Y
%to the beginning of the file.
\ifx\nopictures Y\else{\ifx\epsfloaded Y\else\input epsf \fi
\global\let\epsfloaded=Y
\centerline{\ifx\picnaturalsize N\epsfxsize \picsize\fi \epsfbox{\picfilename}}}\fi
%%End InstantTeX Picture

\ni 3) Comparando el desarrollo en serie de Taylor de la funci\'on generadora, con su desarrollo en t\'erminos de los polinomios de {Legendre,} probar que 
$$P_n(\cos\,\theta) = {(-1)^n\over n!}r^{n + 1} {\partial^n\over \partial z^n}\left({1\over r}\right) .$$
\ni Aqu\'{\i}  puede ser \'util le figura siguiente:
\input epsf.sty

%%Begin InstantTeX Picture
\let\picnaturalsize=N
\def\picsize{5cm}
\def\picfilename{FETI598.eps}
%If you do not have the picture file add:
%\let\nopictures=Y
%to the beginning of the file.
\ifx\nopictures Y\else{\ifx\epsfloaded Y\else\input epsf \fi
\global\let\epsfloaded=Y
\centerline{\ifx\picnaturalsize N\epsfxsize \picsize\fi \epsfbox{\picfilename}}}\fi
%%End InstantTeX Picture

\ni 4) Probar la siguiente representaci\'on integral para la funci\'on de bessel $J_n(x)$
$$J_n^2(x)={1\over \pi}\int_0^\pi J_{2n}(2x\,\cos\alpha)\,d\alpha = {2\over \pi}\int_0^{\pi/2} J_{2n}(2x\cos\alpha)\,d\alpha.$$
\ni {\it Sugerencia:}  Una posibilidad es escribir a $J_{2n}(x)$ como una integral doble, digamos sobre $\theta$ y $\phi$ y hacer el cambio de variable $2\alpha=\theta - \phi\,\,;\,\, 2\beta=\theta + \phi.$ Comparar el resultado con alguna representaci\'on integral apropiada.
\par
\ni 5) La transformada seno de Fourier se define como

$$\eqalign{\tilde f_s(k) = &\sqrt{2\over \pi}\int_{0}^\infty f_s(t)\,\hbox{sen}\,
kt\, dt\cr
 f_s(x) = &\sqrt{2\over \pi}\int_{0}^\infty \tilde f_s(k)\,\hbox{sen}\,
kx\, dk }$$

\ni y una expresi\'on equivalente para la transformada coseno.
Muestren que la transformada de Fourier seno y coseno de $e^{-at}$
son, respectivamente,

$$ \eqalign{\tilde f_s(k) = &\sqrt{2\over \pi}{k\over k^2 +a^2}\cr
\tilde f_c(k) = &\sqrt{2\over \pi}{a\over k^2 +a^2}} $$

\ni Usando el teorema del residuo, encuentren las relaciones
inversas

$$  \eqalign{\int_{0}^\infty {k\,\hbox{sen}\,(kx)\over k^2+a^2}\,dk =
&{\pi\over 2}\,e^{-ax}\cr
 {}\cr
 \int_{0}^\infty
{\hbox{cos}\,(kx)\over k^2+a^2}\,dk = &{\pi\over 2 a}\,e^{-ax}}$$



\vfill \eject
\end

\vfill
\eject
\end
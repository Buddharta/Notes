% Created 2025-02-06 Thu 20:56
% Intended LaTeX compiler: pdflatex
\documentclass[letterpaper]{article}
\usepackage[utf8]{inputenc}
\usepackage[T1]{fontenc}
\usepackage{graphicx}
\usepackage{longtable}
\usepackage{wrapfig}
\usepackage{rotating}
\usepackage[normalem]{ulem}
\usepackage{amsmath}
\usepackage{amssymb}
\usepackage{capt-of}
\usepackage{hyperref}
\usepackage{lmodern} % Ensures we have the right font
\usepackage[utf8]{inputenc}
\usepackage{graphicx}
\usepackage{amsmath, amsthm, amssymb, amsfonts, amssymb, amscd}
\usepackage[table, xcdraw]{xcolor}
%\usepackage{mdsymbol}
\usepackage{tikz-cd}
\usepackage{float}
\usepackage[spanish, activeacute, ]{babel}
\usepackage{color}
\usepackage{transparent}
\graphicspath{{./figs/}}
\usepackage{makeidx}
\usepackage{afterpage}
\usepackage{array}
\usepackage{braket}
\usepackage{pst-node}
\newtheorem{teorema}{Teorema}[section]
\newtheorem{prop}[teorema]{Proposici\'on}
\newtheorem{cor}[teorema]{Corolario}
\newtheorem{lema}[teorema]{Lema}
\newtheorem{def.}{Definici\'on}[section]
\newtheorem{afir}{Afirmaci\'on}
\newtheorem{conjetura}{Conjetura}
\renewcommand{\figurename}{Figura}
\renewcommand{\indexname}{\'{I}ndice anal\'{\i}tico}
\newcommand{\zah}{\ensuremath{ \mathbb Z }}
\newcommand{\rac}{\ensuremath{ \mathbb Q }}
\newcommand{\nat}{\ensuremath{ \mathbb N }}
\newcommand{\prob}{\textbf{P}}
\newcommand{\esp}{\mathbb E}
\newcommand{\eje}{{\newline \noindent \sc \textbf{Ejemplo. }}}
\newcommand{\obs}{{\newline \noindent \sc \textbf{Observación. }}}
\newcommand{\dem}{{\noindent \sc Demostraci\'on. }}
\newcommand{\bg}{\ensuremath{\overline \Gamma}}
\newcommand{\ga}{\ensuremath{\gamma}}
\newcommand{\fb}{\ensuremath{\overline F}}
\newcommand{\la}{\ensuremath{\Lambda}}
\newcommand{\om}{\ensuremath{\Omega}}
\newcommand{\sig}{\ensuremath{\Sigma}}
\newcommand{\bt}{\ensuremath{\overline T}}
\newcommand{\li}{\ensuremath{\mathbb{L}}}
\newcommand{\ord}{\ensuremath{\mathbb{O}}}
\newcommand{\bs}{\ensuremath{\mathbb{S}^1}}
\newcommand{\co}{\ensuremath{\mathbb C }}
\newcommand{\con}{\ensuremath{\mathbb{C}^n}}
\newcommand{\cp}{\ensuremath{\mathbb{CP}}}
\newcommand{\rp}{\ensuremath{\mathbb{RP}}}
\newcommand{\re}{\ensuremath{\mathbb R }}
\newcommand{\hc}{\ensuremath{\widehat{\mathbb C} }}
\newcommand{\pslz}{\ensuremath{\mathrm{PSL}(2,\mathbb Z) }}
\newcommand{\pslr}{\ensuremath{\mathrm{PSL}(2,\mathbb R) }}
\newcommand{\pslc}{\ensuremath{\mathrm{PSL}(2,\mathbb C) }}
\newcommand{\hd}{\ensuremath{\mathbb H^2}}
\newcommand{\slz}{\ensuremath{\mathrm{SL}(2,\mathbb Z) }}
\newcommand{\slr}{\ensuremath{\mathrm{SL}(2,\mathbb R) }}
\newcommand{\slc}{\ensuremath{\mathrm{SL}(2,\mathbb C) }}
\newcommand{\mdlr}{\ensuremath{\mathrm{M}}}
\date{\today}
\title{Tarea 4}
\begin{document}

\maketitle

\textbf{Fecha de entrega: Sabado 15 de Marzo de 2025}

\begin{enumerate}
  \item Sea \(X=(\mathcal{C}[a,b],\|\cdot\|_{\infty})\) el espacio de \emph{Banach} de las funciones continuas en \([a,b]\) a \(\co\) con la norma del supremo (tambi'en llamada norma infinito) y sea \(Y=(\mathcal{C}[a,b],\|\cdot\|_{2})\) el mismo espacio pero ahora con la norma 2 o la norma dada por el producto interno tambi'en conocida como la norma de \emph{las funciones cuadrado inegrables}.
        \begin{itemize}
          \item[a)] Demuestre que el operador identidad \(I:X\to Y\) \(I\phi=\phi\) es inyectivo, supreyectivo y continuo. Calcule su norma.\\
            \emph{\textbf{NOTA: Tenga mucho cuidado con la noci'on de continuidad, f'ijese con mucho cuidado en las normas.}}
          \item[b)] Explique por qu'e es imposible que \(I:Y\to X\) sea continuo.\\
            \emph{Sugerencia: vea que propiedades tiene un espacio y el otro no.}
          \item[c)] Concluya que cualquier conjunto denso \(D\subset X\) es denso en \(Y\), en particular los polinomios trigonom'etricos.
        \end{itemize}
  \item Sea \(\mathcal{T}_{m}\) el espacio de polinomios trigonom'etricos de grado \(m\) y \(T\in\mathcal{T}_{m}\),
        \[
        T(x)=\frac{\alpha_{0}}{2}+\sum_{k=1}^{m}\alpha_{k}\cos(kx)+\beta_{k}\sin(kx)
        \]
        \begin{itemize}
          \item[a)] Muestre que \(s_{n}(T)(x)=T(x)\) para toda \(n\geq m\).
          \item[b)] Calcule \(\|f-T\|^{2}_{2}\).\\
                \emph{Sugerencia:Desarrolle el cuadrado, calcule el producto de \(f\) con \(T\) en t'erminos de los cofiecientes de Fourier de \(f\).}
          \item[c)] Concluya que
                \[
                \inf\{\|f-T\|^{2}_{2}\,:\,T\in\mathcal{T}_{m}\}=\|f-s_{m}(f)\|^{2}_{2}
                \]
        \end{itemize}
  \item Sea \(f:[-\pi,\pi]\to\co\) diferenciable, tal que
        \[
        f(x)=\sum_{n\in\zah}c_{n}e^{int},\quad f'(x)=\sum_{n\in\zah}c'_{n}e^{int}
        \]
        Calcule los \(c'_{n}\) en t'erminos de \(c_{n}\), justifique formalmente. Explique entonces que sucede con las series de \(\mathrm{Re}(f)\) e \(\mathrm{Im}(f)\).
  \item Demuestre que el kernels de  Dirichlet y de Fej'er complejos son
        \[
        D_{n}(x)=\sum_{k=-n}^{n}e^{ikx},\quad K_{n}(x)=\frac{1}{n}\sum_{k=-n}^{n}\left(1-\frac{|k|}{n}\right)e^{ikx}
        \]
        Respectivamente.
\end{enumerate}
\end{document}

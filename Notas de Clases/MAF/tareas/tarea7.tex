% Created 2025-02-06 Thu 20:56
% Intended LaTeX compiler: pdflatex
\documentclass[letterpaper]{article}
\usepackage[utf8]{inputenc}
\usepackage[T1]{fontenc}
\usepackage{graphicx}
\usepackage{longtable}
\usepackage{wrapfig}
\usepackage{rotating}
\usepackage[normalem]{ulem}
\usepackage{amsmath}
\usepackage{amssymb}
\usepackage{capt-of}
\usepackage{hyperref}
\usepackage{lmodern} % Ensures we have the right font
\usepackage[utf8]{inputenc}
\usepackage{graphicx}
\usepackage{amsmath, amsthm, amssymb, amsfonts, amssymb, amscd}
\usepackage[table, xcdraw]{xcolor}
%\usepackage{mdsymbol}
\usepackage{tikz-cd}
\usepackage{float}
\usepackage[spanish, activeacute, ]{babel}
\usepackage{color}
\usepackage{transparent}
\graphicspath{{./figs/}}
\usepackage{makeidx}
\usepackage{afterpage}
\usepackage{array}
\usepackage{braket}
\usepackage{pst-node}
\newtheorem{teorema}{Teorema}[section]
\newtheorem{prop}[teorema]{Proposici\'on}
\newtheorem{cor}[teorema]{Corolario}
\newtheorem{lema}[teorema]{Lema}
\newtheorem{def.}{Definici\'on}[section]
\newtheorem{afir}{Afirmaci\'on}
\newtheorem{conjetura}{Conjetura}
\renewcommand{\figurename}{Figura}
\renewcommand{\indexname}{\'{I}ndice anal\'{\i}tico}
\newcommand{\zah}{\ensuremath{ \mathbb Z }}
\newcommand{\rac}{\ensuremath{ \mathbb Q }}
\newcommand{\nat}{\ensuremath{ \mathbb N }}
\newcommand{\prob}{\textbf{P}}
\newcommand{\esp}{\mathbb E}
\newcommand{\eje}{{\newline \noindent \sc \textbf{Ejemplo. }}}
\newcommand{\obs}{{\newline \noindent \sc \textbf{Observación. }}}
\newcommand{\dem}{{\noindent \sc Demostraci\'on. }}
\newcommand{\bg}{\ensuremath{\overline \Gamma}}
\newcommand{\ga}{\ensuremath{\gamma}}
\newcommand{\fb}{\ensuremath{\overline F}}
\newcommand{\la}{\ensuremath{\Lambda}}
\newcommand{\om}{\ensuremath{\Omega}}
\newcommand{\sig}{\ensuremath{\Sigma}}
\newcommand{\bt}{\ensuremath{\overline T}}
\newcommand{\li}{\ensuremath{\mathbb{L}}}
\newcommand{\ord}{\ensuremath{\mathbb{O}}}
\newcommand{\bs}{\ensuremath{\mathbb{S}^1}}
\newcommand{\co}{\ensuremath{\mathbb C }}
\newcommand{\con}{\ensuremath{\mathbb{C}^n}}
\newcommand{\cp}{\ensuremath{\mathbb{CP}}}
\newcommand{\rp}{\ensuremath{\mathbb{RP}}}
\newcommand{\re}{\ensuremath{\mathbb R }}
\newcommand{\hc}{\ensuremath{\widehat{\mathbb C} }}
\newcommand{\pslz}{\ensuremath{\mathrm{PSL}(2,\mathbb Z) }}
\newcommand{\pslr}{\ensuremath{\mathrm{PSL}(2,\mathbb R) }}
\newcommand{\pslc}{\ensuremath{\mathrm{PSL}(2,\mathbb C) }}
\newcommand{\hd}{\ensuremath{\mathbb H^2}}
\newcommand{\slz}{\ensuremath{\mathrm{SL}(2,\mathbb Z) }}
\newcommand{\slr}{\ensuremath{\mathrm{SL}(2,\mathbb R) }}
\newcommand{\slc}{\ensuremath{\mathrm{SL}(2,\mathbb C) }}
\newcommand{\mdlr}{\ensuremath{\mathrm{M}}}
\date{\today}
\title{Tarea 7 (Tarea de los chavites)}
\begin{document}

\maketitle

\textbf{Fecha de entrega: Mi'ercoles 30 de Abril de 2025}

\begin{enumerate}
  \item El prop'osito de este problema es calcular la funci'on generadora de los polinomio de Hermite a partir de las ecuaciones recursivas. 
        \begin{itemize}
          \item[a)] En clase vimos que los polinomio de Hermite son solucion a la ecuaci'on $H_{n}''(x)-2xH_{n}(x)+2nH_{n}(x)=0,$ para cada $n\in\nat$ y se calculan como
\[
      H_{n}=\sum_{k=0}^{\lfloor n/2\rfloor}\frac{(-1)^{k}n!}{k!(n-2k)!}(2x)^{n-2k},\quad\lfloor\cdot\rfloor\text{ es la funci'on piso.}
\]
          A partir de la ecuaci'on de $H_{n}$ y su definici'on, demuestre que se cumplen las relaciones de recurrencia
          \begin{align*}
            H'_{n}(x)&=2nH_{n-1}(x)\\
            H_{n+1}(x)&=2xH_{}(x)-2nH_{n-1}(x).
          \end{align*}      
          \item[b)] Se define la \emph{funcion generadora} para la familia $\{H_{n}\}_{n\in\nat}$ como la \emph{serie formal}
          \[
                \mathcal{G}(x,z):=\sum_{n=0}\frac{H_{n}(x)}{n!}z^{n}=1+\sum_{n=1}\frac{H_{n}(x)}{n!}z^{n}
          \]
                Utilice las relaciones de recurrencia y derive formalmente (derive t'ermino a t'ermino) para mostrar que
          \begin{equation}\label{gen-herm}
                \dfrac{\partial\mathcal{G}}{\partial z}(x,z)=2x\mathcal{G}(x,z)-2z\mathcal{G}
          \end{equation}
                \emph{Hint:Antes de derivar, haga los cambios de 'indices y factorize de tal forma que al derivar se obtenga $\mathcal{G}$}
          \item[c)] Integre la ecuaci'on \ref{gen-herm} con respecto a $z$ para obtener la ecuaci'on generadora vista en clase.
        \end{itemize}
  \item Encuentre las \emph{funciones de Green de los siguientes problemas de frontera y ``resuelva'' las ecuaciones}. En caso de una funci'on abstracta, simplemente deje expresada la soluci'on en t'erinos de $f$. 
        \begin{enumerate}
          \item $y''=x^{2}$ en $[0,1]$ donde $y(0)=y(1)=0$
          \item $y''+k^{2}y=f(x)$ en $[a,b]$, donde $y(a)=y(b)=0$.
        \end{enumerate}
\end{enumerate}
\end{document}

% Created 2025-02-06 Thu 20:56
% Intended LaTeX compiler: pdflatex
\documentclass[letterpaper]{article}
\usepackage[utf8]{inputenc}
\usepackage[T1]{fontenc}
\usepackage{graphicx}
\usepackage{longtable}
\usepackage{wrapfig}
\usepackage{rotating}
\usepackage[normalem]{ulem}
\usepackage{amsmath}
\usepackage{amssymb}
\usepackage{capt-of}
\usepackage{hyperref}
\usepackage{lmodern} % Ensures we have the right font
\usepackage[utf8]{inputenc}
\usepackage{graphicx}
\usepackage{amsmath, amsthm, amssymb, amsfonts, amssymb, amscd}
\usepackage[table, xcdraw]{xcolor}
%\usepackage{mdsymbol}
\usepackage{tikz-cd}
\usepackage{float}
\usepackage[spanish, activeacute, ]{babel}
\usepackage{color}
\usepackage{transparent}
\graphicspath{{./figs/}}
\usepackage{makeidx}
\usepackage{afterpage}
\usepackage{array}
\usepackage{braket}
\usepackage{pst-node}
\newtheorem{teorema}{Teorema}[section]
\newtheorem{prop}[teorema]{Proposici\'on}
\newtheorem{cor}[teorema]{Corolario}
\newtheorem{lema}[teorema]{Lema}
\newtheorem{def.}{Definici\'on}[section]
\newtheorem{afir}{Afirmaci\'on}
\newtheorem{conjetura}{Conjetura}
\renewcommand{\figurename}{Figura}
\renewcommand{\indexname}{\'{I}ndice anal\'{\i}tico}
\newcommand{\zah}{\ensuremath{ \mathbb Z }}
\newcommand{\rac}{\ensuremath{ \mathbb Q }}
\newcommand{\nat}{\ensuremath{ \mathbb N }}
\newcommand{\prob}{\textbf{P}}
\newcommand{\esp}{\mathbb E}
\newcommand{\eje}{{\newline \noindent \sc \textbf{Ejemplo. }}}
\newcommand{\obs}{{\newline \noindent \sc \textbf{Observación. }}}
\newcommand{\dem}{{\noindent \sc Demostraci\'on. }}
\newcommand{\bg}{\ensuremath{\overline \Gamma}}
\newcommand{\ga}{\ensuremath{\gamma}}
\newcommand{\fb}{\ensuremath{\overline F}}
\newcommand{\la}{\ensuremath{\Lambda}}
\newcommand{\om}{\ensuremath{\Omega}}
\newcommand{\sig}{\ensuremath{\Sigma}}
\newcommand{\bt}{\ensuremath{\overline T}}
\newcommand{\li}{\ensuremath{\mathbb{L}}}
\newcommand{\ord}{\ensuremath{\mathbb{O}}}
\newcommand{\bs}{\ensuremath{\mathbb{S}^1}}
\newcommand{\co}{\ensuremath{\mathbb C }}
\newcommand{\con}{\ensuremath{\mathbb{C}^n}}
\newcommand{\cp}{\ensuremath{\mathbb{CP}}}
\newcommand{\rp}{\ensuremath{\mathbb{RP}}}
\newcommand{\re}{\ensuremath{\mathbb R }}
\newcommand{\hc}{\ensuremath{\widehat{\mathbb C} }}
\newcommand{\pslz}{\ensuremath{\mathrm{PSL}(2,\mathbb Z) }}
\newcommand{\pslr}{\ensuremath{\mathrm{PSL}(2,\mathbb R) }}
\newcommand{\pslc}{\ensuremath{\mathrm{PSL}(2,\mathbb C) }}
\newcommand{\hd}{\ensuremath{\mathbb H^2}}
\newcommand{\slz}{\ensuremath{\mathrm{SL}(2,\mathbb Z) }}
\newcommand{\slr}{\ensuremath{\mathrm{SL}(2,\mathbb R) }}
\newcommand{\slc}{\ensuremath{\mathrm{SL}(2,\mathbb C) }}
\newcommand{\mdlr}{\ensuremath{\mathrm{M}}}
\date{\today}
\title{Tarea 5}
\begin{document}

\maketitle

\textbf{Fecha de entrega: Domingo 30 de Marzo de 2025}

\begin{enumerate}
  \item Sean $y_{1}(x)$, $y_{2}(x)$ soluciones linealmente independientes de la ecuaci'on homogenea
        \[
            y''(x)+p(x)y'(x)+q(x)y=0,\quad x\in I.
        \]
        Muestre que
        \[
        p(x)=\frac{y_{2}(x)y''_{1}(x)-y_{1}(x)y''_{2}(x)}{W(y_{1},y_{2})(x)}\quad q(x)=\frac{y'_{1}(x)y''_{2}(x)-y'_{2}(x)y''_{1}(x)}{W(y_{1},y_{2})(x)}
        \]
\noindent\emph{Hint: Recuerde la ecuaci'on del Wronskiano y su derivaci'on.}
  \item Resuelva las siguientes ecuaciones (encuentre todas las soluciones)
        \begin{enumerate}
          \item $y''-5y'+6y=x^{3}e^{2x}$
          \item $y''+y'=\sec(x)$
        \end{enumerate}

  \item Sea \(U\) un operador unitario, demuestre que todo valor propio $\lambda$ de \(U\) tiene norma uno, i.e. $|\lambda|=1$.
  \item Sea $\rho(x)>0$ una funci'on medible positiva en $(X,\sig,\mu)$, demuestre que la integral de toda funci'on integrable $f$ en $(X,\sig,\nu_{\rho})$ donde
        \[
        \nu_{\rho}(E)=\int_{E}\rho(x)\,d\mu(x),\text{ es igual a }\int f(x)\,d\nu_{\rho}(x)=\int f(x)\rho(x)\,d\mu(x).
        \]
        Concluya que entonces que \((\mathcal{L}^{2}(X,\nu_{\rho}),\braket{\cdot|\cdot})\) es isomorfo a \((\mathcal{L}^{2}(X,\mu),\braket{\cdot|\cdot}_{\rho})\), donde
        \[
        \braket{f|g}_{\rho}=\braket{f|g\rho}=\int\overline{f(x)}g(x)\rho(x)\,d\mu(x).
        \]
\noindent\emph{Hint: Pruebelo para funciones simples y aproxime.}
\end{enumerate}
\end{document}

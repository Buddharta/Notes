% Created 2025-02-06 Thu 20:56
% Intended LaTeX compiler: pdflatex
\documentclass[letterpaper]{article}
\usepackage[utf8]{inputenc}
\usepackage[T1]{fontenc}
\usepackage{graphicx}
\usepackage{longtable}
\usepackage{wrapfig}
\usepackage{rotating}
\usepackage[normalem]{ulem}
\usepackage{amsmath}
\usepackage{amssymb}
\usepackage{capt-of}
\usepackage{hyperref}
\usepackage{lmodern} % Ensures we have the right font
\usepackage[utf8]{inputenc}
\usepackage{graphicx}
\usepackage{amsmath, amsthm, amssymb, amsfonts, amssymb, amscd}
\usepackage[table, xcdraw]{xcolor}
%\usepackage{mdsymbol}
\usepackage{tikz-cd}
\usepackage{float}
\usepackage[spanish, activeacute, ]{babel}
\usepackage{color}
\usepackage{transparent}
\graphicspath{{./figs/}}
\usepackage{makeidx}
\usepackage{afterpage}
\usepackage{array}
\usepackage{braket}
\usepackage{pst-node}
\newtheorem{teorema}{Teorema}[section]
\newtheorem{prop}[teorema]{Proposici\'on}
\newtheorem{cor}[teorema]{Corolario}
\newtheorem{lema}[teorema]{Lema}
\newtheorem{def.}{Definici\'on}[section]
\newtheorem{afir}{Afirmaci\'on}
\newtheorem{conjetura}{Conjetura}
\renewcommand{\figurename}{Figura}
\renewcommand{\indexname}{\'{I}ndice anal\'{\i}tico}
\newcommand{\zah}{\ensuremath{ \mathbb Z }}
\newcommand{\rac}{\ensuremath{ \mathbb Q }}
\newcommand{\nat}{\ensuremath{ \mathbb N }}
\newcommand{\prob}{\textbf{P}}
\newcommand{\esp}{\mathbb E}
\newcommand{\eje}{{\newline \noindent \sc \textbf{Ejemplo. }}}
\newcommand{\obs}{{\newline \noindent \sc \textbf{Observación. }}}
\newcommand{\dem}{{\noindent \sc Demostraci\'on. }}
\newcommand{\bg}{\ensuremath{\overline \Gamma}}
\newcommand{\ga}{\ensuremath{\gamma}}
\newcommand{\fb}{\ensuremath{\overline F}}
\newcommand{\la}{\ensuremath{\Lambda}}
\newcommand{\om}{\ensuremath{\Omega}}
\newcommand{\sig}{\ensuremath{\Sigma}}
\newcommand{\bt}{\ensuremath{\overline T}}
\newcommand{\li}{\ensuremath{\mathbb{L}}}
\newcommand{\ord}{\ensuremath{\mathbb{O}}}
\newcommand{\bs}{\ensuremath{\mathbb{S}^1}}
\newcommand{\co}{\ensuremath{\mathbb C }}
\newcommand{\con}{\ensuremath{\mathbb{C}^n}}
\newcommand{\cp}{\ensuremath{\mathbb{CP}}}
\newcommand{\rp}{\ensuremath{\mathbb{RP}}}
\newcommand{\re}{\ensuremath{\mathbb R }}
\newcommand{\hc}{\ensuremath{\widehat{\mathbb C} }}
\newcommand{\pslz}{\ensuremath{\mathrm{PSL}(2,\mathbb Z) }}
\newcommand{\pslr}{\ensuremath{\mathrm{PSL}(2,\mathbb R) }}
\newcommand{\pslc}{\ensuremath{\mathrm{PSL}(2,\mathbb C) }}
\newcommand{\hd}{\ensuremath{\mathbb H^2}}
\newcommand{\slz}{\ensuremath{\mathrm{SL}(2,\mathbb Z) }}
\newcommand{\slr}{\ensuremath{\mathrm{SL}(2,\mathbb R) }}
\newcommand{\slc}{\ensuremath{\mathrm{SL}(2,\mathbb C) }}
\newcommand{\mdlr}{\ensuremath{\mathrm{M}}}
\date{\today}
\title{Tarea 6}
\begin{document}

\maketitle

\textbf{Fecha de entrega: Lunes 7 de Abril de 2025}

\begin{enumerate}
  \item El prop'osito de este problema es encontrar los vectores y valores propios de la ecuacion de \emph{Cauchy-Euler} con valores de fontera
        \begin{align*}
          x^{2}y''+xy'+\lambda y=0,\quad x\in(1,b)\\
          y(1)=y(b)=0
        \end{align*}
        \begin{itemize}
          \item Encuentre $r(x)$ (que depende de $\lambda$) de tal forma que la ecuaci'on anterior es equivalente a $y''+r(x)y=0$, utilice el teorema de comparaci'on de Sturm para comparar con $y''+y=0$ y determinar para qu'e $\lambda$ se obtienen soluciones \emph{oscilantes}.
          \item Resuelva la ecuaci'on de arriba recordando que es de Cauchy-Euler y por lo tanto $y=x^{m}$, encuentre $m$ asumiendo $\lambda>0$.
          \item Use las condiciones de frontera para encontrar el espectro del operador $L$ y demostrar que es \emph{discreto}, donde
                \[
                L=x^{2}\dfrac{d^{2}}{dx^{2}}+x\dfrac{d}{dx}.
                \]
          \item Muestre que $L$ \textbf{NO} es formalmente autoadjunto y calcule la funci'on peso $\rho(x)$ que lo hace autoadjunto.
          \item Calcule el \emph{producto interno modificado} con el para mostrar que en efecto son ortogonales.
        \end{itemize}
  \item Sea $y''-2xy+(K-1)y=0$ la \emph{ecuaci'on de Hermite}, encuentre $\rho$ que la hace \textbf{formalmente autoadjunta}.

  \item Sea $L$ el operador de Sturm-Liouville
        \[
       L=\frac{d}{dx} \left( p(x) \dfrac{d}{dx} \right) + q(x)I.
        \]
        \noindent Demuestre la \emph{identidad de Lagrange}
        \[
        (\phi L\psi-\psi L\phi)(x)=\dfrac{d [p(x)(\phi\psi'-\psi\phi')]}{dx}.
        \]
        \noindent Integre para demostrar la \emph{F'ormula de Green.}
        \[
        \int_{a}^{b}(\phi L\psi-\psi L\phi)(x)\,dx=[p(x)(\phi\psi'-\psi\phi')]|_{a}^{b}.
        \]
%%  \item Utilice el m'etodo de variacion de par'ametros para encontrar la funcion de Green de las siguientes ecuaciones y resuelva (expresado en t'erminos de la integral $f$ y $G$).
%%        \begin{enumerate}
%%          \item $y''-5y'+6y=f(x)$
%%          \item $y''+y'=f(x)$
%%        \end{enumerate}
\end{enumerate}
\end{document}

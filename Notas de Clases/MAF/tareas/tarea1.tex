% Created 2025-02-06 Thu 20:56
% Intended LaTeX compiler: pdflatex
\documentclass[letterpaper]{article}
\usepackage[utf8]{inputenc}
\usepackage[T1]{fontenc}
\usepackage{graphicx}
\usepackage{longtable}
\usepackage{wrapfig}
\usepackage{rotating}
\usepackage[normalem]{ulem}
\usepackage{amsmath}
\usepackage{amssymb}
\usepackage{capt-of}
\usepackage{hyperref}
\usepackage{lmodern} % Ensures we have the right font
\usepackage[utf8]{inputenc}
\usepackage{graphicx}
\usepackage{amsmath, amsthm, amssymb, amsfonts, amssymb, amscd}
\usepackage[table, xcdraw]{xcolor}
%\usepackage{mdsymbol}
\usepackage{tikz-cd}
\usepackage{float}
\usepackage[spanish, activeacute, ]{babel}
\usepackage{color}
\usepackage{transparent}
\graphicspath{{./figs/}}
\usepackage{makeidx}
\usepackage{afterpage}
\usepackage{array}
\usepackage{braket}
\usepackage{pst-node}
\newtheorem{teorema}{Teorema}[section]
\newtheorem{prop}[teorema]{Proposici\'on}
\newtheorem{cor}[teorema]{Corolario}
\newtheorem{lema}[teorema]{Lema}
\newtheorem{def.}{Definici\'on}[section]
\newtheorem{afir}{Afirmaci\'on}
\newtheorem{conjetura}{Conjetura}
\renewcommand{\figurename}{Figura}
\renewcommand{\indexname}{\'{I}ndice anal\'{\i}tico}
\newcommand{\zah}{\ensuremath{ \mathbb Z }}
\newcommand{\rac}{\ensuremath{ \mathbb Q }}
\newcommand{\nat}{\ensuremath{ \mathbb N }}
\newcommand{\prob}{\textbf{P}}
\newcommand{\esp}{\mathbb E}
\newcommand{\eje}{{\newline \noindent \sc \textbf{Ejemplo. }}}
\newcommand{\obs}{{\newline \noindent \sc \textbf{Observación. }}}
\newcommand{\dem}{{\noindent \sc Demostraci\'on. }}
\newcommand{\bg}{\ensuremath{\overline \Gamma}}
\newcommand{\ga}{\ensuremath{\gamma}}
\newcommand{\fb}{\ensuremath{\overline F}}
\newcommand{\la}{\ensuremath{\Lambda}}
\newcommand{\om}{\ensuremath{\Omega}}
\newcommand{\sig}{\ensuremath{\Sigma}}
\newcommand{\bt}{\ensuremath{\overline T}}
\newcommand{\li}{\ensuremath{\mathbb{L}}}
\newcommand{\ord}{\ensuremath{\mathbb{O}}}
\newcommand{\bs}{\ensuremath{\mathbb{S}^1}}
\newcommand{\co}{\ensuremath{\mathbb C }}
\newcommand{\con}{\ensuremath{\mathbb{C}^n}}
\newcommand{\cp}{\ensuremath{\mathbb{CP}}}
\newcommand{\rp}{\ensuremath{\mathbb{RP}}}
\newcommand{\re}{\ensuremath{\mathbb R }}
\newcommand{\hc}{\ensuremath{\widehat{\mathbb C} }}
\newcommand{\pslz}{\ensuremath{\mathrm{PSL}(2,\mathbb Z) }}
\newcommand{\pslr}{\ensuremath{\mathrm{PSL}(2,\mathbb R) }}
\newcommand{\pslc}{\ensuremath{\mathrm{PSL}(2,\mathbb C) }}
\newcommand{\hd}{\ensuremath{\mathbb H^2}}
\newcommand{\slz}{\ensuremath{\mathrm{SL}(2,\mathbb Z) }}
\newcommand{\slr}{\ensuremath{\mathrm{SL}(2,\mathbb R) }}
\newcommand{\slc}{\ensuremath{\mathrm{SL}(2,\mathbb C) }}
\newcommand{\mdlr}{\ensuremath{\mathrm{M}}}
\date{\today}
\title{Tarea 1 (Del Amor y la Amistad)}
\begin{document}

\maketitle

\textbf{Fecha de entrega: Viernes 14 de Febrero de 2025}

\begin{enumerate}
\item Demuestre la Identidad de polarización y la igualdad de Parserval (teorema de Pitágoras):
Sea \((H,\braket{\cdot|\cdot})\) un espacio de Hilbert y \(\beta=\{\ket{\phi_n}\}_{n\in\nat}\) una base de Schauder ortornormal, entonces para cada \(\ket{\psi}\in H\)
\[
        \|\psi\|^2=\sum_{n=0}^{\infty}|\braket{\phi_n|\psi}|^2.
   \]
\item Obtenga los coeficientes de Fourier complejos (con la base \(\{e^{int}\}_{n\in\zah}\)) para la función \(f(\theta)=\theta\) en el intervalo \([-\pi,\pi)\), use el hecho de que \(f(-\theta)=-f(\theta)\) para deducir la serie de Fourier real de \(f\).
\item Demuestre que \(f_n(x)=n^2xe^{-nx}\) es una susesión de funciones continuas en \(\re\) que converge uniformemente a \(0\) en los intervalos \([a,\infty)\) con \(a>0\).
\item Sea \(A\in\textrm{M}_{n\times n}(\co)\) una matriz con entradas complejas, demuestre que la matriz exponencial de \(A\)
\[
        \exp(A)=\sum_{n=0}^{\infty}\frac{A^n}{n!}\quad\text{donde } A^0=I_n.
   \]
Existe y además \(\|\exp(A)\|\leq e^{\|A\|}\), donde \(\|A\|\) es la norma del operador definido por la matriz \(A\). Calcule \(\exp(A)\) para una matriz diagonal
\[
        \begin{pmatrix}\lambda_1 & 0 & \dots & 0\\
                       0 & \lambda_2 & \dots & 0\\
                                \vdots\\
                       0 & 0 & \dots & \lambda_n
        \end{pmatrix}.
   \]
\end{enumerate}
\end{document}

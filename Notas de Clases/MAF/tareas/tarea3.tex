% Created 2025-02-06 Thu 20:56
% Intended LaTeX compiler: pdflatex
\documentclass[letterpaper]{article}
\usepackage[utf8]{inputenc}
\usepackage[T1]{fontenc}
\usepackage{graphicx}
\usepackage{longtable}
\usepackage{wrapfig}
\usepackage{rotating}
\usepackage[normalem]{ulem}
\usepackage{amsmath}
\usepackage{amssymb}
\usepackage{capt-of}
\usepackage{hyperref}
\usepackage{lmodern} % Ensures we have the right font
\usepackage[utf8]{inputenc}
\usepackage{graphicx}
\usepackage{amsmath, amsthm, amssymb, amsfonts, amssymb, amscd}
\usepackage[table, xcdraw]{xcolor}
%\usepackage{mdsymbol}
\usepackage{tikz-cd}
\usepackage{float}
\usepackage[spanish, activeacute, ]{babel}
\usepackage{color}
\usepackage{transparent}
\graphicspath{{./figs/}}
\usepackage{makeidx}
\usepackage{afterpage}
\usepackage{array}
\usepackage{braket}
\usepackage{pst-node}
\newtheorem{teorema}{Teorema}[section]
\newtheorem{prop}[teorema]{Proposici\'on}
\newtheorem{cor}[teorema]{Corolario}
\newtheorem{lema}[teorema]{Lema}
\newtheorem{def.}{Definici\'on}[section]
\newtheorem{afir}{Afirmaci\'on}
\newtheorem{conjetura}{Conjetura}
\renewcommand{\figurename}{Figura}
\renewcommand{\indexname}{\'{I}ndice anal\'{\i}tico}
\newcommand{\zah}{\ensuremath{ \mathbb Z }}
\newcommand{\rac}{\ensuremath{ \mathbb Q }}
\newcommand{\nat}{\ensuremath{ \mathbb N }}
\newcommand{\prob}{\textbf{P}}
\newcommand{\esp}{\mathbb E}
\newcommand{\eje}{{\newline \noindent \sc \textbf{Ejemplo. }}}
\newcommand{\obs}{{\newline \noindent \sc \textbf{Observación. }}}
\newcommand{\dem}{{\noindent \sc Demostraci\'on. }}
\newcommand{\bg}{\ensuremath{\overline \Gamma}}
\newcommand{\ga}{\ensuremath{\gamma}}
\newcommand{\fb}{\ensuremath{\overline F}}
\newcommand{\la}{\ensuremath{\Lambda}}
\newcommand{\om}{\ensuremath{\Omega}}
\newcommand{\sig}{\ensuremath{\Sigma}}
\newcommand{\bt}{\ensuremath{\overline T}}
\newcommand{\li}{\ensuremath{\mathbb{L}}}
\newcommand{\ord}{\ensuremath{\mathbb{O}}}
\newcommand{\bs}{\ensuremath{\mathbb{S}^1}}
\newcommand{\co}{\ensuremath{\mathbb C }}
\newcommand{\con}{\ensuremath{\mathbb{C}^n}}
\newcommand{\cp}{\ensuremath{\mathbb{CP}}}
\newcommand{\rp}{\ensuremath{\mathbb{RP}}}
\newcommand{\re}{\ensuremath{\mathbb R }}
\newcommand{\hc}{\ensuremath{\widehat{\mathbb C} }}
\newcommand{\pslz}{\ensuremath{\mathrm{PSL}(2,\mathbb Z) }}
\newcommand{\pslr}{\ensuremath{\mathrm{PSL}(2,\mathbb R) }}
\newcommand{\pslc}{\ensuremath{\mathrm{PSL}(2,\mathbb C) }}
\newcommand{\hd}{\ensuremath{\mathbb H^2}}
\newcommand{\slz}{\ensuremath{\mathrm{SL}(2,\mathbb Z) }}
\newcommand{\slr}{\ensuremath{\mathrm{SL}(2,\mathbb R) }}
\newcommand{\slc}{\ensuremath{\mathrm{SL}(2,\mathbb C) }}
\newcommand{\mdlr}{\ensuremath{\mathrm{M}}}
\date{\today}
\title{Tarea 3}
\begin{document}

\maketitle

\textbf{Fecha de entrega: Sábado 1 de Marzo de 2025}

\begin{enumerate}
\item Sea $\{f_n\}$ una susesión de funciones medibles tal que $\{f_n(x)\}$ es acotada ara toda $x\in X$, demuestre que las siguientes funciones son medibles
\begin{enumerate}
\item $m_n(x)=\min\{f_i(x)\,:\,i\in\{1,2,\dots,n\}\}$
\item $M_n(x)=\max\{f_i(x)\,:\,i\in\{1,2,\dots,n\}\}$
\item $m(x)=\inf\{f_n(x)\,:\,n\in\nat\}$
\item $M(x)=\min\{f_n(x)\,:\,n\in\nat\}$
\item $f^{*}(x)=\limsup_{n\to\infty}f_n(x)$
\item $f_*(x)=\liminf_{n\to\infty}f_n(x)$
\end{enumerate}
\emph{Sugerencia: Encuentre los análogos conjuntistas de supremos e ínfimos.}

\item El propósito de este ejercicio es demostrar que las integrales de funciones positivas son medidas, en particular las integrales de una distribución de probabilidad son una \emph{medida de probabilidad}
  \begin{itemize}
  \item Sean $A$ y $B$ conjuntos cualesquiera, $\chi_A$ y $\chi_B$ sus respectivas fucniones características, demuestre que para la unión $A\cup B$ y la intersección $A\cap B$
    \begin{align*}
    \chi_{A\cup B}&=\chi_A+\chi_B-\chi_{A\cap B}\\
    \chi_{A\cap B}&=\chi_A\chi_B
    \end{align*}
  \item Sea $(X,\sig,\mu)$ un espacio de medida y $E\in\sig$ un conjunto nulo, demuestre que para toda $f:X\to\re$ se cumple que
    \[
    \int_E f d\mu=\int f\chi_E=0.
    \]
  \emph{Sugerencia: Demuestrelo para funciones simples y use el teorema de aproximación.}
  \item Sea la función $N(x;\sigma,\nu)$, la distribución normal donde $\sigma$ y $\nu$ son prarámetros constantes que representan la media y desviación estandard, demuestre que la función $\mu_{\sigma\nu}:\mathbb{B}(\re)\to\re$ definida como
  \[
  \mu_{\sigma\nu}(E)=\int_{E}N(x;\sigma,\nu)\,d\lambda=\int\frac{1}{\sqrt{2\pi\sigma^2}}e^{-\frac{(x-\nu)^2}{2\sigma^2}}\chi_E\,d\lambda.
  \]
  Es una medida, con $\mu_{\sigma\nu}(\re)=1$ y que todo conjunto nulo de lebesgue en $\mathbb{B}(\re)$ es nulo para $\mu_{\sigma\nu}$\\
  \end{itemize}
\item Sea $X=\nat$, $\sig=\mathcal{P}(\nat)$ y $\mu:\sig\to\re^{+}$ definida por
  \[
  \mu(E)=\sum_{j\in E}  \frac{1}{2^j}\text{ ó }\mu(\emptyset)=0.
  \]
\begin{itemize}
  \item[i)] Demuestre que $\mu$ es una medida con $\mu(X)=1$
  \item[ii)] Sea $f_n=2^n\chi_{\{n+1,n+1,\dots\}}$, demuestre que
      \[
      \int_{\nat}f_n\,d\mu=1 \quad\forall n\in\nat\text{ y que } f_n\to 0.
      \]
      Explique por qué no se cumple el teorma de convergencia dominada.
\end{itemize}
\end{enumerate}
\end{document}

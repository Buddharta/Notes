% Created 2025-01-06 Mon 18:37
% Intended LaTeX compiler: pdflatex
\documentclass[letterpaper]{book}
\usepackage[utf8]{inputenc}
\usepackage[T1]{fontenc}
\usepackage{graphicx}
\usepackage{longtable}
\usepackage{wrapfig}
\usepackage{rotating}
\usepackage[normalem]{ulem}
\usepackage{amsmath}
\usepackage{amssymb}
\usepackage{capt-of}
\usepackage{hyperref}
\usepackage{graphicx}
\usepackage{amsmath, amsthm, amssymb, amsfonts, amssymb, amscd}
\usepackage[table, xcdraw]{xcolor}
%\usepackage{mdsymbol}
\usepackage{tikz-cd}
\usepackage{float}
\usepackage[spanish, activeacute, ]{babel}
\usepackage{color}
\usepackage{transparent}
\graphicspath{{./figs/}}
\usepackage{afterpage}
\usepackage{array}
\usepackage{pst-node}
\usepackage{imakeidx}
\usepackage{braket}

\newcommand{\ind}{\perp\!\!\!\!\perp}
\newtheorem{teorema}{Teorema}[section]
\newtheorem{prop}[teorema]{Proposici\'on}
\newtheorem{cor}[teorema]{Corolario}
\newtheorem{lema}[teorema]{Lema}
\newtheorem{def.}{Definici\'on}[section]
\newtheorem{afir}{Afirmaci\'on}
\newtheorem{conjetura}{Conjetura}
\renewcommand{\figurename}{Figura}

\newcommand{\zah}{\ensuremath{ \mathbb Z }}
\newcommand{\rac}{\ensuremath{ \mathbb Q }}
\newcommand{\nat}{\ensuremath{ \mathbb N }}
\newcommand{\prob}{\textbf{P}}
\newcommand{\esp}{\mathbb E}
\newcommand{\exe}{{\noindent \sc \textbf{Ejercicio. }}}
\newcommand{\eje}{{\noindent \sc \textbf{Ejemplo. }}}
\newcommand{\obs}{{\noindent \sc \textbf{Observación. }}}
\newcommand{\sol}{{\noindent \sc \textbf{Solución. }}}
\newcommand{\dem}{{\noindent \sc \textbf{Demostraci\'on. }}}
\newcommand{\QED}{\ensuremath{\hspace*{0em plus 1fill}\blacksquare}}

\newcommand{\bg}{\ensuremath{\overline \Gamma}}
\newcommand{\ga}{\ensuremath{\gamma}}
\newcommand{\fb}{\ensuremath{\overline F}}
\newcommand{\la}{\ensuremath{\Lambda}}
\newcommand{\om}{\ensuremath{\Omega}}
\newcommand{\sig}{\ensuremath{\Sigma}}
\newcommand{\bt}{\ensuremath{\overline T}}
\newcommand{\li}{\ensuremath{\mathbb{L}}}
\newcommand{\ord}{\ensuremath{\mathbb{O}}}
\newcommand{\bs}{\ensuremath{\mathbb{S}^1}}
\newcommand{\co}{\ensuremath{\mathbb C }}
\newcommand{\con}{\ensuremath{\mathbb{C}^n}}
\newcommand{\cp}{\ensuremath{\mathbb{CP}}}
\newcommand{\rp}{\ensuremath{\mathbb{RP}}}
\newcommand{\re}{\ensuremath{\mathbb R }}
\newcommand{\hc}{\ensuremath{\widehat{\mathbb C} }}
\newcommand{\pslz}{\ensuremath{\mathrm{PSL}(2,\mathbb Z) }}
\newcommand{\pslr}{\ensuremath{\mathrm{PSL}(2,\mathbb R) }}
\newcommand{\pslc}{\ensuremath{\mathrm{PSL}(2,\mathbb C) }}
\newcommand{\hd}{\ensuremath{\mathbb H^2}}
\newcommand{\slz}{\ensuremath{\mathrm{SL}(2,\mathbb Z) }}
\newcommand{\slr}{\ensuremath{\mathrm{SL}(2,\mathbb R) }}
\newcommand{\slc}{\ensuremath{\mathrm{SL}(2,\mathbb C) }}
\newcommand{\mdlr}{\ensuremath{\mathrm{M}}}
\author{Carlos Eduardo Martínez Aguilar}
\date{\today}
\title{Notas de MAF}
\hypersetup{
 pdfauthor={Carlos Eduardo Martínez Aguilar},
 pdftitle={Notas de Matemáticas Avanzadas para la Física},
 pdfkeywords={},
 pdfsubject={},
 pdflang={Esp}}

\makeindex
\begin{document}

\maketitle
\tableofcontents

\chapter{Introducción}

\noindent El nombre ``Matemáticas Avanzadas para la Física'' es bastante peculiar para un curso de un semestre de licenciatura, es un nombre muy poco descriptivo para una materia, uno al leerlo o escucharlo inmediatamente se puede preguntar ``¿A qué se refiere con ``Matemáticas Avanzadas''?'', ``¿Qué se abarca en un curso así?'' o ``¿No se supone que los primeros 5 semestres de la carrera de física se dedican al estudio de matemáticas avanzadas para la física?''. Si se está familiariazado con temas de física (se esperara que la mayor parte de las personas que toman este curso lo estén), entonces al escuchar ``Matemáticas Avanzadas'', uno se puede imaginar casi cualquier tema de matemáticas desarrollado durante el último siglo ya que es muy dificil encontrar ramas de la matemática que no se apliquen a los problemas físicos de los últimos años, desde topología algebraica hasta temas de estadística paramétrica.
Sin embargo, este es un curso preciso de licenciatura, por lo cual se debe acotar en los temas que se deber abordar, al revisar el temario de este curso, es claro que la mentalidad que se utilizó en su elaboración fue la de intentar abarcar los temas con aplicabilidad a la mayor parte de la física actual, al mismo tiempo se intentó mantener continuidad con el antiguo plan de estudios de física, en el cual la presente materia no existía y en su lugar se daba un curso de ``Funciones Especiales y Transformadas Integrales'' o ``FETI'' en corto. Viendo esto, queda más claro que ``Matemáticas Avanzadas para la Física'' se refiere más precisamente a ``métodos del analisis matemático, funcional y complejo para el estudio de soluciones de ecuaciones diferenciales ordinarias y parciales originarias de la física''.
Muchas presonas que han cusado esta materia nos han mencionado que el problema con una materia así es que los profesores tienden a enfocarse solamente en los métodos para terminar con el amplio temario, en lugar de desarrollar la intuición física y matemática para la resolución de éstos problemas. Nosotros como matemáticos impartiremos un curso de matemáticas, sin embargo esto no significa que nos enfocaremos en demostrar teoremas o solamente desarrollar la teoría, significa que expondremos la intución matemática detrás de los problemas sin perder el rigor matemático que permite generalizar estos métodos para otros problemas y al mismo tiempo sin ignorar la aplicabilidad de estos métodos. Intentaremos mostrar que a pesar que parece que son muchos métodos y mucha teoría, en realidad estos métodos son una extensión del álgebra lineal por medio del cálculo para la soluón de ecuaciones diferenciales. Aunado a estos pretendemos que esta materia sea una preparación matemática para la mecánica cuántica por lo que haremos énfasis en problemas de mecánica cuántica y en la teoría espectral de operadores, además que utilizaremos la notación de Dirac para espacios de Hilbert.\\
%%\vspace{1cm}
%%
%%\noindent \textbf{TEMARIO} (El orden puede variar un poco).
%%
%%\begin{enumerate}
%%\item Introducción y conceptos preliminares:
%%\begin{itemize}
%%\item Espacios vectoriales y notación de Dirac.
%%\item Espacios Hilbert como espacios vectoriales de dimensión infinita.
%%\item Funcionales lineales, operadores y sus representaciones.
%%\item Formalismo de la teoría de la medida y el teorema de Riemann-Lebesgue.
%%\item Repaso de convergencia de series y variable compleja.
%%\end{itemize}
%%
%%\item Análisis de Fourier:
%%\begin{itemize}
%%\item Series de Fourier.
%%\item Condiciones de Dirichlet.
%%\item Convergencia de las series de Fourier.
%%\item Teorema de Fourier.
%%\item Series ortogonales en 2 variables.
%%\item Aplicaciones.
%%\end{itemize}
%%
%%\item Teoría de Strurm-Liouville
%%\begin{itemize}
%%\item Operadores auto adjuntos.
%%\item Operadores diferenciales auto adjuntos.
%%\item Funciones propias y funciones de peso.
%%\item Funciones de Green para ecuaciones no homogeneas.
%%\end{itemize}
%%
%%\item Ecuación de Laplace:
%%\begin{itemize}
%%\item Ecuación de Laplace en 2 dimensiones.
%%\item Separación de Variables
%%\item Armónicos esféricos.
%%\item Ecuación de Laplace en coordenadas cilíndricas.
%%\item Ecuación y Funciones de Bessel.
%%\end{itemize}
%%
%%\item Funciones Especiales y Polinomios Ortogonales:
%%\begin{itemize}
%%\item Definición y Ejemplos.
%%\item Familias ortogonales de polinomios (Laguerre, Hermite, Legendre y otros).
%%\item Función generadora y Aplicaciones.
%%\item Definición de la función Beta y su relación con la función Gama.
%%\item Definición de la Delta de Dirac.
%%\item Función Zeta de Riemann.*
%%\end{itemize}
%%
%%\item Ecuaciones de Onda:
%%\begin{itemize}
%%\item Análisis Vectorial y Teorema de descomposición de Helmholtz.
%%\item La Cuerda y la membrana circular vibrantes
%%\item Condiciones de la frontera.
%%\item Ecuación de Helmholtz en una y dos dimensiones.
%%\item Solución en problemas con valores en la frontera (con el formalismo de Fourier).
%%\item Ecuación de Shrodinger como ecuación de onda y ejemplos.
%%\end{itemize}
%%
%%\item Ecuaciones de Calor:
%%\begin{itemize}
%%\item Solución fundamental.
%%\item El flujo de calor.
%%\item Movimiento Browniano*.
%%\end{itemize}
%%
%%\item Distribuciones (funciones generalizadas):
%%\begin{itemize}
%%\item Distribuciones.
%%\item Derivadas débiles.
%%\item La delta de dirac y la función de Heaviside.
%%\item Funciones de Schwartz y distribuciones temperadas.
%%\end{itemize}
%%
%%\item Transformadas Integrales:
%%\begin{itemize}
%%\item Transformada de Fourier.
%%\item Propiedades y aplicaciones de la transformada de Fourier.
%%\item Transformada de Laplace.
%%\item Propiedades y aplicaciones de la transformada de Laplace.
%%\item Relación con las transformadas de Fourier.
%%\item Transformada de Fourier discreta y algoritmos*.
%%\end{itemize}
%%
%%\item Funciones de Green para ecuaciones diferenciales parciales:
%%\begin{itemize}
%%\item Transformada de Fourier para la ecuación de calor.
%%\item Transformada de Fourier para la ecuación de onda.
%%\item La ecuación de Poisson.
%%\end{itemize}
%%\end{enumerate}

\chapter{Espacios de Hilbert}

\section{Espacios vectoriales y notación de Dirac}

\noindent Comencemos recordando algunos hechos sobre vectores y espacios vectoriales, pero ahora desde la perspectiva y notación de Dirac. La notación de Dirac se asocia principalmente a la mecánica cuántica, sin embargo, es marco teoríco general para los espacios de Hilbert, los cuales con son los que trabajaremos pricipalmente en este curso. La notación de Dirac funciona explicitamente para trabajar con espacios de Hilbert de funciones (todo espacio de Hilbert es el fondo un espacio de funciones), donde el vector asociado a una función $\phi$ se denota como $\ket{\phi}$, al cual llamado ``ket''. Por lo tanto, un espacio vectorial sobre un campo \(\mathbb{F}\) (en este curso siempre tomaremos \(\mathbb{F} = \mathbb{R}\) o \(\mathbb{F} = \mathbb{C}\)) se concibe como un conjunto \(V\) de kets, el cual es dotado las operaciones de adición \(+\) y producto escalar \(\cdot\) que cumplen las siguientes propiedades

\begin{itemize}
    \item \textbf{Conmutatividad}: \(\ket{\phi} + \ket{\psi} = \ket{\psi} + \ket{\phi}\quad\forall\,\{\ket{\psi},\ket{\phi}\}\subset V.\)
    \item \textbf{Asociatividad}: \((\ket{\varphi}+\ket{\phi}) + \ket{\psi} = \ket{\varphi}+(\ket{\psi} + \ket{\phi})\quad\forall\,\{\ket{\psi},\ket{\phi},\ket{\varphi}\}\subset V.\)
    \item \textbf{Identidad}: \(\exists! \, 0 \in V\) tal que \(0 + \ket{\phi} = \ket{\phi}\quad\forall\,\ket{\phi}\in V.\)
\end{itemize}

Del último punto, haremos notar que $0$ y $\ket{0}$ son distintos vectores, $0$ será simepre el vector nulo y $\ket{0}$ será un vector no nulo que representa su número en una base numerada. Ahora, la operación de multiplicación por un escalar \(\lambda \in \co\) cumple

\begin{itemize}
    \item \textbf{Distributidad en \(V\)}: \(\lambda(\ket{\phi_1} + \ket{\phi_2}) = \lambda\ket{\phi_1} + \lambda\ket{\phi_2}\quad\forall\,\{\ket{\phi_1},\ket{\phi_2}\}\subset V,\,\lambda\in\co.\)
    \item \textbf{Distributidad en \(\co\)}: \((\lambda + \mu)\ket{\phi} = \lambda\ket{\phi} + \mu\ket{\phi}\quad\forall\,\ket{\phi}\in V,\,\{\lambda,\mu\}\subset\co.\)
\end{itemize}

A menudo trabajaremos con espacios vectoriales dotados de un producto interno, lo cual otorga una forma generalizada de trabajar con funcionales lieales, veremos más de esto en la parte de operadores y funcionales.
\begin{def.}
Un \emph{producto interno o producto Hermitiano} es un mapeo \( \braket{\cdot|\cdot} : V^{*} \times V \to \co\) que cumple \footnote{¡Cuidado! Es muy común que algunos autores definan el producto interno como lineal en la primera entrada, en lugar de la segunda como lo he hecho aquí. He elegido esta forma para maximizar la concordancia con tus clases de Mecánica Cuántica}

\begin{itemize}
    \item \textbf{Simetría conjugada}: \(\braket{\phi | \psi} = \overline{\braket{\psi | \phi}}\)
    \item \textbf{Linealidad}: \(\braket{\phi | \lambda \psi} = \lambda\braket{\phi|\psi}\quad\forall\,\lambda\in\co.\)
    \item \textbf{Aditividad}: \(\braket{\varphi | \phi + \psi} = \braket{\varphi | \phi} + \braket{\varphi|\psi}\)
    \item \textbf{Definida positiva}: \(\braket{\phi | \phi} \geq 0\,\forall \bra{\psi} \in V\), con igualdad si y solo si \(\phi = 0\).
\end{itemize}
\end{def.}
Vemos ahora algunos ejemplos de los espacios con producto interno más comjnes
\eje Los espacios \(\re^n\) y \(\co^n\) con el producto punto usual son espacios vectoriales con un producto interno, donde el producto punto usual lo definimos como
\[
\text{En \(\re^n\) se define como: }(x_1,\dots,x_n)\cdot(y_1,\dots,y_n)=x_1y_1+\dots+x_ny_n
\]
\[
\text{En \(\co^n\) se define como: }(\alpha_1,\dots,\alpha_n)\cdot(\beta_1,\dots,\beta_n)=\overline{\alpha_1}\beta_1+\dots+\overline{\alpha_n}\beta_n
\]
\eje Normalmente trabajaremos con espacios de funciones, uno de los más simples es \(\mathcal{C}[a,b]=\{f:[a,b]\rightarrow\co\,:\,\text{f es continua}\}\), con el producto interno
\[
\braket{f|g}=\int_a^b\overline{f}(t)g(t)dt.
\]
Nota que si nuestros vectores son reales, entonces la propiedad \(\braket{u|v} = \overline{\braket{v|u}}\) implica que \(\braket{\cdot|\cdot}\) es simétrica en sus argumentos. En este caso, el mapeo \(\braket{\cdot|\cdot}: V^{*} \times V \to \re\) es bilineal. Si \(\mathbb{F} = \co\), el mapa a veces se llama sesquilineal. Es aquí donde es necesario mencionar los ``bras'' como parte de la notación de Dirac, los bras se denotn por el símbolo contrario a los ``kets'', $\bra{\varphi}$ y estos representan \emph{vectores duales o covectores}, o equivalentemente, funcionales lineales \hbox{$\bra{\varphi}:V\rightarrow\mathbb{F}$} con aplicación dada por el ``braket'' $\bra{\varphi}\ket{\phi}=\braket{\varphi|\phi}$, esto significa que los funcionales lineales provienen del producto interno y por lo tanto $V\cong V^{*}$ canonicamente por medio del ``braket''. Esta es una propiedad exclusiva a los espacios vectoriales con producto interno y más aún a los \emph{espacios de Hilbert}, esto lo veremos a mayor profundidad en la sección de operdores lineales. Se le conoce a los espacios vectoriales con producto interno como \emph{espacios Hermitianos} aunque esta distinción no es ampliamente usada.

Un producto interno en \(V\) proporciona automáticamente a un espacio vectorial una norma, la cual permite realizar operaciones típicas del análisis matemático y de la geometría en \(V\).
\begin{def.}
Una norma en \(V\) es una función \(\Vert\cdot\Vert:V\rightarrow\re^{+}\) que cumple lo siguiente
\begin{itemize}
    \item \textbf{Definida positiva}: \(\Vert\phi\Vert \geq 0\,\forall \bra{\psi} \in V\), con igualdad si y solo si \(\phi = 0\).
    \item \textbf{Homogeneidad}: \(\Vert \lambda \psi\Vert = |\lambda|\Vert\phi\Vert\quad\forall\,\lambda\in\co.\)
    \item \textbf{Desigualdad del triángulo}: \(\Vert \phi + \psi\Vert \leq \Vert\phi\Vert + \Vert\psi\Vert\quad\forall\,\{\ket{\psi},\ket{\phi}\}\subset V.\)
\end{itemize}
\end{def.}
\eje Dada una función continua \(f: \om\subset\re^n \to \mathbb{R}\) (o \(\mathbb{C}\)), donde \(\om\) es un subconjunto compacto, la \emph{norma del supremo} se define como
    \[
    \|f\|_\infty = \sup_{x \in X} |f(x)|.
    \]
A un espacio vectoral con una norma \(\Vert\cdot\Vert\) se le llama \emph{espacio normado}. Ahora, para definir una norma para un producto normado, podemos definir la longitud de un vector \(\ket{\phi}\) como la norma \(\Vert\phi\Vert=\sqrt{\braket{\phi|\phi}}\). Aquí es importante recalcar que no todo espacio vectorial normado tiene un producto interno y más aún, se pueden clasificar las normas que vienen de un producto interno con la \emph{ley del paralelogramo}

\[
\Vert x+y\Vert^2+\Vert x-y\Vert^2=2\big(\Vert x\Vert^2+\Vert y\Vert^2\big)
\]
\noindent relacionada con la ley del paralelogramo, esta la identidad de polarización, la cual permite recuperar el producto interno a travez de su norma
\[
\braket{x|y}=\frac{1}{4}\big(\Vert x+y\Vert^2-\Vert x-y\Vert^2-i\Vert x+iy\Vert^2+i\Vert x-iy\Vert^2\big)
\]
\exe Demuestre la ley de polarización para un espacio con producto interno.

Aunado a la identidad de polarización, está la desigualdad de Cauchy-Schwartz que relaciona a la norma de los vectores con su producto interno
\begin{equation}
  |\braket{x|y}|\leq\| x\|\,\| y\|
\end{equation}
\exe Demuestre la desigualdad de Cauchy-Schwarz para espacios vectoriales de dimensión finita.
Para espacios vectoriales reales con producto interno, es posible definir el ángulo entre dos vectores, en cuyo caso se define como

\[
\theta = \arccos\left(\frac{\braket{u | v}}{\sqrt{\braket{u | u}} \sqrt{\braket{v | v}}}\right).
\]

\noindent donde la desigualdad de Cauchy-Schwartz \(\braket{u | v}^2 \leq \Vert u\Vert^2\Vert v \Vert^2\) asegura que esto tenga sentido.

\begin{def.}
Un conjunto de vectores \(\beta=\{\ket{\phi_{\alpha}}\,:\,\alpha\in \Lambda \}\) forma una base de Hammel de \(V\) si cualquier elemento \(u \in V\) puede escribirse de manera única como cobinación lineal de sus elementos, es decir
\[
u = \sum_{i\in F\subset\Lambda} \lambda_i \ket{\phi_i},
\]

\noindent para algunos escalares \(\lambda_i\in\mathbb{F}\), donde \(F\subset\Lambda\) es un conjunto finito.

\noindent Dado cualquier conjunto \(\beta\subset V\) el conjunto de combinaciones lineales de elementos de \(\beta\) se denotará \(\langle\beta\rangle\).

\noindent Una base \(\{\ket{\phi_{\alpha}}\}\) es ortogonal con respecto al producto interno si \(\braket{\phi_i | \phi_j}=\delta_{ij}\), la base es ortonormal si \(\Vert \phi_i\Vert = 1\).

\noindent La dimensión del espacio vectorial es el número o cardinalidad de cualquier base.
\end{def.}

\begin{teorema}
  Todo espacio vectorial tiene una base.
\end{teorema}

Cuando tenemos una base ortonormal, podemos usar el producto interno para descomponer explícitamente un vector general en esta base por medio de proyecciones. Por ejemplo, si

\[
u = \sum_{i=1}^n \lambda_i \ket{\phi_i},
\]
entonces
\[
\braket{\phi_j| u} = \braket{\phi_j| \sum_{i=1}^n \lambda_i \ket{\phi_i}} = \sum_{i=1}^n \lambda_i\braket{\phi_j |\phi_i} = \lambda_j,
\]

donde usamos las propiedades de aditividad y linealidad de \(\braket{\cdot|\cdot}\), así como la ortonormalidad de la base, entonces podemos escribir a \(u\) como
\[
u = \sum_{i=1}^n \ket{\phi_i}\braket{\phi_i|u}.
\]

En general \(\ket{\phi}\bra{\psi}\) es el \emph{operador proyección o proyector en el espacio generado por \(\ket{\psi}\)} y en general, una base ortonormal se puede caracterizar por la siguiente igualdad entre el operador identidad en elpacio total \(V\) y la suma de las proyecciones en los espacios de las componentes de la base
\[
I_{V}=\sum_{\alpha\in\Lambda}\ket{\phi_{\alpha}}\bra{\phi_{\alpha}}.
\]
\exe Demuestre que cualesquiera dos bases tienen el mismo número de vectores.

\subsection{Proceso de Gram-Schmidt}

\noindent El proceso de Gram-Schmidt es un método para transformar un conjunto de vectores linealmente independientes \(\{\ket{\phi_1}, \ket{\phi_2}, \dots, \ket{\phi_n}\}\) en un conjunto de vectores ortonormales \(\{\ket{e_1}, \ket{e_2}, \dots, \ket{e_n}\}\) que generan el mismo subespacio vectorial. El primer vector de la base ortonormal se obtiene normalizando \(\ket{\phi_1}\)

\[
\ket{e_1} = \frac{\ket{\phi_1}}{\Vert\phi_1\Vert},
\]

Para cada vector \(\ket{\phi_i}\) con \(i > 1\), se calcula su proyección sobre los vectores ortonormales ya obtenidos y se resta esta proyección de \(\ket{\phi_i}\) para obtener un vector ortogonal a todos los anteriores. Luego, se normaliza el resultado, es decir para \(i = 2, 3, \dots, n\) sea

\[
\ket{\psi_i} = \ket{\phi_i} - \sum_{j=1}^{i-1} \braket{e_j | \phi_i} \ket{e_j},
\]

\noindent por lo que

\[
\ket{e_i} = \frac{\ket{\psi_i}}{\Vert\psi_i\Vert}.
\]

El conjunto resultante \(\{\ket{e_1}, \ket{e_2},\dots \ket{e_n}\}\) es una base ortonormal ya que claremente son vectores de norma 1 y para todo \(k<i\)
\begin{align*}
\braket{e_j|\psi_i}&=\braket{e_k|\ket{\phi_i} - \sum_{j=1}^{i-1} \braket{e_j | \phi_i} \ket{e_j}}= \braket{e_k|\phi_i}-\sum_{j=1}^{i-1} \braket{e_j | \phi_i} \braket{e_k|e_j}\\
                   &=\braket{e_k|\phi_i}-\sum_{j=1}^{i-1} \braket{e_j | \phi_i} \delta_{kj}=\braket{e_k|\phi_i}-\braket{e_k | \phi_i}=0.
\end{align*}

\section{Espacios de Banach y Hilbert}

\noindent Al trabajar con espacios vectoriales de dimensión infinita, serán necesarias las nociones métricas y topológicas de dichos espacios, es decir a partir de ahora siempre trabajaremos en espacios donde podremos hablar de \emph{convergencia}, esto es pues trabajaremos muchas veces con \emph{series} (de Fourier o Laurent) en lugar de sumas.
Empezemos recordando las nociones de \emph{convergencia} y \emph{sucesíon de Cauchy}, sea \(X,\Vert\cdot\Vert\) un espacio normado, decimos que una sucesión \(\{\ket{x_n}\}_{n\in\nat}\subset X\) es
\begin{itemize}
  \item Convergente si existe \(\ket{\chi}\in X\) tal que para toda \(\epsilon\in\re^{+}\), existe \(N\in\nat\) tal que \(\Vert x_n-\chi\Vert<\epsilon\) si \(N<n\) y de denota \(\ket{x_n}\rightarrow\ket{\chi}\).
  \item De Cauchy si para toda \(\epsilon\in\re^{+}\), existe \(N\in\nat\) tal que \(\Vert x_n-x_m\Vert<\epsilon\) si \(N<n,\,N<m\).
  \end{itemize}
\exe Pruebe que toda susesión convergente es de Cauchy.
\begin{def.}
  A un espacio vectorial normado \((X,\Vert\cdot\Vert)\) se le llama un \emph{espacio de Banach}, si es completo en el sentido métrico, es decir, si toda sucesión de Cauchy es convergente. Igualmente a un espacio con producto interno \((H,\braket{\cdot|\cdot})\) se le llama \emph{espacio de Hilbert} si es un espacio de Banach con su norma inducida.
  \end{def.}
\eje Todo espacio vectorial sobre \(\re\) (y por lo tanto sobre \(\co\)) de dimensión finita es completo, la prueba de esto requiere un poco más de teoría pero en sí se sigue del hecho de que \(\re\) es completo.\\
La noción de convergencia, nos permite hablar de la cerradura de un conjunto para espacios normados, además permite también hablar de subconjuntos o subespacios \emph{densos}.
\begin{def.}
  Sea \(X,\Vert\cdot\Vert\) un espacio normado y sea \(\om\subset X\), definimos la cerradura de \(\om\), denotado por \(\overline{\om}\) como el conjunto de puntos límite de \(\om\), es decir, \(\ket{x}\in\overline{\om}\), si y sólo si existe una susesión \(\{\ket{x_n}\}_{n\in\nat}\subset\om\) tal que \(\ket{x_n}\rightarrow\ket{x}\).
  \end{def.}
Claramanete \(A\subset\overline{\om}\), pues podemos tomar siempre las sucesiones contantes en \(a_n=a\in\om\). Decimos que un conjunto es \emph{cerrado} si \(\om=\overline{\om}\).\\
\exe Demuestre que todo subespacio de dimensión finita es cerrado.\\

Ahora con la nociones de susesiones y convergencia, es posible hablar claramente de nociones como \emph{series de Fourier} y \emph{series de polinomios ortogonales}. Primero definimos una serie en un espacio normado como la susesion convergente de sumas finitas, donde representamos a su límite como una suma infinita, a la cual llamamos una \emph{serie convergente}
\[
    S_k=\sum_{n=1}^k\ket{\phi_n}\rightarrow S=\sum_{n=1}^{\infty}\ket{\phi_n}\text{ cuando }k\rightarrow\infty.
\]

Así podemos hablar de bases para espacios vectoriales donde en lugar de expresar todos los vectores como sumas finitas, podemos representarlos como series de vectores básicos, esto facilita mucho la forma en la que podemos representar vectores, por ejemplo definir una base de Hammel para el espacio \(\mathcal{C}[a,b]\) es muy complicado y se necesitan una cantidad no numerable de vectores, pero con series vamos a poder encontrar una base incluso numerable.

\begin{def.}
  Sea \((X,\|\cdot\|)\) un espacio vectorial normado, decimos que un subconjunto \(\beta\subset X\) es una base de Schauder si el espacio generado por \(\beta\), \(V=\langle\beta\rangle\) es denso en \(X\), es decir
  \[
    X=\overline{V}=\overline{\langle\beta\rangle}.
  \]
  En el caso de un espacio con producto interno \(H,\braket{\cdot|\cdot}\), si para todo par de vectores \(\{\ket{e_i},\ket{e_j}\}\subset\beta\), se cumple
  \[
    \braket{e_i|e_j}=\delta_{ij},
  \]
  entonces como en el caso de las bases de Hammel, decimos que \(\beta\) es una base \emph{ortonormal}.
  \end{def.}
\obs Si se aplica el proceso de Gram-Schmidt para una base de Schauder, se obtiene una base de Schauder ortonormal.\\


\section{Operadores Lineales}
\noindent Durante todo el curso trabajaremos con ecuaciones diferenciales (ordinarias o praciales) lineales, algunos ejemplos son
\begin{itemize}
    \item \textbf{Ecuación de Laplace}:
    \[
    \nabla^2 \phi = 0,
    \]
    donde \(\nabla^2\) es el Laplaciano.

    \item \textbf{Ecuación de Calor}:
    \[
    \frac{\partial u}{\partial t} = \alpha \nabla^2 u,
    \]
    donde \(u(\mathbf{x}, t)\) es la temperatura en la posición \(\mathbf{x}\) y tiempo \(t\), y \(\alpha\) es la difusividad térmica.

    \item \textbf{Ecuación de Onda}:
    \[
    \frac{\partial^2 u}{\partial t^2} = c^2 \nabla^2 u,
    \]
    donde \(u(\mathbf{x}, t)\) es la amplitud de la onda en la posición \(\mathbf{x}\) y tiempo \(t\), y \(c\) es la velocidad de propagación de la onda.

    \item \textbf{Ecuación de Schrödinger}:
   \[
   i\hbar \frac{\partial \psi(\mathbf{x}, t)}{\partial t} = \hat{H} \psi(\mathbf{x}, t)= \Big(-\frac{\hbar^2}{2m}\nabla^2+V(\mathbf{x})\Big)\psi(\mathbf{x}, t),
   \]
    donde \(\psi(\mathbf{x})\) es la función de onda, \(V(\mathbf{x})\) es el potencial, \(\hbar\) es la constante de Planck reducida y \(m\) es la masa de la partícula.

    \item \textbf{Operadores de Sturm-Liouville}:
    \[
    \mathcal{L} \phi = -\frac{d}{dx} \left( p(x) \frac{d\phi}{dx} \right) + q(x) \phi = \lambda w(x) \phi,
    \]
    donde \(\mathcal{L}\) es el operador de Sturm-Liouville, \(p(x)\), \(q(x)\) y \(w(x)\) son funciones dadas, \(\phi(x)\) es la función propia y \(\lambda\) es el valor propio asociado.
\end{itemize}

\noindent La linealidad en estas ecuaciones significa que si se nos dan dos soluciones \(\phi_1\) y \(\phi_2\) de una de estas ecuaciones (digamos, la ecuación de onda), entonces \(\lambda_1 \phi_1 + \lambda_2 \phi_2\) también es una solución para constantes arbitrarias \(\lambda_1\) y \(\lambda_2\), esto quiere decir que la abstracción adecuada para estas ecuaciones son los \emph{operadores lineales}, de esta forma podemos pensar a una ecuación diferencial en términos de \emph{operador diferencial lineal}

Con una posible excepción, la razón real por la que todas estas ecuaciones son lineales es la misma: son aproximaciones. La forma más común en que surgen las ecuaciones lineales es al perturbar ligeramente un sistema general. Independientemente de las ecuaciones complicadas que gobiernan la dinámica de la teoría subyacente, si solo consideramos el primer orden en las pequeñas perturbaciones, encontraremos una ecuación lineal esencialmente por definición. Por ejemplo, la ecuación de onda dará una buena descripción de las ondulaciones en la superficie de un estanque tranquilo o de la luz que viaja a través de un panel de vidrio.

La posible excepción es la ecuación de Schrödinger en la Mecánica Cuántica. Conocemos muchas formas de generalizar esta ecuación, como hacerla relativista o pasar a la Teoría Cuántica de Campos, pero en cada caso, el análogo de la ecuación de Schrödinger siempre permanece exactamente lineal.
\begin{def.}
  Sean \(V\) y \(W\) espacios vectoriales sobre el mismo campo \(\mathbb{F}\), se define un operador o transformación lineal entre \(V\) y \(W\) como una función \(\mathcal{L}:V\rightarrow W\) que cumple
  \[
    \mathcal{L}(\lambda_1\ket{\phi_1}+\lambda_2\ket{\phi_2})= \lambda_1\mathcal{L}\ket{\phi_1}+\lambda_2\mathcal{L}\ket{\phi_2}\quad\forall\{\ket{\phi_1},\ket{\phi_2}\}\subset V,\,\{\lambda_1,\lambda_2\}\subset\mathbb{F}
  \]
\noindent en el caso que \(W=\mathbb{F}\), en lugar de llamar a \(\mathcal{L}\) operador lineal, se le llama \emph{funcional lineal}.
\end{def.}
Recordemos que todo operador lineal \(\mathcal{L}\) tiene sus espacios núcleo (o Kernel) e imagen, los cuales se definen
\begin{itemize}
  \item El núcleo o kernel \(\textrm{Ker}(\mathcal{L})=\{\ket{\phi}\in V \,:\,\mathcal{L}\ket{\phi}=0\}\).
  \item La imagen \(\textrm{Im}(\mathcal{L})=\{\mathcal{L}\ket{\phi}\in W\,:\,\ket{\phi}\in V\}\).
  \end{itemize}
Notamos que tanto el núcleo como la imagen son espacios vectoriales.\\
\eje Sea \(\mathcal{C}^1(a,b)\) el espacio de funciones continuamente diferenciables en el intervalo abierto \((a,b)\subset\re\), entonces el operador \emph{derivada} \(D:\mathcal{C}^1(a,b)\rightarrow\mathcal{C}(a,b)\), definido como
\[
D[f](x)=\dfrac{df}{dx}(x)=f'(x).
\]
\noindent es un operador lineal. Más en general, están los \emph{polinomios diferenciales}, los cuales son expresiones de tipo
\[
    \mathcal{L}=\alpha_n(z)D^n+\dots+\alpha_1(z)D+a_0(z)I,\text{ $\alpha_i(z)$ función compleja.}
\]
\eje En espacios vectoriales de dimensión finita se pueden categorizar todos los operadores lineales entre ellos, ensí todos son matrices con entradas en al campo, en el caso de espacios complejos, toda trasformación lineal \(T:\co^n\rightarrow\co^m\) está dado por una matriz \(A\in\textrm{M}_{n\times m}(\co)\) de tal forma que \(T(Z)=AZ\) en la bases canónicas de \(\co^n\) y \(\co^m\) respectivamente. En el caso \(m=1\), tenemos los funcionales lineales, los cuales todos se evaluán pr medio de la multiplicación de un vector columna por una matriz de \(n\times 1\) columnas, es decir, si \(f:\co^n\rightarrow\co\) es un fucnional lineal, entonces
\[
f(z_1,\dots,z_n)=(\alpha_1,\dots,\alpha_n)\begin{pmatrix}z_1\\ \vdots \\ z_n\end{pmatrix}=\sum_{k=1}^n \alpha_kz_k = (\overline{\alpha}_1,\dots,\overline{\alpha}_n)\cdot(z_1,\dots,z_n).
\]
\noindent Por lo tanto podemos pensar al funcional lineal \(f\) como el vector \((\overline{\alpha}_1,\dots,\overline{\alpha}_n)\in\con\), tal que \(f(z_1,\dots,z_n)=(\overline{\alpha}_1,\dots,\overline{\alpha}_n)\cdot(z_1,\dots,z_n)\).

\begin{def.}
Sean \((X, \|\cdot\|_X)\) y \((Y, \|\cdot\|_Y)\) dos espacios normados. Un operador lineal \(T: X \to Y\) se dice \textbf{acotado} si existe una constante \(C > 0\) tal que para todo \(x \in X\) se cumple
\[
\|T(x)\|_Y \leq C \|x\|_X.
\]
La menor de tales constantes \(C\) se denomina la \textbf{norma del operador} \(T\) y se denota por \(\|T\|\).
\end{def.}
\exe Demuestre que ee puede demostrar que es posible calcular la norma de un operador acotado como
\[
\|A\|=\sup\{\|A\ket{\psi}\|\,:\,\|\psi\|=1\}=\sup\{\frac{\|A\ket{\psi}\|}{\|\psi\|}\,:\,\ket{\psi}\neq 0\}.
\]
\obs Sea \(T: X \to Y\) un operador lineal entre espacios normados. Las siguientes afirmaciones son equivalentes:
\begin{enumerate}
    \item \(T\) es continuo en \(X\).
    \item \(T\) es continuo en cero \(0 \in X\).
    \item \(T\) es Lipschitz continuo en \(X\).
    \item \(T\) es acotado.
\end{enumerate}

\eje Un ejemplo clásico de operador de Fredholm es el operador integral definido por
\[
T(f)(x) = \int_a^b K(x, y) f(y) \, dy,
\]
donde \(K(x, y)\) es un núcleo continuo en \([a, b] \times [a, b]\). Este operador actúa sobre el espacio de funciones continuas \(C([a, b])\) y es un operador de Fredholm con índice cero, observamos que como \(K\) es continua y  \([a, b] \times [a, b]\) es compacto, entonces \(\|T\|\leq(b-a)\|K\|_{\infty}\).

\obs Dados \((X,\|\cdot\|_X\) y \((Y,\|\cdot\|_Y\) espacios normados, se denota el espacio de operadores acotados como
\[
\mathcal{B}(X,Y)=\{A:X\to Y\,:\,\text{ A es acotado}\}.
\]
\noindent Este espacio es un espacio vectorial normado ya que \(\|\cdot\|_Y\) es una norma por lo tanto claramente se cumple que
\begin{itemize}
    \item \(\|A\| \geq 0\,\forall A \in\mathcal{B}(X,Y) \), con igualdad si y solo si \(\|A\ket{psi}\| = 0\,\forall\psi\in X\implies A = 0\).
    \item \(\|\lambda A\| = |\lambda|\|A\|\quad\forall\,\lambda\in\co.\)
    \item \(\|(A + B)\ket{\psi}\| \leq \|A\ket{\psi}\| + \|B\ket{\psi}\|\leq \|A\|\|\psi\| + \|B\|\|\psi\| \implies\| A + B\| \leq \|A\| + \|B\|\quad\forall\,\{A,B\}\subset\mathcal{B}(X,Y).\)
\end{itemize}

Por lo tanto los espacios \(\mathcal{B}(X,Y)\) son espacios normados, uno naturalmente se puede prefuntar ¿Cuándo son espacios de Banach? y la respuesta es bastante sencilla, sólo se necesita que \(Y\) sea de Banach, en generál resulta que los espacios de funciones continuas de muchas índoles \(\mathcal{C}(X,Y)\) son completos cuando \(Y\) también lo es, sin embargo la demostración de este hecho se sale de los propósitos del curso.

Aunado a lo anterior, si \(A:X\to Y\) y \(B:Y\to Z\) con \((Z,\|\cdot\|)\) normado, son operadores acotados entonces \(\|BA\|\leq\|A\|B\|\). Por lo tanto para operadores \(A:X\to X\), denotado \(\mathcal{B}(X)\), se cumple que
\[
\|AB\|\leq\|A\|\|B\|\,\,\forall A,B\in\mathcal{B}(X)\implies\|A^n\|\leq\|A\|^n\,\forall n\in\nat.
\]

\begin{def.}
Sea \(\mathcal{A},\|\cdot\|_{\mathcal{A}}\) un espacio de Banach tal que además de las operaciones de suma y producto por excalar, tiene una operación producto \(\mathcal{A}\times\mathcal{A}\to\mathcal{A}\) que cumple
\begin{itemize}
    \item \textbf{Distributividad:} \(\alpha(\beta+\gamma)=\alpha\beta+\alpha\gamma\quad\forall\alpha,\beta,\gamma\in\mathcal{A}\).
    \item \textbf{Coninuidad de multiplicación:} \(\|\alpha\beta\|\leq\|\alpha\|\|\beta\|\,\forall\alpha,\beta\in\mathcal{A}\)
\end{itemize}
entonces llamaremos a \(\mathcal{A}\) un \emph{álgebra de Banach}
\end{def.}

\eje Si \(X\) es un espacio de Banach, entonces \(\mathcal{B}\) es un álgebra de Banach

Ahora si  nos centramos un poco en los funcionales lineales, si \((V,\braket{\cdot|\cdot})\) es un espacio vectorial con producto interno, y \(\ket{\psi}\) es un vector fijo, notamos que el mapeo \(f_{\psi}:\ket{\phi}\mapsto\braket{\psi|\phi}\) es claramente un funcional lineal por los axiomas del producto interno, más aún por la desigualdad de Cauchy-Schwarz, para toda \(\ket{\phi}\), se tiene
\[
|f_{\psi}\ket{\phi}|=|\braket{\psi|\phi}|\leq\|\psi\|\,\|\phi\|\,\implies\,\|f_{\psi}\|\leq\|\psi\|
\]
\begin{def.}
Sea \(X,\|\cdot\|\) un espacio vectorial normado, definimos el espacio dual como \(V^*=\{f:X\rightarrow\co\,:\,f \text{es linal y continua}\}\)
\end{def.}

Por lo anterior, entonces se establece que en un espacio con producto interno \((H,\braket{\cdot|\cdot})\), existe un operador lineal acotado \(\varphi:H\rightarrow H^*\)

\begin{teorema}{Representación de Riez}
  Sea \((H,\braket{\cdot|\cdot})\) un espacio de Hilbert, entonces el operador lineal acotado \(\varphi:H\rightarrow H^*\) definido por
  \[
  \varphi\ket{\psi}=f_{\psi}\quad f_{\psi}\ket{\phi}=\braket{\psi|\phi}
  \]
  es un \emph{isomorfimo isométrico}.
  \end{teorema}
A partir de ahora nos olvisaremos de la notación \(f_{\psi}\) y simplemente usaremos la notación clásica de Dirac \(\bra{\psi}\).

\obs Si \((H,\braket{\cdot|\cdot})\) es un espacio con producto interno no completo, como hemos mencionado \(\co\) es de Banach y por lo tanto \(H^{*}\) es también un espacio de Banach, por lo tanto no es posible que \(H\cong H^{*}\) ya que esto implicaría que \(H\) sería completo.

\subsection{Criterios de convergencia de series}
\noindent Ya hemos mencionado sobre convergencia de series, sin embargo determinar cuando una serie es convergente o no requiere de un meticuloso análisis. Recordemos que hemos definimos una serie en un espacio normado como la susesion convergente de sumas finitas, donde representamos a su límite como una suma infinita \(\sum_{n=0}^{\infty}\ket{\psi_n}\), la cual decimos que es una \emph{serie convergente} si
\[
    S_k=\sum_{n=1}^k\ket{\phi_n}\rightarrow S=\sum_{n=1}^{\infty}\ket{\phi_n}\text{ cuando }k\rightarrow\infty.
\]
\noindent Más aún si la serie de números reales definidos por sus normas  \(\sum_{n=0}^{\infty}\|\psi_n\|\) converge, entonces diremos que la serie converge \emph{absolutamente}.
Por lo tanto aqui se presentan los criterios más comunes de convergencia de series para espacios de \emph{Banach}, por lo que todos los epacios que consideraremos en esta sección son de esta índole. El primer criterio y además el más útil es el \emph{criterio de Cauchy}
\begin{prop}{Criterio de Cauchy}
  Sea \(\sum_{n=0}^{\infty}\ket{\phi_n}\) una serie en un espacio de Banach, ésta es convergente si y sólo si \(\forall\epsilon\in\re^{+}\) existe \(N\in\nat\) tal que si \(n>N\)
  \[
    \|\sum_{m=n+1}^{n+k}\ket{\phi_m}\|<\epsilon\quad\forall k\in\nat
  \]
\end{prop}
\dem La demostración se sigue del hecho de que al estar trabajando en espacio completo entonces son equivalentes para la susesión de sumas parciales
\[
    S_n=\sum_{m=0}^{n}\ket{\phi_m}\text{ es convergente }\Leftrightarrow\,\{S_n\}_{n\in\nat}\text{ es de Cauchy}.
\]
\noindent Por lo tanto, al ser de Cauchy, para todo \(\epsilon\in\re^{+}\) existe \(N\in\nat\) tal que si \(n>N\)
\[
   \|S_n-S_{n+k}\|=\|\sum_{m=0}^{n}\ket{\phi_m}-\sum_{m=0}^{n+k}\ket{\phi_m}\|=\|\sum_{m=n+1}^{n+k}\ket{\phi_m}\|<\epsilon\quad\forall k\in\nat.
\]
\obs Si se aplica el criterio anterior al caso \(k=1\), entonces tenemos que los elementos de una serie necesariamente deben cumplir
\[
    \ket{\psi_n}\rightarrow 0\text{ cuando } n\to\infty
\]
si no, la serie \emph{diverge}.
\QED
\begin{prop}
  Si la serie \(\sum_{n=0}^{\infty}\ket{\psi_n}\) converge absolutamente, entonces converge.
  \end{prop}
\dem Aplicando el criterio de Cauchy para la serie \(\sum_{n=0}^{\infty}\|\psi_n\|\) y la desigualdad del triángulo, entonces sea \(\epsilon\in\re\) y \(N\in\nat\) tal que para todo \(n>N\)
\[
 \|\sum_{m=n+1}^{n+k}\ket{\psi_m}\|\leq \sum_{m=n+1}^{n+k}\|\psi_m\|<\epsilon\quad\forall k\in\nat
\]
entonces por el criterio de Cauchy, \(\sum_{n=0}^{\infty}\ket{\psi_n}\) converge.
\QED\\
Ahora el siguiente teorema nos permite analizar las series en espacios de Banach como series de números reales
\begin{teorema}{Prueba M de Weierstrass}
  Sea \(\{\ket{\phi_n}\}_n\) una susesión en un espacio de Banach y \(\{M_n\}_{n\in\nat}\subset\re^{+}\) una sucesión real positiva tal que se satisface
  \begin{align*}
    &i)\,\|\phi_n\|\leq M_n \quad\forall n\in\nat.\\
    &ii)\sum_{n=0}^{\infty}M_n\text{ converge}.
    \end{align*}
  entonces la serie de vectores
  \[
    \sum_{n=0}^{\infty}\ket{\phi}
  \]
  converge.
\end{teorema}
\dem Como \(\sum_{n=0}^{\infty}M_n\) converge, al igual que en la proposición anterior aplicamos el criterio de Cauchy y la desigualdad del triangulo, entonces sea \(\epsilon\in\re\) y \(N\in\nat\) tal que para todo \(n>N\)
\[
 \|\sum_{m=n+1}^{n+k}\ket{\psi_m}\|\leq \sum_{m=n+1}^{n+k}\|\psi_m\|\leq \sum_{m=n+1}^{n+k}M_m<\epsilon\quad\forall k\in\nat
\]
entonces por el criterio de Cauchy, \(\sum_{n=0}^{\infty}\ket{\psi_n}\) converge absolutamente.
\QED\\
\eje Sea \(A\) un operador acotado en un espacio de banach,entonces el operador \(\exp(A)\) existe y además es acotado y su norma cumple que \(\|\exp(A)\|\leq e^{\|A\|}\), donde \(\exp(A)\) se define como la serie con sumas parciales
\[
S_k=\sum_{n=0}^{k}\frac{A^n}{n!}\implies\exp(A)=\sum_{n=0}^{\infty}\frac{A^n}{n!}.
\]
entonces por la desigualdad del triangulo y el hecho de que para todo \(\rho\in\re\) la serie exponencial converge
\[
e^{\rho}=\sum_{n=0}^{\infty}\frac{\rho^n}{n!},
\]
tenemos que \(\|A^n/n!\|\leq\|A\|^n/n!=M_n\), entonces por la prueba M, la serie definida por las sumas \(S_k\) convergen y por lo tanto \(\exp(A)\) existe y
\[
\|\exp(A)\|=\Big\|\sum_{n=0}^{\infty}\frac{A^n}{n!}\Big\|\leq\sum_{n=0}^{\infty}\Big\|\frac{A^n}{n!}\Big\|\leq\sum_{n=0}^{\infty}\frac{\|A\|^n}{n!}=e^{\|A\|}.
\]

\subsubsection{Criterios de convergencia de series reales.}
\noindent El teorema de la prueba M de Weierstrass implica que la convergencia de muchas series en espacios de Banach se pueden determinar de la convergencia de series de números reales, por lo tanto recordamos alguos criterios comunes de series reales. Primero veamos un ejemplo muy útil

\eje Si \(\rho\in[0,1)\), entonces la \emph{serie geométrica} converge, donde se define y calcula la serie geométrica como
\[
\sum_{n=0}^{\infty}\rho^n=\frac{1}{1-\rho}.
\]
Donde por inducción se demuestra que la serie de sumas parciales es
\[
s_n=\sum_{k=0}^{n}\rho^k=\frac{1-\rho^{n+1}}{1-\rho}\to\frac{1}{1-\rho}\text{ cuando }n\to\infty,\,\rho<1.
\]
Por lo tanto, esta serie \emph{diverge} si \(\rho=1\). Una observación es que entonces para operadores en espacios de Banach, si \(\|A\|=1\), entonces la convergencia de la serie
\[
\sum_{n=0}^{\infty}A^n.
\]
\text{NO} puede ser determinada por la prueba M, pero \emph{no significa que converja}.


\begin{prop}{Criterio de Comparación}
Sean \(\sum_{n=1}^\infty a_n\) y \(\sum_{n=1}^\infty b_n\) dos series tales que \(0 \leq a_n \leq b_n\) para todo \(n\). Si \(\sum_{n=1}^\infty b_n\) converge, entonces \(\sum_{n=1}^\infty a_n\) también converge
\end{prop}
\dem Dado que \(0 \leq a_n \leq b_n\), las sumas parciales de \(\sum a_n\) están acotadas por las sumas parciales de \(\sum b_n\). Como \(\sum b_n\) converge, sus sumas parciales están acotadas. Por lo tanto, las sumas parciales de \(\sum a_n\) también están acotadas y, por el teorema de convergencia monótona, \(\sum a_n\) converge.


\begin{prop}{Criterio del Cociente}
Sea \(\sum_{n=1}^\infty a_n\) una serie con \(a_n > 0\) para todo \(n\). Si existe un límite:
    \[
    L = \lim_{n \to \infty} \frac{a_{n+1}}{a_n},
    \]
entonces:
\begin{itemize}
    \item Si \(L < 1\), la serie converge.
    \item Si \(L > 1\), la serie diverge.
    \item Si \(L = 1\), el criterio no es concluyente.
\end{itemize}
\end{prop}
\dem Supongamos \(L < 1\). Entonces, existe \(N \in \mathbb{N}\) y \(r \in (L, 1)\) tal que para todo \(n \geq N\),
    \[
    \frac{a_{n+1}}{a_n} < r.
    \]
    Por lo tanto, para \(n \geq N\),
    \[
    a_{n+1} < r a_n.
    \]
    Iterando, obtenemos:
    \[
    a_{N+k} < r^k a_N.
    \]
    La serie \(\sum_{k=0}^\infty r^k a_N\) es una serie geométrica convergente (pues \(r < 1\)), y por el criterio de comparación, \(\sum_{n=1}^\infty a_n\) converge.

    Si \(L > 1\), existe \(N \in \mathbb{N}\) tal que para todo \(n \geq N\),
    \[
    \frac{a_{n+1}}{a_n} > 1.
    \]
    Esto implica que \(a_{n+1} > a_n\), por lo que los términos no tienden a cero y la serie diverge.
\QED

\begin{prop}{Criterio de la Raíz}:
    Sea \(\sum_{n=1}^\infty a_n\) una serie con \(a_n \geq 0\) para todo \(n\). Si existe un límite:
    \[
    L = \lim_{n \to \infty} \sqrt[n]{a_n},
    \]
    entonces:
    \begin{itemize}
        \item Si \(L < 1\), la serie converge.
        \item Si \(L > 1\), la serie diverge.
        \item Si \(L = 1\), el criterio no es concluyente.
    \end{itemize}
\end{prop}
\dem Supongamos \(L < 1\). Entonces, existe \(N \in \mathbb{N}\) y \(r \in (L, 1)\) tal que para todo \(n \geq N\),
    \[
    \sqrt[n]{a_n} < r.
    \]
    Por lo tanto, \(a_n < r^n\) para \(n \geq N\). La serie \(\sum_{n=N}^\infty r^n\) es una serie geométrica convergente (pues \(r < 1\)), y por el criterio de comparación, \(\sum_{n=1}^\infty a_n\) converge.

    Si \(L > 1\), existe \(N \in \mathbb{N}\) tal que para todo \(n \geq N\),
    \[
    \sqrt[n]{a_n} > 1.
    \]
    Esto implica que \(a_n > 1\), por lo que los términos no tienden a cero y la serie diverge.
\QED

\begin{def.}{limsup y liminf}
\end{def.}


\subsubsection{Propiedades de las Series Convergentes}
\noindent\textbf{Linealidad}:
Si \(\sum_{n=1}^\infty a_n = A\) y \(\sum_{n=1}^\infty b_n = B\), entonces para cualquier \(\alpha, \beta \in \mathbb{R}\),
    \[
    \sum_{n=1}^\infty (\alpha a_n + \beta b_n) = \alpha A + \beta B.
    \]
\dem Sean \(S_N = \sum_{n=1}^N a_n\) y \(T_N = \sum_{n=1}^N b_n\). Entonces:
    \[
    \sum_{n=1}^N (\alpha a_n + \beta b_n) = \alpha S_N + \beta T_N.
    \]
Tomando el límite cuando \(N \to \infty\), obtenemos:
    \[
    \lim_{N \to \infty} \left( \alpha S_N + \beta T_N \right) = \alpha A + \beta B.
    \]
\QED\\
\noindent\textbf{Multiplicación de Series (Teorema de Mertens)}:
    Si \(\sum_{n=1}^\infty a_n = A\) y \(\sum_{n=1}^\infty b_n = B\) son series absolutamente convergentes, entonces su producto de Cauchy converge a \(AB\), donde el producto de Cauchy se define como:
    \[
    \sum_{n=1}^\infty c_n, \quad \text{con } c_n = \sum_{k=1}^n a_k b_{n-k}.
    \]
\dem
\begin{teorema}
  Supongase que \(\{a_n\}_{n\in}\nat\) es una susesión de números positivos decreciente \(a_1\geq a_2\geq a_3\geq\dots\geq a_n\geq\dots\geq0\),entonces
  \[
   \sum_{n=1}^\infty a_n\text{ converge }\iff  \sum_{k=0}^\infty 2^ka_{2^k}\text{ converge}
  \]
\end{teorema}
\dem
\eje La función \(\zeta\)

\section{Teoría de la Medida}
\chapter{Series de Fourier}
\noindent Ahora tenemos las herramientas para establecer las \emph{series de Fourier} como aproximadores de funciones, sin embargo para hacer esto, haremos uso de un teorema muy importante en el análisis matemático, llamado el \emph{Teorema de Stonne-Weierstrass}, por lo que necesitamos las iguientes definiciones

\begin{def.}{Álgebra de Funciones}
    Sea \(\om\subset\re^n\) un espacio compacto. Un conjunto \(\mathcal{A}\) de funciones continuas \(f: \om \to \mathbb{R}\) (o \(\mathbb{C}\)) se llama una \emph{álgebra} si es cerrado bajo:
    \begin{itemize}
        \item Suma: \(f + g \in \mathcal{A}\) para todo \(f, g \in \mathcal{A}\).
        \item Producto: \(f \cdot g \in \mathcal{A}\) para todo \(f, g \in \mathcal{A}\).
        \item Multiplicación por escalares: \(\alpha f \in \mathcal{A}\) para todo \(\alpha \in \mathbb{R}\) (o \(\mathbb{C}\)) y \(f \in \mathcal{A}\).
    \end{itemize}
\end{def.}
\begin{def.}{Conjunto de funciones que Separan Puntos}
Un conjunto \(\mathcal{A}\) de funciones en \(\om\) se dice que \textbf{separa puntos} si para todo par de puntos distintos \(x, y \in \om\), existe una función \(f \in \mathcal{A}\) tal que \(f(x) \neq f(y)\).
\end{def.}

\begin{teorema}{Teorema de Stone-Weierstrass}
Sea \(\om\subset\re^n\) un espacio compacto y \(\mathcal{A}\) una subálgebra de \(C(\om, \mathbb{R})\) (o \(C(\om, \mathbb{C})\)), el espacio de todas las funciones continuas en \(\om\) con valores en \(\mathbb{R}\) (o \(\mathbb{C}\)). Si \(\mathcal{A}\) satisface:
\begin{enumerate}
    \item \(\mathcal{A}\) contiene las funciones constantes.
    \item \(\mathcal{A}\) separa puntos.
\end{enumerate}
Entonces, \(\mathcal{A}\) es \textbf{densa} en \(C(\om, \mathbb{R})\) (o \(C(\om, \mathbb{C})\)) con respecto a la norma del supremo. Es decir, para toda función continua \(f \in C(\om, \mathbb{R})\) (o \(C(\om, \mathbb{C})\)) y para todo \(\epsilon > 0\), existe una función \(g \in \mathcal{A}\) tal que:
\[
\|f - g\|_\infty < \epsilon.
\]
\end{teorema}



\begin{thebibliography}{9}

\bibitem{Wein} H.F. Weinberger ``A First Course In Partial Differential Equations With Complex Variables and Transform Methods'', Dover Books On Mathematics, New York.
\bibitem{Lebedev} N.N. Lebedev ``Special Functions \& Their Applications'', Dover Books On Mathematics, New York.
\bibitem{Afken} Arfken, George B. et all, ``Mathematical Methods for Physicist, A Comprehensive Guide.'', Ed. Elsevier, Seventh Edition.
\bibitem{Appfel} Appfel, Walter, ``Mathematics for Physics \& Physicists.'', Ed. Princeton University Press, 2007.
\bibitem{Churchill} Churchill R. V. and Brown J. W., ``Fourier Series and Boundary Value Problems'', Ed McGraw-Hill.
\bibitem{Cour-Hil} Courant R and D. Hilbert, ``Methods of Mathematical Physics.'', vol. 1 y 2.
\bibitem{Tijoniv} Tijonov A, ``Ecuaciones de la Física Matemática'', Ed. MIR Moscu, Segunda edición, 1980.
\bibitem{Vladimiriv} Vladimirov V. S. , ``Equations of Mathematical Physics'', Ed. URSS Moscow, 1996.
\bibitem{Kreyszig} E. Kreyszig ``Introductory Functional Analysis With Applications'', John Wily \& Sons.

\end{thebibliography}

\end{document}

\documentclass[main.tex]{subfiles}
\begin{document}
\chapter{Teor'ia de Sturm-Liouville}
\noindent Si recordamos la motivaci'on para el estudio de las series de Fourier, intentamos encontrar soluciones generales para la \emph{ecuación de Schrödinger en una dimensión} con potencial \(V(x)\) \emph{independiente del tiempo}. Para resolverla utilizamos el \emph{método de separación de variables o método de Bernoulli}, el cual divide el problema en dos ecuaciones,

\begin{enumerate}
    \item \textbf{Ecuación Temporal}:
    \[
    i\hbar \dfrac{d \phi(t)}{dt} = E \phi(t).
    \]
    Cuya solución es
    \[
    \phi(t) = \phi(0) e^{-iEt/\hbar}.
    \]

    \item \textbf{Ecuación Espacial}:
    \[
    -\frac{\hbar^2}{2m} \dfrac{d^2 \varphi(x)}{dx^2} + V(x) \varphi(x) = E \varphi(x).
    \]
    La cual define el \emph{operador hamiltoniano cuántico}
    \[
    H =-\frac{\hbar^2}{2m} \dfrac{d^2 }{dx^2} + V(x)I,
    \]
    además \(E\) representa la energía del estado estacionario, la energía cumplen con una ecuación del tipo
    \[
    H\varphi=E\varphi,
    \]
    también conocida como ecuación de vectores propios o \emph{ecuación espectral}.
\end{enumerate}
La ecuaci'on  espectral para el operador hamiltoniano con un potencial independiente del tiempo, es es el tipo de ecuaciones y el tipo de operadores que estudiaremos en esta secci'on, estos operadores son conocidos como los operadores de \emph{Sturm-Liouville}. Estos operadores provienen de manera natural de la resoluci'on de ecuaciones diferenciales parcialers con ciertas condiciones de frontera, de la misma forma que cuando resolvimos la ecuaci'on de Schrödinger. El nombre de esta teor'ia y operadores proviene de los matem'aticos \emph{Jaques Sturm} y \emph{Joseph Liouville} quienes estudiaron este tipo de ecuaciones en el siglo XIX.

\begin{def.}{Operadores de Sturm-Liouville}
  Sean \(p(x)\) y \(q(x)\)son funciones dadas, se define \(\mathcal{L}\) el operador de Sturm-Liouville como
  \[
    \mathcal{L}= -\frac{d}{dx} \left( p(x) \dfrac{d}{dx} \right) + q(x)I.
  \]
  Ahora, estudiaremos el caso cuando \(\phi(x)\) es una función propia y \(\lambda\) es el valor propio asociado con un peso \(w(x)\), es decir la ecuaci'on
  \[
    \mathcal{L}\phi(x)= -\frac{d}{dx}\left( p(x) \dfrac{d\phi}{dx}(x)\right) + q(x)\phi(x)=\lambda w(x)\phi(x).
  \]
\end{def.}
\section{Ecuaciones diferenciales lineales de orden dos.}
\noindent Comenzamos con un repaso de ecuaciones diferenciales ordinarias de orden dos, as'i, cc consideramos la ecuación diferencial ordinaria de segundo orden en el intervalo real \( I \), dada por
\[
a_0(x)y''+a_1(x)y'+a_2(x)y= f(x),
\]
donde \(a_0\), \(a_1\), \(a_2\) y \(f\) son funciones complejas definidas en \(I\). Cuando \(f\equiv 0\) en \(I\), la ecuación se llama \textbf{homogénea}; de lo contrario, se llama \textbf{no homogénea}. Cualquier función (compleja) \(\phi \in C^2(I)\) es una solución de la ecuación si la sustitución de \(y\) por \(\phi\) resulta en la identidad

\[
a_0(x)\phi''(x)+a_1(x)\phi'(x)+a_2(x)\phi(x)=f(x) \quad \text{para todo }x\in I.
\]

Si denotamos el operador diferencial de segundo orden

\[
L=a_0(x)\frac{d^2}{dx^2}+a_1(x)\frac{d}{dx}+a_2(x)I,
\]

entonces la ecuaci'on la escribimos la forma \(Ly= f\). El operador \(L\) es lineal, es decir,
\[
L(c_1\phi+c_2\psi)=c_1L\phi+c_2 L\psi\quad\forall\phi,\psi \in C^2(I),\,c_1,c_2\in\co
\]

De igual manera si

\[
L\phi = 0, \quad L\psi=0,
\]

entonces

\[
L(c_1\phi+c_2\psi)=c_1L\phi+c_2L\psi=0,
\]

para cualquier par de constantes \(c_1\) y \(c_2\). Esto se conoce como el \textbf{principio de superposición}.
Si la función \(a_0\) no se anula en ningún punto de \(I\), la ecuación puede dividirse por \(a_0\) para obtener
\[
y'' + p(x) y' + q(x) y = g(x),\quad p(x)=\frac{a_1(x)}{a_0(x)},\,q(x)=\frac{a_2(x)}{a_0(x)},\,g(x)=\frac{f(x)}{a_0(x)}.
\]
Claramente ambas ecuaciones son equivalentes, en el sentido de que tienen el mismo conjunto de soluciones. La ecuación se dice entonces \textbf{regular} en \(I\); de lo contrario, si existe un punto \(c\in I\) donde \(a_0(c)=0\), la ecuación es \textbf{singular}, y \(c\) se denomina \textbf{punto singular} de la ecuación.

De acuerdo con el teorema de existencia y unicidad para ecuaciones lineales, si las funciones \(q\), \(r\) y \(g\) son continuas en \(I\) y \(x_0\) es un punto en \(I\), entonces, para cualesquiera dos números \(\xi\) y \(\eta\), existe una única solución \(\phi\) en \(I\) tal que
\[
\phi(x_0) = \xi, \quad \phi'(x_0) = \eta.
\]

A continuación, enumeramos algunas propiedades bien conocidas de las soluciones de la ecuación, que pueden encontrarse en muchas introducciones estándar a las ecuaciones diferenciales ordinarias.

\begin{enumerate}
    \item La ecuación homogénea
    \[
    y''+p(x)y'+q(x)y=0,\quad x\in I,
    \]
    tiene dos soluciones linealmente independientes \(y_1(x)\) y \(y_2(x)\) en \(I\). Una combinación lineal de estas dos soluciones
    \[
    c_1y_1+c_2y_2,
    \]
    donde \(c_1\) y \(c_2\) son constantes arbitrarias, es la \textbf{solución general}; es decir, cualquier solución de la ecuación está dada por esta combinaci'on para algunos valores de \(c_1\) y \(c_2\). Cuando \(c_1=c_2=0\), obtenemos la llamada \textbf{solución trivial} la cual, siempre es una solución de la ecuación homogénea. Por el teorema de unicidad, es la única solución si \(\xi=\eta=0\).

    \item Si \(y_p(x)\) es cualquier solución particular de la ecuación no homogénea, entonces
    \[
    y_p+c_1y_1+c_2y_2
    \]
    es la solución general de la ecuaci'on no homogenea. Aplicando las condiciones iniciales, las constantes \(c_1\) y \(c_2\) se determinan y obtenemos la solución única del sistema de ecuaciones.

    \item Cuando los coeficientes \(p\) y \(q\) son constantes, la solución general de la ecuación tiene la forma
    \[
    y(x)=c_1 e^{m_1 x} + c_2 e^{m_2 x},
    \]
    donde \(m_1\) y \(m_2\) son las raíces del polinomio caracter'istico \(m^2+pm+q=0\) cuando las raíces son distintas. Si \(m_1=m_2=m\), entonces por variaci'on de par'ametros, la solución toma la forma
    \[
    y(x)=c_1 e^{m x}+c_2xe^{m x},
    \]
    donde las funciones \(e^{mx}\) y \(xe^{mx}\) son claramente linealmente independientes.

    \item Cuando \(a_0(x)=x^2\), \(a_1(x)=ax\) y \(a_2(x)=b\), donde \(a\) y \(b\) son constantes, la versión homogénea de la ecuación se convierte en
    \[
    x^2y''+axy'+by=0,
    \]
    que se llama la \textbf{ecuación de Cauchy-Euler}, su solución general está dada por
    \[
    y(x)=c_1 x^{m_1}+c_2 x^{m_2},
    \]
    donde \(m_1\) y \(m_2\) son las raíces distintas de \(m^2+(a-1)m+b=0\). De igual manera usando variaci'on de par'ametros, cuando \(m_1=m_2=m\), la solución es
    \[
    y(x)=c_1 x^m+c_2x^m\log x.
    \]

    \item El \emph{teorema de Cauchy-Kovaleskaya} asegura que si los coeficientes \(q(x)\) y \(r(x)\) son funciones analíticas en algún punto \(x_0\) en el interior de \(I\), es decir, cada uno puede representarse en un intervalo abierto centrado en \(x_0\) por una serie de potencias en \((x-x_0)\), entonces la solución general también es analítica en \(x_0\) y se representa por una serie de potencias de la forma
    \[
    y(x)=\sum_{n=0}^\infty c_n (x - x_0)^n\text{ alrededor de }x_{0}.
    \]
    La serie converge en la intersección de los dos intervalos de convergencia (de \(q(x)\) y \(r(x)\)) e \(I\). El \emph{m'etodo de Frobenius} nos dice que si sustituimos esta serie en la ecuación, podemos expresar los coeficientes \(c_n\), para todo \(n\in\{2, 3, 4,\dots\}\), en términos de \(c_0\) y \(c_1\), que permanecen arbitrarios, veremos m'as de esto cuando veamos las \emph{funciones de Bessel}.
\end{enumerate}

Con \(I=[a,b]\), las soluciones de la ecuación pueden estar sujetas a condiciones de frontera en \(a\) y \(b\). Estas pueden tomar una de las siguientes formas

\begin{enumerate}
    \item[(i)] \(y(c)=\xi\), \(y'(c)=\eta\), \(c\in\{a,b\}\),
    \item[(ii)] \(y(a)=\xi\), \(y(b)=\eta\),
    \item[(iii)] \(y'(a)=\xi\), \(y'(b)=\eta\).
\end{enumerate}

Cuando las condiciones de frontera se dan en el mismo punto \(c\), como en \((i)\), a menudo se denominan \textbf{condiciones iniciales}, como se mencionó anteriormente. Para obtener una solución única de la ecuación, el punto \(c\) no necesita, en general, ser uno de los extremos del intervalo \(I\), y puede ser cualquier punto interior. Pero en esta presentación, como en la mayoría de las aplicaciones físicas, las condiciones de contorno (o iniciales) siempre se imponen en los extremos de \(I\). Las formas \((i)\) a \((iii)\) de las condiciones de contorno pueden generalizarse mediante el par de ecuaciones

\begin{align*}
&\alpha_1 y(a)+\alpha_2 y'(a)+\alpha_3y(b)+\alpha_4y'(b)=\xi,\\
&\beta_1 y(b)+\beta_2y'(b)+\beta_3y(a)+\beta_4y'(a)=\eta,
\end{align*}

donde \(\alpha_i\) y \(\beta_i\) son constantes que satisfacen \(\sum_{i=1}^4|\alpha_i|>0\) y \(\sum_{i=1}^4|\beta_i|>0\), es decir, tales que no todos los \(\alpha_i\) o \(\beta_i\) son cero. El sistema de ecuaciones anterior se llama un \textbf{problema de valores de frontera}.

Las condiciones de frontera se llaman \textbf{homogéneas} si \(\xi=\eta=0\), y \textbf{separadas} si \(\alpha_3=\alpha_4=\beta_3=\beta_4=0\). Las condiciones de frontera separadas, que tienen la forma

\[
\alpha_1y(a)+\alpha_2y'(a)=\xi,\quad\beta_1y(b)+\beta_2y'(b)=\eta,
\]

son de particular importancia en este estudio. Otro par importante de condiciones homogéneas, que resultan de una elección especial de los coeficientes, está dado por

\[
y(a)=y(b),\quad y'(a)=y'(b).
\]

Las ecuaciones se llaman \textbf{condiciones de contorno periódicas} y significa que las soluciones se pueden expresar en t'erminos de sus \emph{series de Fourier}. Nótese que las condiciones periódicas están acopladas, no separadas.
\begin{def.}[Wronskiano]
Para cualesquiera dos funciones \(f,g\in C^1\), el determinante
\[
W(f, g)(x) = \begin{vmatrix}
f(x) & g(x) \\
f'(x) & g'(x)
\end{vmatrix} = f(x)g'(x) - g(x)f'(x)
\]
se llama el \textbf{Wronskiano} de \( f \) y \( g \). El símbolo \( W(f, g)(x) \) a veces se abrevia como \( W(x) \).
\end{def.}

El Wronskiano deriva su importancia en el estudio de ecuaciones diferenciales de los siguientes lemas.

\begin{lema}
Si \(y_1\) y \(y_2\) son soluciones de la ecuación homogénea
\begin{equation}\label{eq-homogenea}
y''+p(x)y'+q(x)y=0,\quad x\in I,
\end{equation}
donde \(q in C(I)\), entonces o bien \(W(y_1,y_2)(x)=0\) para todo \(x\in I\), o bien \(W(y_1,y_2)(x)\neq0\) para cualquier \(x\in I\).
\end{lema}
\dem Recodamos que \(W(y_{1},y_{2})(x)=y_{1}(x)y'_{2}(x)-y'_{1}(x)y_{2}(x)\), entonces al clacular su derivada se obtiene que
\[
  W'(y_{1},y_{2})=(y'_{1}y'_{2}+y_{1}y^{''}_{2})-(y^{''}_{1}y_{2}+y'_{1}y'_{2})=y_{1}y^{''}_{2}-y^{''}_{1}y_{2}.
\]
Ahora como \(y_{1}\),\(y_{2}\) son soluciones de la ecuaci'on homogenea \ref{eq-homogenea}, entonces
\begin{align*}
  y_1''+p(x)y_1'+q(x)y_1=0\\
  y_2''+p(x)y_2'+q(x)y_2=0.\\
\end{align*}
Por lo que si multipllicamos por \(y_{2}\) a la primera ecuaci'on y por \(y_{1}\) a la primera, obtenemosl al restarlas
\begin{align*}
  & y''_{2}(x)y_1(x)-y(x)_{2}y''_1(x) + p(x)(y'_{2}y_1-y_{2}(x)y_1'(x))=0\\
  &\text{entonces } W'(y_{1},y_{2})(x) + p(x)W(y_{1},y_{2})(x)= 0.\\
\end{align*}
As'i al integrar se obtiene que
\[
  W(y_{1},y_{2})=c\exp\left(-\int_{a}^{x}p(t)\,dt\right)\quad x\in I.
\]
\QED\\
\obs Las consecuencias del lema anterior son que las soluciones de la ecuaci'on \ref{eq-homogenea} o son \emph{linealmente independientes} o son linealmente dependientes.
\section{Los ceros de las soluciones}
\noindent Resolver ecuaciones diferenciales es en general bastante complicado e incluso imposible, por lo menos explicitamente en t'erminos de funciones elementales. Sin embargo es posible obterer informaci'on qualitativa de las soluciones, por ejemplo es posible determinar ortogonalidad de las soluciones. Este es un tema amplio en el estudio de las ecuaciones diferenciales, sin embargo en esta secci'on estudiaremos los ceros de la ecuaciones lineales de orden dos, ya que incluso estas no siempre tienen soluciones en t'erminos de funciones elementales. En concreto primero estudiaremos los ceros de las soluciones a ecuaciones del tipo

\begin{equation}
y''+p(x)y'+q(x)y=0,\quad x\in I,
\end{equation}

Primero, estudiamos el ejemplo protot'ipico de estas ecuaciones (uno que s'i es \emph{soluble}) y luego veremos que el comportamieto de sus ceros son t'ipicos para este tipo de ecuaciones.

\eje La ecuaci'on de segundo grado m'as simple
\[
  y''(x)+y(x)=0\quad x\in\re.
\]
Cuyas soluciones linealmente independientes son \(\cos(x)\) y \(\sin(x)\) (las bases de los fen'omenos ondulatorios), ya que como podemos observar
\[
  W(\cos,\sin)(x)=\cos^{2}(x)+\sin^{2}(x)=1,
\]
es decir en general las soluciones son de la forma
\[
  y(x)=c_{1}\cos(x)+c_{2}\sin(x).
\]
Ahora si nos fijamos \(\sin(x)=0\) si \(x\in\{n\pi\,:\,n\in\zah\}\) y \(\cos(x)=0\) si y s'olo si \hbox{\(x\in\{(2k+1)/2\pi\,:\,k\in\zah\}\)}. Es decir, por la forma ondulatoria de las soluciones, los ceros son \emph{aisaldos e intercalados}
\[
  \dots-\frac{3\pi}{2}<-\pi<-\frac{\pi}{2}<0<\frac{\pi}{2}<\pi<\frac{3\pi}{2}<\dots
\]
Ahora, si modificamos esta ecuaci'on un poco a la ecuaci'on
\[
  y''(x)+ky(x)=0\quad k\in\re\text{ constante y }x\in\re,
\]
si \(k>0\) las soluciones son practicamente las mismas y s'olo se multiplica por \(k\) en el argumento y los ceros siguen siendo aislados e intercalados, sin embargo si \(k<0\), entonces las soluciones son \(e^{\sqrt{|k|}x}\) y \(e^{-\sqrt{|k|}x}\) la cuales nunca se anulan y las combiaciones lineales tienen a lo m'as un cero.

\begin{lema}
  Si \(y(x)\) es una soluci'on no trivial de la ecuaci'on \ref{eq-homogenea}, entonces los ceros de \(y(x)\) son aislados en \(I\).
\end{lema}
\dem Si \(y(x_{0})=0\), entonces \(y'(x_{0})\neq0\) pues por unicidad, si la derivada se anula, \(y\) ser'ia la soluci'on trivial, por lo que \(y'(x)\neq\) en un subinvervalo de \(I\) \((\epsilon-x_{0},\epsilon-x_{0})\), lo cual implica que alredor de \(x_{0}\), \(y\) es \textbf{extrictamente} creciente o decreciente y por lo que no hay ning'un otro cero en dicho intervalo.
\QED\\

\begin{teorema}[Teorema de separaci'on de Sturm]
Si \(y_1\) y \(y_2\) son soluciones linealmente independientes de la ecuación
\[
y''+p(x)y'+q(x)y=0, \quad x\in I,
\]
entonces los ceros de \(y_1\) son distintos de los de \(y_2 \), y las dos secuencias de ceros se alternan, es decir, \(y_1\) tiene exactamente un cero entre dos ceros sucesivos de \(y_2\), y viceversa.
\end{teorema}
\dem Dado que \(y_1\) y \(y_2\) son linealmente independientes, su Wronskiano
\[
W(y_1,y_2)(x)=y_1(x)y_2'(x)-y_2(x)y_1'(x)
\]
no se anula y, por lo tanto, tiene un signo constante en \(I\). Nótese primero que \(y_1\) y \(y_2\) no pueden tener un cero común, de lo contrario \(W\) se anularía ahí. Entonces, suponemos que \(x_1\) y \(x_2\) son dos ceros sucesivos de \(y_2\) respectivamente, entonces
\begin{align*}
W(x_1)&=y_1(x_1)y_2'(x_1)\neq0,\\
W(x_2)&=-y_1(x_2) y_2'(x_2)\neq0,
\end{align*}
y los números \(y_1(x_1)\), \(y_1(x_2)\), \(y_2'(x_1)\) y \(y_2'(x_2)\) son todos distintos de cero. Dado que \(y_2'\) es continua en \(I\), \(x_1\) tiene una vecindad abierta \(U_1\) donde el signo de \(y_2'\) no cambia, y de manera similar existe un abierto \(U_2\) de \(x_2\) donde \(y_2'\) no cambia de signo. Sin embargo, los signos de \(y_2'\) en \(U_1\cap I\) y \(U_2\cap I\) no pueden ser iguales, ya que si \(y_2\) es creciente en uno de los intervalos, entonces debe ser decreciente en el otro. Para que \(W(x)\) tenga un signo constante en \(I\), \(y_1(x_1)\) y \(y_1(x_2)\) deben tener signos opuestos; por lo tanto, \(y_1\), siendo continua, tiene al menos un cero entre \(x_1\) y \(x_2\). No puede haber más de un cero, ya que si \(x_3\) y \(x_4\) son dos ceros de \(y_1\) que se encuentran entre \(x_1\) y \(x_2\), podemos usar el mismo argumento para concluir que \(y_2\) se anula entre \(x_3\) y \(x_4\). Pero esto contradice la suposición de que \(x_1\) y \(x_2\) son ceros consecutivos de \(y_2\).\QED\\
\begin{cor}
Si dos soluciones de \(y''+p(x)y'+q(x)y=0\) tienen un cero común en \(I\), entonces son linealmente dependientes.
\end{cor}
Para estudiar la distribución de ceros de la ecuación \ref{eq-homogenea}, sería mucho más conveniente si pudiéramos eliminar el término medio \(qy'\) transformando la ecuación a
\[
u''+\rho(x)u=0.
\]
Para ello lo que se hace es tomar
\[
y(x)=u(x)v(x),
\]
de modo que
\begin{align*}
y'(x) &= u'(x) v(x) + u(x) v'(x),\\
y''(x) &= u''(x) v(x) + 2 u'(x) v'(x) + u(x) v''(x).
\end{align*}
Sustituyendo en la ecuación obtenemos
\[
v(x)u''(x)+ (2v'(x)+p(x)v(x))u'(x) + (v''(x)+p(x)v'(x)+r(x)v(x))u(x)=0.
\]
Así, eligiendo \(2v'(x)+p(x)v(x)=0\), tenemos que
\[
v(x)=\exp\left(-\frac{1}{2}\int_a^xp(t)\,dt\right),
\]
y
\[
\rho(x)=q(x)-\frac{1}{4}p^2(x)-\frac{1}{2}p'(x).
\]
La función exponencial \(v\) nunca se anula en \(\re\), por lo que los ceros de \(u\) coinciden con los de \(y\).
\begin{teorema}[Teorema de Comparación de Sturm]\label{strum-comp}
Sean \(\phi\) y \(\psi\) soluciones no triviales de las ecuaciones
\[
y''+r_1(x)y=0,\quad u''(x)+r_2(x)u(x)=0,\quad x\in I,
\]
respectivamente, y supongamos que \(r_1(x)\geq r_2(x)\) para todo \(x\in I\). Entonces, \(\phi\) tiene al menos un cero entre cada dos ceros consecutivos de \(\psi\), a menos que \(r_1(x)\equiv r_2(x)\) y \(\phi\) sea un múltiplo constante de \(\psi\).
\end{teorema}
\dem Sean \(x_1\) y \(x_2\) dos ceros consecutivos de \(\psi\) en \(I\), y supongamos que \(\phi\) no tiene ceros en el intervalo abierto \((x_1,x_2)\). Asumamos que tanto \(\phi\) como \(\psi\) son positivas en \((x_1, x_2)\); de lo contrario, cambiamos el signo de la función negativa. Dado que \(\phi'\) y \(\psi'\) son continuas, se sigue que \(\psi'(x_1)\geq0\) y \(\psi'(x_2)\leq0\), y por lo tanto, el Wronskiano de \(\phi\) y \(\psi\) satisface
\[
W(x_1)=\phi(x_1)\psi'(x_1)\geq0,\quad W(x_2)=\phi(x_2)\psi'(x_2)\leq 0.
\]
Pero como
\[
W'(x)=\phi(x)\psi''(x)-\phi''(x)\psi(x)=[r_1(x)-r_2(x)]\phi(x)\psi(x)\geq0\quad\forall x\in(x_1,x_2),
\]
\(W\) es una función creciente en \((x_1,x_2)\). Esto contradice las desigualdades anteriores a menos que \(r_1(x)-r_2(x)\equiv0\) y \(W(x)\equiv0\), en cuyo caso \(\phi\) y \(\psi\) son linealmente dependientes.
\QED\\
\begin{cor}
Sea \(\phi\) una solución no trivial de \(y''+r(x)y=0\) en \(I\). Si \(r(x)\leq0\), entonces \(\phi\) tiene a lo sumo un cero en \(I\).
\end{cor}
\dem Si la solución \(\phi\) tiene dos ceros en \(I\), digamos \(x_1\) y \(x_2\), entonces, por el Teorema \ref{strum-comp}, la solución \(\psi(x)\equiv1\) de \(u''= 0\) debe anularse en \((x_1,x_2)\), lo cual es imposible.
\begin{def.}
  Si una soluci'on de \ref{eq-homogenea}, tiene infinitos ceros, entonces diremos que las soluciones son \emph{ondulatorias}.
\end{def.}
\eje Veremos que la funciones de Bessel son \emph{ondulatorias}. Las funciones de Bessel son las soluciones a las ecuaciones
\begin{equation}\label{eq-bessel}
  y''+\frac{1}{x}y'+\left(1-\frac{\nu^{2}}{x^{2}}\right)y=0,\quad x\in\re^{+}
\end{equation}
conocidas como las \emph{ecuaciones de Bessel de orden\(\nu\).} Multiplicando por \(v(x)=e^{1/2\log(x)}=\sqrt{x}\), estas ecuaciones se transforman en
\[
  u''+\left(1+\frac{1-4\nu^{2}}{4x^{2}}\right)u=0.
\]
Observamos que
\[
  q(x)=1+\frac{1-4\nu^{2}}{4x^{2}}\implies\begin{cases}
                                          q(x)\geq1\text{ si } \nu\in[0,\frac{1}{2}],\\
                                          q(x)<1\text{ si } \nu>\frac{1}{2}.\\
                                         \end{cases}
\]
Ahora podemos comparar con \(u''+u=0\) para obtener que para cualquier soluci'on a la ecuaci'on de Bessel se cumple alguna de las siguientes opciones
\begin{enumerate}
\item[i)] Si \(\nu\in[0,1/2]\), entonces todos los subintervalos de \(\re^{+}\) de longitud \(\pi\) tienen \textbf{por lo menos} un cero.
\item[ii)] Si \(\nu>1/2\), entonces todos los intervalos de longitud \(pi\) de \(\re^{+}\) tienen \textbf{al menos} un cero.
\item[iii)]Si\(\nu=1/2\), entonces la distancia entre ceros es \textbf{exactemente} \(pi\).

\end{enumerate}
\section{Operadores adjuntos de Hilbert}
\noindent Ahora pensaremos a las ecuaciones previamente mencionadas como \emph{operadores} en \(\mathcal{L}^{2}(I)\), entonces hablaremos un poco de \emph{operadores lineales en espacios de Hilbert} de la forma $T:V\subset\mathcal{H}\to\mathcal{H}$. Recordemos que el dominio e imagen se definen como
\begin{itemize}
    \item El \textbf{dominio} de $T$, denotado $\mathcal{D}(T)$, es el subespacio de $\mathcal{H}$ donde $T$ está definido.
    \item La \textbf{imagen} de $T$, denotado $\mathcal{R}(T)$, es el conjunto $Im(T)=\{T(x) : x \in \mathcal{D}(T)\}$.
\end{itemize}
En los epacios de Hilbert es posible definir a partir de un operador lineal y continuo, \(T\) un operador especial \(T^{*}\), el cual es llamado el \emph{operador adjunto de Hilbert de \(T\)}. Vale la pena mencionar que la noci'on de \emph{operador adjunto} es genral y no necesariamente exclusiva a los espacios de Hilbert o espacios con producto interno, sin embargo la retpresentaci'on que usaremos utilizando el producto interno, s'i es unica en esoacios de Hilbert. Podemos definir y representar al operador adjunto gracias al \textbf{teorema de representaci'on de Riez} (teorema \ref{riez}), ya que para \(T\) lineal, entonces para todo \(\bra{\psi}\in\mathcal{H}^{*},\,\ket{\phi}\mapsto\braket{\psi|T(\phi)}\) tambi'en es lineal, entonces por representaci'on de Riez exite un \emph{bra que depende de} \(\psi\), al cual denotamos por \(T^{*}\bra{\psi}\in\mathcal{H}^{*}\) tal que
\[
  \braket{T^{*}(\psi)|\phi}=\braket{\psi| T(\phi)}, \quad \forall \ket{\phi} \in \mathcal{D}(T), \, \bra{\psi} \in \mathcal{D}(T^*).
\]
\noindent Adem'as como \(\braket{\cdot|\cdot}\) y \(T\) son lineales, entonces \(T^{*}:D(T^{*})\to\mathcal{H}^{*}\) es un operador lineal y si \(T\) es \emph{acotado}, entonces \(T^{*}\) tambi'en lo es.

\begin{def.}
Dado un operador $T$ en $\mathcal{H}$, su \textbf{adjunto} $T^*$ es el operador que satisface
\[
\braket{T^{*}(\psi)|\phi}=\braket{\psi| T(\phi)}, \quad \forall \ket{\phi} \in \mathcal{D}(T), \, \bra{\psi} \in \mathcal{D}(T^*).
\]
\end{def.}
\obs Si $\braket{\phi|\psi}=0$ para toda $\psi$, entonces $\bra{\phi}=0$ pues $\|\phi\|^{2}=\braket{\phi|\phi}=0$, por esto mismo si $\braket{\phi|T\psi}=0$ para toda $\bra{\phi}$, entonces $T=0$, o equivalentemente
\[
  \braket{\phi|T\psi}=\braket{\phi|S\psi},\,\forall\ket{\phi},\ket{\psi}\in\mathcal{H}\implies\,S=T.
\]
\noindent\textbf{Propiedades del Adjunto}
\begin{enumerate}
  \item $(T + S)^* = T^* + S^*$ si $\mathcal{D}(T) \cap \mathcal{D}(S)$ es denso.
  \item $(TS)^* = S^* T^*$ si $\mathcal{D}(TS)$ es denso.
  \item $(T^*)^* = T$ (si $T$ es cerrado y densamente definido).
  \item $(\alpha T)^* = \overline{\alpha} T^*$ para $\alpha \in \co$.
  \item $\|T\|=\|T^{*}\|$, o equivalentemente $\|T\|^{2}=\|T^{*}\|\|T\|=\|T\|\|T^{*}\|$.
  \item Si $\lambda$ es valor propio de $T$, entonces $\overline{\lambda}$ es valor propio de $T^{*}$.
\end{enumerate}
\dem Sean $\ket{\phi}$ y $\ket{\psi}$ arbitarios, para demostrar $1.$, simplemente calculamos
\begin{align*}
  \braket{(S+T)^{*}\phi|\psi}&=\braket{\phi|(S+T)\psi}=\braket{\phi|S\psi+T\psi}\\
  &=\braket{\phi|T\psi}+\braket{\phi|S\psi}=\braket{T^{*}\phi|\psi}+\braket{S^{*}\phi|\psi}=\braket{(S^{*}+T^{*})\phi|\psi}.
\end{align*}
Por lo tanto $(S+T)^{*}=S^{*}+T^{*}$, para $2.$
\[
  \braket{\phi|TS\psi}=\braket{T^{*}\phi|S\psi}=\braket{S^{*}T^{*}\phi|\psi}.
\]
$3.$ Sea $(T^{*})^{*}=T^{**}$, entonces comprobamos que para todo $\ket{\phi},\ket{\psi}\in\mathcal{D}(T)$, se cumple
\[
  \braket{T^{**}\phi|\psi}=\braket{\phi|T^{*}\psi}=\braket{T\phi|\psi}.
  \]
Por lo tanto $T=T^{**}$, si $T$ es cerrado y densamente definida, es decir $\mathcal{D}(T)=\mathcal{H}$. Para $4.$,
\[
  \braket{\phi|(\alpha T)\psi}=\braket{\phi|\alpha(T\psi)}=\braket{\overline{\alpha}\phi|T\psi}=\braket{T^{*}(\overline{\alpha}\phi)|\psi}=\braket{\overline{\alpha}T^{*}\phi|\psi}.
\]
Por lo tanto $(\alpha T)^{\alpha}=\overline{\alpha}T^{*}$. Para $5.$, por los incisos $2.$ y $3.$, $(T^{*}T)^{*}=T^{*}T$ simplemente calculamos para toda $\ket{\phi}$
\[
  \|T\phi\|^{2}=\braket{T\phi|T\phi}=\braket{T^{*}T\phi|\phi}\leq\|T^{*}\|T\|\|\phi\|^{2}.
\]
Lo que implica que $\|T\|^{2}\leq\|T^{*}\|\|T\|$, haciendo lo mismo con $T^{*}$ y recordando que $T^{**}=T$ se obtiene la otra desigualdad, notamos que esto implica que $T^{*}T=0$, entonces $T=0$ pues $\|T\|$ es norma.
Por 'ultimo, sea $\lambda$ valor propio de $T$ y $\ket{\phi}$ es tal que $T\phi=\lambda\ket{\phi}$, entonces sea $\ket{\psi}$
\[
  \braket{\overline{\lambda}\psi|\phi}=\braket{\psi|\lambda\phi}=\braket{\phi|T\phi}=\braket{T^{*}\psi|\phi}
\]
Como $\ket{\psi}$ es arbitraria, entonces tiene que existir una tal que $T^{*}\psi=\overline{\lambda}\ket{\psi}$.
\begin{def.}{Operadores Autoadjuntos}
Un operador $T$ es \textbf{autoadjunto} si $T = T^*$, es decir
\[
  \braket{T(\psi)|\phi}=\braket{\psi| T(\phi)}, \quad \forall \ket{\phi} \in \mathcal{D}(T), \, \bra{\psi} \in \mathcal{D}(T^*).
\]
Los operadores autoadjuntos son fundamentales en mecánica cuántica, ya que representan observables físicos.
\end{def.}
\eje Para todo operador $T$, como vimos anteriormente $S=T^{*}T$ es autoadjunto pues
\[
  S^{*}=(T^{*}T)^{*}=T^{*}T^{**}=T^{*}T=S.
\]
\eje\textbf{Matrices Hermitianas}
En $\mathcal{H} = \con$, un operador $T$ puede representarse como una matriz $A\in\co^{n \times n}$. $A$ es \textbf{Hermitiana} (autoadjunta) si:
\[
A = A^*, \quad \text{donde } A^* = \overline{A}^T.
\]
Por ejemplo, la matriz
\[
A = \begin{pmatrix}
2 & i \\
-i & 3
\end{pmatrix}
\]
es Hermitiana, ya que $A^* = A$.

Algunas propiedades de los operadores autoadjuntos
\begin{itemize}
  \item Sus valores poropios son reales ya que si \(\ket{\phi}\) es vector propio de norma uno de $T$ autoadjunto con valor propio \(\lambda\), entonces
        \begin{align*}
        \lambda=\lambda\|\phi\|^{2}=\lambda\braket{\phi|\phi}=\braket{\phi|\lambda\phi}=\braket{\phi|T\phi}=\braket{T\phi|\phi}=\overline{\lambda}
        \end{align*}
  \item Sus vectores propios de valores distintos son ortogonales, es decir, si $\ket{\phi}$ tiene valor propio $\lambda$ y $\ket{\psi}$ tiene valor propio \emph{distinto} $\mu$, entonces
        \begin{align*}
        \lambda\braket{\psi|\phi}=\braket{\psi|\lambda\phi}=\braket{\psi|T\phi}=\braket{T\psi|\phi}=\braket{\mu\psi|\phi}=\mu\braket{\psi|\phi}
        \end{align*}
        Recordamos que $\mu$ es real por el punto anterior, entoces si $\braket{\psi|\psi}$ no es cero $\mu=\lambda$ lo cual es imposible.

\end{itemize}
\eje\textbf{Operadores Unitarios}
Un operador $U$ es \textbf{unitario} si
\[
U^* U = U U^* = I,
\]
donde $I$ es el operador identidad. Los operadores unitarios preservan el producto interno
\[
\braket{ U(x)| U(y)} = \braket{ x | y}, \quad \forall x, y \in \mathcal{H}.
\]

\eje\textbf{Matrices Unitarias}
En $\mathbb{C}^n$, una matriz $U \in \mathbb{C}^{n \times n}$ es unitaria si
\[
U^* U = U U^* = I.
\]
Por ejemplo, la matriz
\[
U = \frac{1}{\sqrt{2}} \begin{pmatrix}
1 & -i \\
1 & i
\end{pmatrix}
\]
es unitaria, ya que $U^* U = I$.
\begin{teorema}[Teorema Espectral]
Sea \(T\) un operador autoadjunto (matriz hermitiana) en $\con$, entonces existe una base ortonormal de \(\con\) de vectores propios tal que la matriz de cambio de base \(U\) es unitaria y
\[
UTU^{*}= \begin{pmatrix}\lambda_{1}& 0 & \dots & 0\\ 0 & \lambda_{2} & \dots & 0\\ \vdots & \vdots & \cdots & \vdots\\ 0 & 0 & \dots & \lambda_{n} \end{pmatrix}
\]
\end{teorema}
\obs Notamos que si $T$ es autoadjunto l'imite de operadores con rango de dimensi'on finita, entonces es v'alido  el teorema espectral para $T$ si lo hacemos en cada uno de sus componentes de dimension finita.
\subsection{Operadores de Sturm-Liouville autoadjuntos}
\noindent Ahora veremos cu'al es el adjunto de un operador diferencial ordinario de orden dos
\[
L=a_0(x)\frac{d^2}{dx^2}+a_1(x)\frac{d}{dx}+a_2(x)I,\quad x\in [a,b]\subset\re,\{a_{0},a_{1},a_{2}\}\subset\mathcal{C}^{2}[a,b]
\]
El cual se define como un operador
\[
  L:\mathcal{L}^{2}([a,b],\mu)\cap\mathcal{C}^{2}[a,b]\to\mathcal{L}^{2}([a,b],\mu).
\]
\noindent Donde $\mu$ va a ser la \emph{medida de Lebesgue} $\lambda$ o una medida derivada de la misma. Ahora usamos integraci'on por partes para calcular $L^{*}$
\begin{align*}
  \braket{\phi|L\psi}&=\int_{a}^{b}\overline{\phi}(a_{0}(x)\psi''+a_{1}(x)\psi'+a_{2}(x)\psi)\,d\lambda(x)\\
                     &=\int_{a}^{b}a_{0}\overline{\phi}\psi''+a_{1}\overline{\phi}\psi'+a_{2}\overline{\phi}\psi\,d\lambda(x)\\
                     &=a_{0}\overline{\phi}\psi'|_{a}^{b}-\int_{a}^{b}(a_{0}(x)\overline{\phi})'\psi'\,d\lambda(x)\\
                     &+a_{1}\overline{\phi}\psi|_{a}^{b}-\int_{a}^{b}(a_{1}(x)\overline{\phi})'\psi\,d\lambda(x)
                     +\int_{a}^{b}\overline{\phi}a_{2}\psi\,d\lambda(x)\\
                     &=(a_{0}\overline{\phi}\psi'-\psi(a_{0}\overline{\phi})')|_{a}^{b} +\int_{a}^{b}(a_{0}(x)\overline{\phi})''\psi\,d\lambda(x)\\
                     &a_{1}\overline{\phi}\psi|_{a}^{b}-\int_{a}^{b}(a_{1}\overline{\phi})'\psi\,d\lambda
                       +\int_{a}^{b}\overline{\phi}a_{2}\psi\,d\lambda(x)\\
                     &=\braket{(\overline{a_{0}}\phi)''-(\overline{a_{1}}\phi)'+\overline{a_{0}}\phi|\psi}+
                       [a_{0}(\overline{\phi}\psi'-\psi\overline{\phi}')+(a_{1}-a_{0}')\overline{\phi}\psi]|_{a}^{b}
\end{align*}
Esto es indicativo de la definici'on de $L^{*}$, es decir definimos $L^{*}$ el \emph{adjunto formal} como
\begin{align*}
  L^{*}\phi(x)&=(\overline{a_{0}}(x)\phi)''-(\overline{a}_{1}(x)\phi)'+\overline{a_{2}}(x)\phi\\
             &=\overline{a_{0}(x)}\phi''+(2\overline{a_{0}(x)}'-\overline{a_{1}(x)})\phi'+(\overline{a_{0}(x)}''-\overline{a_{1}(x)}'+\overline{a_{2}(x)}\phi
\end{align*}
Por lo tanto,
\[
  L^{*}=\overline{a_0(x)}\frac{d^2}{dx^2}+(2\overline{a_{0}(x)}'-\overline{a_1(x)})\frac{d}{dx}+(\overline{a_{0}(x)}''-\overline{a_{1}(x)}'+\overline{a_2(x)})I.
\]
Entonces para que $L$ sea \emph{formalmente autoadjunto,} es necesario que
\[
  \overline{a_{0}}=a_{0},\quad 2\overline{a_{0}}'-\overline{a_{1}}=a_{1},\quad\overline{a_{0}}''-\overline{a_{1}}'+\overline{a_{2}}=a_{2},
\]
es decir, $a_{0}$, $a_{1}$ y $a_{2}$ son funciones \emph{reales} y $a_{1}=a_{0}'$, en cuyo caso, escribimos $a_{0}=p$ y $a_{2}=q$ y por lo tanto el operador se escribe de la forma
\[
    L=\frac{d}{dx} \left( p(x) \dfrac{d}{dx} \right) + q(x)I,
\]
los cuales ya hab'iamos definido como \emph{operadores de Sturm-Liouville}. Ahora para que $L$ sea \emph{auto adjunto}, es necesario que se cumpla
\[
  p(\phi'\overline{\psi}-\phi\overline{\psi'})=0\text{ para todas }\phi,\psi.
\]
Ahora, nosotros estaremos interesados en el problema de valores propios de $-L$, es decir en resolver ecuaciones del tipo
\[
  Lu+\lambda u=0.
\]
La raz'on por la cual nos interesan los valores propios de $L$ y no de $-L$ es porque los valores propios de $L$ resultan negativos cuando $p>0$, como el caso de $y''+ky=0$.

\eje El operador $L=-(\dfrac{d^{2}}{dx^{2}})$ es formalmente autoadjunto ($p=-1$ y $q=0$), recordamos que sus \emph{eigenfunciones o funciones propias} son las soluciones a la ecuaci'on
\[
  u''+\lambda u=0.
\]
M'as a'un si imponemos condiciones de forntera $u(0)=u(\pi)=0$ entonces es f'acil ver que necesariamente $\lambda>0$ y como se vio previamente los vectores propios normalizados son los monomios trigonom'etricos
\[
  \sqrt{\frac{2}{\pi}}\sin(nx)\quad n\in\nat.
\]
Los cuales sabemos que son \emph{ortonormales}.

\noindent Ahora, \textbf{¿Qu'e se puede hacer cuando $a_{0}'\neq a_{1}$ en $L$?}. Resulta que podemos multiplicar a $L$ por una funci'on positiva $\rho(x)>0$ de tal forma que $L$ es \emph{formalmente autoadjunto} para un \emph{producto punto} modificado o \emph{medida modificada}. Primero, sea
\[
L=a_0(x)\frac{d^2}{dx^2}+a_1(x)\frac{d}{dx}+a_2(x)I,\quad x\in [a,b]\subset\re,
\]
\noindent donde $a_{0}\in\mathcal{C}^{2}(a,b)$ $a_{1}\in\mathcal{C}^{1}(a,b)$ y $a_{2}\in\mathcal{C}[a,b]$, entonces si consideramos el operador $\tilde{L}=\rho L$, sabemos por lo anteriormente discutido que $\tilde{L}$ es formalmente autoadjunto si y s'olo si
\[
  \rho a_{1}=(\rho a_{0})'=\rho'a_{0}+\rho a_{0}',
\]
esta es una ecuaci'on diferencial de orden uno cuya soluci'on claramente es
\[
  \rho(x)=\frac{c}{a_{0}(x)}\exp\left(\int_{a}^{x}\frac{a_{1}(t)}{a_{0}(t)}\,dt\right).
\]
Escogiento una $c$ adecuada podemos suponer que $\rho(x)>0$. Por lo tanto simempre podemos multiplicar a $L$ por una funcion positiva tal que el resultado sea formalmene autoadjunto sin embargo, esto modifica el problema original ya que al umtiplicar por $\rho$, en realidad estamos resolviendo un problema de valores propio de tipo
\[
  \rho(x)Lu(x)+\rho(x)\lambda u(x)=0.
\]
\noindent Donde la condici'on para que $L$ sea autoadjunto ahora se transforma en
\[
  \rho p(\phi'\overline{\psi}-\phi\overline{\psi'})|_{a}^{b}=0\text{ para todas }\phi,\psi.
\]
Esto sugiere \emph{modificar el producti interno o  la medida} para que se siga cumpliendo la condici'on de operador auto adjunto, es decir se toma el producto interno
\begin{equation} \braket{\phi|\psi}_{\rho}=\int_{a}^{b}\overline{\phi(x)}\psi(x)\rho(x)\,d\lambda(x)=\int_{a}^{b}\overline{\phi(x)}\psi(x)\,d\mu_{\rho}(x).
\end{equation}
\noindent Donde la medida modificada es
\[
  \mu_{\rho}(E):=\int_{E}\rho(x)\,d\lambda(x)\quad E\in\mathbb{B}(\re).
\]
\obs Suponinendo que $\rho(x)$ es continua en $x\in[a,b]$ intervalo compacto, entonces existen $\{c_{1},c_{2}\}\subset\re^{+}$ tales que $0<c_{1}<\rho(x)<c_{2}<\infty$ para toda $x\in[a,b]$, esto implica que
\begin{equation}\label{norm-equiv}
  \sqrt{c_{1}}\|u\|_{2}\leq\|u\|_{\rho}\leq\sqrt{c_{2}}\|u\|_{2}.
\end{equation}
Es decir
\[
  \int_{a}^{b}|u(x)|^{2}\,d\lambda(x)<\infty\iff \int_{a}^{b}|u(x)|^{2}\,d\mu_{\rho}(x)<\infty,
\]
es decir, los espacios $\mathcal{L}^{2}([a,b],\lambda)$ y $\mathcal{L}^{2}([a,b],\mu_{\rho})$ contienen a las mismas funciones y m'as a'un, las desigualdades \ref{norm-equiv} en el leguange de espacios normados se le conoce como \emph{normas equivalentes} y esto significa que ambos espacios tienen \emph{la misma topolog'ia}, es decir las nociones de convergencia son equivalentes, sin embargo el producto punto es \textbf{distinto.}

\subsection{El oscilador arm'onico cu'antico y los polinomonios de Herminte}
\noindent Ahora veremos uno de los ejemplos m'as importantes de un problema de Strum-Liouville, el operador hamiltioniano para un potencial independiente del tiempo, es decir la ecuaci'on
\begin{equation}\label{hamilton}
    H\varphi=-\frac{\hbar^2}{2m} \frac{d^2 \varphi(x)}{dx^2} + V(x) \varphi(x) = E \varphi(x).
\end{equation}
por las discusiones anteriores, sabemos que es \emph{formalmente autadjunto}, entonces $E\in\re$, adem'as como $\hbar^{2}/2m>0$, entonces $E\in\re^{=}$. Ahora como ejemplo espec'ifico, consideramos el potencial $V(x) = \frac{1}{2}m\omega^{2}x^2$,entonces la ecuación queda como
\begin{equation}
\left[-\frac{\hbar^2}{2m}\frac{d^2}{dx^2} + \frac{1}{2}m\omega^2x^2\right]\phi(x) = E\phi(x)
\end{equation}
Para resoverla, realizamos el cambio de variable
\[
  \xi = \sqrt{\frac{m\omega}{\hbar}}x,
\]
\noindent adem'as tomamos $\lambda = 2E/\hbar\omega$ y la ecuación se transforma en

\begin{equation}\label{oscil2}
\frac{d^2\psi}{d\xi^2} + (\lambda - \xi^2)\psi = 0
\end{equation}
Para resolver la ecuaci'on hacemos el an'alsis asint'otico $\xi \to \pm\infty$, entonces dominan los términos cuadráticos
\begin{equation}
\frac{d^2\psi}{d\xi^2} \approx \xi^2\psi \implies \psi(\xi) \sim e^{-\xi^2/2}
\end{equation}
as'i proponemos
\begin{equation}\label{oscil-herm}
\psi(\xi) = H(\xi)e^{-\xi^2/2}.
\end{equation}
Resulta que $H(\xi)$ son \emph{los polinomios de Hermite} ya que cumplen con la famosa \textbf{ecuación fiferencial de Hermite}, esto es ya que al sustituir \ref{oscil-herm} en la ecuaci'on \ref{oscil2}, se obtiene
\begin{align*}
  \dfrac{d\psi}{d\xi}&=\dfrac{dH}{d\xi}e^{-\xi^{2}/2}-\xi H(\xi)e^{-\xi/2}
                       =\left(\dfrac{dH}{d\xi}-\xi H(\xi)\right)e^{-\xi^{2}/2}\\
  \dfrac{d^{2}\psi}{d\xi^{2}}&=\left(\dfrac{d^{2}H}{d\xi^{2}}-H(\xi)-\xi\dfrac{dH}{d\xi}\right)e^{\xi^{2}/2}+
                               \xi\left(\dfrac{dH}{d\xi}-\xi H(\xi)\right)e^{-\xi^{2}/2}\\
                     &=\left(\dfrac{d^{2}H}{d\xi^{2}}-2\xi\dfrac{dH}{d\xi}+(\xi^{2}-1)H(\xi)\right)e^{\xi^{2}/2}\\
  &\implies\frac{d^2\psi}{d\xi^2}+(\lambda-\xi^2)\psi=\left(H''(\xi)-2\xi H'(\xi)+(\lambda-1)H(\xi)\right)e^{\xi^{2}/2}\\
  &\implies H''(\xi) - 2\xi H'(\xi) + (\lambda - 1)H(\xi) = 0.
\end{align*}
Ahora para encontrar las soluciones de esta ecuaci'on diferencial lineal y ordianaria de orden dos, se utiliza el \emph{m'etodo de Frobenius}, veremos este m'etodo con mayor profundidad en el siguiente cap'itulo, pero como ejemplo de este m'etodo, esta ecuaci'on es muy ilustrativa.

\section*{Método de Frobenius}
\noindent Por el teorema de \emph{Cauchy-kovalevskaya}, podemos asegurar que las soluciones de una ecuaci'on diferencial lineal con coeficientes anal'iticos, son anal'iticas. Por lo tanto, podemos buscat soluciones en serie de potencias alrededor de \(\xi = 0\)
\begin{equation}
H(\xi) = \sum_{k=0}^\infty a_k \xi^k,
\end{equation}
entonces calculamos las derivadas
\begin{align*}
H'(\xi) &= \sum_{k=1}^\infty k a_k \xi^{k-1} \\
H''(\xi) &= \sum_{k=2}^\infty k(k-1) a_k \xi^{k-2}
\end{align*}
Sustituyendo en la ecuaci'on de Hermite, obtenemos
\begin{equation}
\sum_{k=2}^\infty k(k-1)a_k\xi^{k-2}-2\xi\sum_{k=1}^\infty k a_k \xi^{k-1} + (\lambda - 1) \sum_{k=0}^\infty a_k \xi^k = 0
\end{equation}
Igualando coeficientes para cada potencia de \(\xi\)
\[
\sum_{k=0}^\infty \left[(k+2)(k+1)a_{k+2} + (-2k + \lambda - 1)a_k\right] \xi^k = 0
\]
Esto determina una relación de recurrencia en los coeficientes de la serie ya que para que la serie sea solución \(\forall \xi\), ya que se debe cumplir
\begin{equation}
a_{k+2} = \frac{2k - (\lambda - 1)}{(k+2)(k+1)} a_k
\end{equation}
Aplicando la recursi'on hasta los primeros coeficientes implica que todos los coeficientes depenenden de $a_{0}$ si son pares y $a_{1}$ si son inpares, y podemos escribir a $H(\xi)$ como
\[
  H(\xi)=a_{0}H_{par}(\xi)+a_{1}H_{inpar}(\xi).
\]
Ahora, para obtener soluciones físicamente aceptables es necesario que las soluciones sean \emph{polinomios}, es decir, exigimos que la serie corte en algún \(k = n\). Esto se debe a que para tener soluciones f'isicamente acjpetables, es necesario que las soluciones sean \emph{integrabes}, ya que es necesario obterner una  \emph{distribuci'on de probabilidad} que describa el comportamiento de la onda cu'antica. Cuando las soluciones no son polinomios, \textbf{no} son inegrables ya que 

\[
  a_{j+2}\approx\frac{2}{j}a_{j}\text{ y asi } a_{j}\approx \frac{C}{(j/2)!}\quad C\in\re^{+},
\]
entonces si nos fijamos solamente en los coeficientes pares obtenemnos que la parte par de $H$ cumple
\[
  H_{par}(\xi)\approx C\sum_{j+0}^{\infty}=Ce^{\xi^{2}}
\]
y esto har'ia que la soluci'on no fuera integrable, se puede hacer algo similar para los coeficientes impares, por lo tanto es necesario exigir que $H(\xi)$ sea un polinomio.

La conocida \emph{soluci'on algebraica} a este problema utiliza los famosos \emph{operadores de creaci'on y aniquilaci'on}, esta forma de solucionar el problema hace m'as evidente las siguientes consecuencias de cuatizaci'on:
\[
2n - (\lambda - 1) = 0 \implies \lambda = 2n + 1
\]
Lo cual reproduce la cuantización de energía
\[
E_n = \hbar\omega\left(n + \frac{1}{2}\right)
\]
As'i las soluciones son
\begin{itemize}
\item \textbf{Solución par}: \(a_1 = 0\), \(a_0 \neq 0\)
\[
H_{n=2k}(\xi) = \sum_{m=0}^{\lfloor n/2 \rfloor} (-1)^m \frac{n!}{(n-2m)!m!} (2\xi)^{n-2m}
\]

\item \textbf{Solución impar}: \(a_0 = 0\), \(a_1 \neq 0\)
\[
H_{n=2k+1}(\xi) = \sum_{m=0}^{\lfloor (n-1)/2 \rfloor} (-1)^m \frac{n!}{(n-2m)!m!} (2\xi)^{n-2m}
\]
\end{itemize}
As'i, se toman alternadamente $a_{1}=0$ o $a_{0}=1$ para cada $n\in\nat$ seg'un su paridad, con esto se obtiene la familia completa de soluciones $\{H_{n}(\xi)\}_{n\in\nat}$ que son \emph{ortogonales} cuando se considera el producto punto modificado por $e^{\xi^{2}/2}$, los primeros polinomios son
\begin{enumerate}
  \item $H_{0}=1$
  \item $H_{1}=2\xi$
  \item $H_{2}=4\xi^{2}-2$
  \item $H_{3}=8\xi^{3}-12\xi$
  \item $H_{4}=16\xi-48\xi^2=12$
\end{enumerate}
\noindent\textbf{La funci'on generadora}
Muchas familias de polinomios tienen una \emph{funcion generadora}, en general una funci'on generadora es una serie \emph{formal} cuyos coeficientes son la familia de polinomios en cuestion, en el caso de los polinomios de Hermite, la funci'on generadora proviene de la expansi'on en serie de Taylor de la funci'on
\[
  \mathcal{G}(\xi,z)=e^{2z\xi-\xi^{2}}=\sum_{n=0}^{\infty}\frac{H_{n}(\xi)z^{n}}{n!}
\]
Por lo tanto se puede calcular los polinomios de Hermite como
\begin{equation}
H_n(\xi) = (-1)^n e^{\xi^2}\frac{d^n}{d\xi^n}e^{-\xi^2}
\end{equation}
entonces, al derivar esta serie con respecto a $z$ y con respecto a $\xi$ y al igualar las series resultantes (por unicidad de sus coeficientes) se obtienen las relaciones de recurrencia
\begin{align}
H_{n+1}(\xi) &= 2\xi H_n(\xi) - 2n H_{n-1}(\xi) \\
\frac{d}{d\xi}H_n(\xi) &= 2n H_{n-1}(\xi).
\end{align}
De las cuales se obtiene la ecuaci'on diferncial de Hermite
\[
  H_{n}''(\xi) - 2\xi H_{n}'(\xi) + 2nH_{n}(\xi) = 0.
\]

\exe  Derive la funci'on generadora con respecto a $z$ y $\xi$ para obtener las relaciones de recurrencia, demuestre que los polinomios de Hermite como fueron definidos anteriormente cumplen con las relaciones de recurrencia y encuentre la ecuaci'on diferencial de los polinomios de Hermite a partir de las relaciones de recurrencia.

Por 'ultimo las las soluciones completas normalizadas al oscilador arm'onico son
\begin{equation}
\psi_n(x) = \left(\frac{m\omega}{\pi\hbar}\right)^{1/4}\frac{1}{\sqrt{2^n n!}}H_n\left(\sqrt{\frac{m\omega}{\hbar}}x\right)e^{-m\omega x^2/2\hbar}
\end{equation}

De donde es importante resaltar las primeras soluciones ya que corresponden con los \emph{niveles m'as bajos de energ'ia} posible

\begin{itemize}
\item Estado fundamental ($n=0$):
\begin{equation}
\psi_0(x) = \left(\frac{m\omega}{\pi\hbar}\right)^{1/4}e^{-m\omega x^2/2\hbar}, \quad E_0 = \frac{1}{2}\hbar\omega
\end{equation}

\item Primer excitado ($n=1$):
\begin{equation}
\psi_1(x) = \left(\frac{m\omega}{\pi\hbar}\right)^{1/4}\sqrt{2}\left(\sqrt{\frac{m\omega}{\hbar}}x\right)e^{-m\omega x^2/2\hbar}
\end{equation}
\end{itemize}

Por lo que hemos visto, la teor'ia de Sturm-Liouville establece formalmente que los polinomios Hermite satisfacen
\begin{equation}
\int_{-\infty}^\infty H_n(\xi)H_m(\xi)e^{-\xi^2}d\xi = \sqrt{\pi}2^n n! \delta_{nm}
\end{equation}
lo que garantiza la ortonormalidad de las soluciones $\psi_n(x)$.
En las siguientes secciones veremos que Las funciones propias $\{\psi_n\}_{n=0}^\infty$ forman una base completa en $L^2(\mathbb{R})$.

\section{Las funciones de Green y Existencia de funciones propias}
\noindent Como hemos visto, buscamos resolver el problema de vetores propios y valores propios para los operadores de Sturm-Liouville $-\mathcal{L}$, donde
\begin{equation}\label{SL-eq}
  \mathcal{L}=\dfrac{d}{dx}\left(p(x)\dfrac{d}{dx}\right)+q(x)I\quad x\in[a,b]
\end{equation}
es decir, buscamos resolver la ecuacion diferencial $\mathcal{L}\phi+\lambda\phi=0$ con las condiciones de frontera
\begin{eqnarray}\label{SL-fron}
  &\alpha_{1}\phi(a)+\alpha_{2}\phi'(a)=0\quad|\alpha_{1}|+|\alpha_{2}|>0\\
  &\beta_{1}\phi(b)+\beta_{2}\phi'(b)=0\quad|\beta_{1}|+|\beta_{2}|>0
\end{eqnarray}
Como hemos visto, esto hace que $\mathcal{L}$ sea autoadjunto y por lo tanto sus valores propios son reales y sus vectores propios de valores propios distintos sean ortogonales. Ahora lo que queremos es determinar a todovector como \emph{superposici'on} de vectores propios de $\mathcal{L}$, es decir, definimos los espacios propios
\[
  E_{\lambda}=\mathrm{Ker}(\mathcal{L}+\lambda I),\lambda\in\co
\]
asi definimos al \emph{espactro puntual} como los $\lambda\in\co$ tal que $E_{\lambda}\neq 0$ y denotamos al espectro de $\mathcal{L}$ como $\sigma(\mathcal{L})$, en nuetro caso $\sigma(\mathcal{L})\subset\re$, pero para operadores en general eso no es cierto. Entonces queremos que se cumpla la siguiente descomposici'on
\[
  \mathcal{L}^{2}([a,b],\mu)=\bigoplus_{\lambda\in\sigma(\mathcal{L})}E_{\lambda}.
\]
Para encontrar los vectores propios de nuestros operadores, usaremos las \emph{funciones de Green}, las cuales en general resulevel el problema $\mathcal{L}\phi=f$ con condiciones de frontera \ref{SL-fron} para $f$ continua.

\begin{def.}
Una funci'on de Green para el problema de Dirichlet (Sturm-Liouville) dado por
\begin{align*}
  &\mathcal{L}=\dfrac{d}{dx}\left(p(x)\dfrac{d}{dx}\right)+q(x)I\quad x\in[a,b]\\
  &\alpha_{1}\phi(a)+\alpha_{2}\phi'(a)=0\quad|\alpha_{1}|+|\alpha_{2}|>0\\
  &\beta_{1}\phi(b)+\beta_{2}\phi'(b)=0\quad|\beta_{1}|+|\beta_{2}|>0
\end{align*}
es una funci'on continua $G:[a,b]\times[a,b]\to\re$ que cumple con las siguientes propiedades
\begin{enumerate}
\item Simetr'ia: $G(x,t)=G(t,x)$, adem'as de cumplir con las condiciones de frontera \ref{SL-fron} en cada variable
\begin{align*}\label{Green-fron}
  &\alpha_{1}G(a,t)+\beta_{2}\dfrac{\partial G}{\partial x}(b,t)=0\quad\alpha_{1}G(x,a)+\beta_{2}\dfrac{\partial G}{\partial t}(x,b)=0\\
  &\alpha_{2}G(a,t)+\beta_{2}\dfrac{\partial G}{\partial x}(b,t)=0\quad\alpha_{2}G(x,a)+\beta_{2}\dfrac{\partial G}{\partial t}(x,b)=0
\end{align*}
  \item $G$ es diferenciable en $[a,b][a,b]\setminus\Delta$ donde $\Delta=\{x=t\}$, adem'as, en dicho conjunto $G$ cumple con la ecuaci'on differencial en una de sus variables, es decir, para cada $t$ se cumple
  \[
    \mathcal{L}[G(x,t)]=p(x)\dfrac{\partial^{2} G}{\partial x^{2}}(x,t)+p'(x) + \dfrac{\partial G}{\partial x}+q(x)G(x,t)=0
  \]
  \item $\dfrac{\partial G}{\partial x}$ tiene una discontinuidad en $\Delta$, donde se cumple
        \begin{equation}
          \dfrac{\partial G}{\partial x}(x^{+},x)-\dfrac{\partial G}{\partial x}(x^{-},x):=\lim_{\delta\to^{+}0}\left(\dfrac{\partial G}{\partial x}(x+\delta,x)-\dfrac{\partial G}{\partial x}(x-\delta,x)\right)=\frac{1}{p(x)}
          \end{equation}
\end{enumerate}
\end{def.}

\subsubsection*{Construcci'on de la funci'on de Green:}
\noindent Primero, notamos que podemos suponer que la 'unica solucion a la ecuaci'on $\mathcal{L}\phi=0,$ con las condiciones de frontera \ref{SL-fron} es $\phi=0$, es decir, suponemos que $0$ no es valor propio de $-\mathcal{L}$. Esto siempre es posible ya que si $\alpha\in\re$ no e valor propio de $-\mathcal{L}$, entonces el operador $\mathcal{L}_{\alpha}=\mathcal{L}+\alpha I$, entonces $\lambda$ es valor propio de $-\mathcal{L}$ si y s'olo si $\lambda-\alpha$ es valor propio de $-\mathcal{L_{\alpha}}$, como $\alpha$ no es valor propio de $-\mathcal{L}$, entonces $0$ no es valor propio de $-\mathcal{L}_{\alpha}$. Observamos que para que esto sea v'alido, es necesario que existan reales que no sean valores propios, eso lo veremos con mas detalle posteriormente.
Ahora para contruir $G$, por el teorema de existencia y unicidad, existen dos funciones linealmente independientes $\{v_{1},v_{2}\}$ de la ecuaci'on homogenea $\mathcal{L}\phi=0$ tales que
\begin{align*}
  &v_{1}(a)=\alpha_{2},\,v'_{1}(a)=-\alpha_{1}\quad\implies\quad\alpha_{1}v_{1}(a)+\alpha_{2}v'_{1}(a)=0\\
  &v_{2}(b)=\beta_{2},\,v'_{2}(b)=-\beta_{1}\quad\implies\quad\beta_{1}v_{2}(b)+\beta_{2}v'_{2}(b)=0
\end{align*}
Observamos que si $v_{1}=cv_{2}$ tendr'iamos dos soluciones a $\mathcal{L}\phi=0$ con las condiciones de frontera \ref{SL-fron}, as'i definimos
\begin{equation}
  G(x,t)=\begin{cases}
    v_{1}(t)v_{2}(x)[p(x)W(v_{1},v_{2})(x)]^{-1}\quad a\leq t\leq x\leq b\\
    v_{1}(x)v_{2}(t)[p(x)W(v_{1},v_{2})(x)]^{-1}\quad a\leq x\leq t\leq b.
  \end{cases}
\end{equation}
Entonces, como  $p(x)\neq0$ y $W(v_{1},v_{2})\neq0$, $G$ esta bien definida y es continua, m'as a'un la identidad de Lagrange implica que $p(x)W(v_{1},v_{2})(x)\equiv c$ es constante, ya que
\[
  [p(W(v_{1},v_{2})]':=[p(v_{1}v'_{2}-v'_{1}v_{2})]'=v_{1}\mathcal{L}v_{2}-v_{2}\mathcal{L}v_{1}\equiv 0.
\]
Veamos que $G$ es funci'on de Green; primero $G(x,t)=G(t,x)$ de la definicion, adem'as por como se definieron las condiciones iniciales de $\{v_{1},v_{2}\}$, es claro que $G$ cumple \ref{Green-fron}. Ahora, $G$ claramente es de clase $\mathcal{C}^{2}\left([a,b]\times[a,b]\setminus\Delta\right)$ y como $p(x)W(v_{1},v_{2})(x)\equiv c$ calculamos
  \begin{align*}
    &\mathcal{L}[G(x,t)]=p(x)\dfrac{\partial^{2} G}{\partial x^{2}}(x,t)+p'(x) + \dfrac{\partial G}{\partial x}+q(x)G(x,t)=\\
    &\begin{cases}
    c^{-1}v_{1}(t)[p(x)v''_{2}(x)+p'(x)v'_{2}(x)+q(x)v_{2}(x)]=0\quad a\leq t\leq x\leq b\\
    c^{-1}v_{2}(t)[p(x)v''_{1}(x)+p'(x)v'_{1}(x)+q(x)v_{1}(x)]=0\quad a\leq x\leq t\leq b.
  \end{cases}
  \end{align*}
  As'i claramente $\mathcal{L}G=0$. Por 'ultimo, vemos la discontinuidad de la parcial con respecto a $x$ de $G$ en la diagonal, es decir,
  \begin{align*} 
    \lim_{\delta\to^{+}0}\dfrac{\partial G}{\partial x}(x,x+\delta)-\dfrac{\partial G}{\partial x}(x,x-\delta)
    &=\frac{1}{c}\lim_{\delta\to0}[v_{1}(x)v'_{2}(x+\delta)-v'_{1}(x-\delta)v_{2}(x)]\\
    &=\frac{W(v_{1},v_{2})(x)}{p(x)W(v_{1},v_{2})(x)}=\frac{1}{p(x)}.
  \end{align*}
  Por lo tanto, $G$ es una funci'on de Green. Ahora definimos el siguiente operador tipo Fredholm en $\mathcal{L}^{2}([a,b],\mu)$
  \[
    T[\phi](x)=\int_{a}^{b}\phi(t)G(x,t)\,d\mu(t).
  \]
  \obs La definici'on de la funcion de Green y la demostraci'on de sus exitencia son v'alidas pra cualquier operador diferncial lineal de orden dos.
  \begin{teorema}
    Sea $L$ un operador lineal diferencial de segundo orden y $G$ su funci'on de Green asociada a un problema de frontera de tipo \ref{SL-fron} tal que $0$ no sea un valor propio. Sea $T$ el operador de tipo Fredholm definido por $G$, entonces $T$ es el operador inverso al operador $L$, es decir
    \[
      L[\phi]=f\,\iff\,\phi=T[f]\,\text{ y }\,T[L[\phi]]=\phi.
    \]
    Donde $\phi$ cumple las condiciones de frontera. Es decir, $T$ proporciona las soluciones a cualquier ecuaci'on diferencial lineal no homogenea de orden dos.
  \end{teorema}
  \dem Primero demostremos que $T[f]$ cumple con la ecuacion determinada por
  \[
    L=p(x)\dfrac{d^{2}}{dx^{2}}+q(x)\dfrac{d}{dx}+r(x)I.
  \]
  Sabemos que $T[f]$ es una funci'on de clase $\mathcal{C}^{2}[a,b]$ ya que $G$ es continua y diferenciable casi donde quiera, as'i por el corolario \ref{dom-dif} $T[f]$ dos veces diferenciable. Entonces al calcular las derivadas obtenemos
 \begin{align*}
   (T f)(x)&= \int_{a}^{x} G(x, \xi)f(\xi)\,d\xi + \int_{x}^{b} G(x, \xi)f(\xi)\,d\xi, \\
   (T f)'(x)&=\int_{a}^{x}\dfrac{\partial G}{\partial x}(x, \xi)f(\xi)\,d\xi +
              \int_{x}^{b} \dfrac{\partial G}{\partial x}(x, \xi)f(\xi)\,d\xi, \\
   (T f)''(x)&=\int_{a}^{x} \dfrac{\partial^{2} G}{\partial x^{2}}(x, \xi)f(\xi)\,d\xi +
              \dfrac{\partial G}{\partial x}(x, x^{-})f(x^{-}) \\
           &\quad + \int_{x}^{b} G_{xx}(x, \xi)f(\xi)\,d\xi - G_x(x, x^{+})f(x^{+}),
\end{align*}
donde hemos usado la continuidad de $G$ y $f$ en $\xi = x$ para obtener $(T f)'(x)$. Como
\[
  \dfrac{\partial G}{\partial x}(x, x^{-}) - \dfrac{\partial G}{\partial x}(x, x^{+}) = \frac{1}{p(x)},
\]
entonces obtenemos
\[
  (T f)''(x) = \int_{a}^{x} \dfrac{\partial^{2}G}{\partial x^{2}}(x, \xi)f(\xi)\,d\xi + \int_{x}^{b}
  \dfrac{\partial^{2}G}{\partial^{2}x}(x, \xi)f(\xi)\,d\xi + \frac{f(x)}{p(x)}.
\]
Por lo tanto, por linealidad de la integral, calculamos
\begin{align*}
  L[T[f]]&=p(x)(T[f])''+q(x)(T[f])'+r(x)T[f]\\
         &=\int_{a}^{x} L[G(x,\xi)]f(\xi)\,d\xi + \int_{x}^{b} L[G(x,\xi)]f(\xi)\,d\xi+f(x)\\
         &=f(x)
\end{align*}
Por lo tanto $L[T[f]]=f$, m'as a'un, claramente $T[f]$ cumple con las condiciones de frontera pues $G$ las cumple y la integral es lineal y conmuta con la derivaci'on con respecto a $x$. Ahora calculamos la composici'on opuesta, es decir, sea $\phi$ funci'on de clase $\mathcal{C}^{2}[a,b]$ que cumple con las condiciones de frontera \ref{SL-fron}, entonces calculamos

\begin{align*}
T(L\phi)(x) &= \int_{a}^{x}G(x,\xi)L\phi(\xi)\,d\xi+\int_{x}^{b} G(x,\xi)L\phi(\xi)\,d\xi \\
            &=\left.p(\xi)\left[\phi'(\xi)G(x,\xi)-\phi(\xi)\dfrac{\partial G}{\partial\xi}(x,\xi)\right]\right|_{a}^{x}
            +\int_{a}^{x} \phi(\xi)L_{\xi}[G(x,\xi)]\,d\xi\\
            &+\left. p(\xi)\left[\phi'(\xi)G(x,\xi)-\phi(\xi) \dfrac{\partial G}{\partial\xi}(x,\xi)\right]
              \right|_{x}^{b}+\int_{x}^{b} \phi(\xi)L_{\xi}[G(x,\xi)]\,d\xi \\
\end{align*}
Donde las integrales se hicieron por partes para disminuir los t'erminos de las derivadas de $\phi$ e incrementar las derivadas parciales de G con respecto a $\xi$, es decir en los terminos usuales de la integraci'on por partes
\[
\int_{a}^{b}u(\xi)\,dV(\xi)\,d\xi=\left.uV\right|_{a}^{b}-\int_{a}^{b}V\,du\,d\xi,
\]
se escoge las derivadas de $\phi$ como el termino $dV$ el numero de veces correspondiente as'i es claro que $L_{\xi}[G(x,\xi)]$ es el operador $L$ pero en t'erminos de las derivadas $d/d\xi$. Por lo tanto, como $G(x,\xi)=G(\xi,x)$, entonces $L_{\xi}G=0$ y por lo tanto en el calculo final se obtiene que
\begin{align*}
  T(L\phi)(x)&=\left. p(\xi)\phi(\xi)\dfrac{\partial G}{\partial \xi}(x,\xi)\right|_{x^-}^{x^+}+
               \left. p(\xi)\left[\phi'(\xi)G(x,\xi)-\phi(\xi)\dfrac{\partial G}{\partial \xi}(x,\xi)\right]\right|_{a}^{b}\\
            &=\phi(x).
\end{align*}
Pues $G$ es funcion de Green y por lo tanto cumple con las condiciones de frontera (junto con $\phi$) adem'as
\[
  \left.\dfrac{\partial G}{\partial \xi}(x,\xi)\right|_{x^-}^{x^+}=\frac{1}{p(x)}
\]
Por lo tanto $T=L^{-1}$. \QED

\eje Considere el problema de valores en la frontera
\begin{align*}
u'' + k^2 u &= 0, \quad 0 < x < 1,\,k\in\re\\
u(0) = u(1) &= 0.
\end{align*}
Como ya hemos visto, la solución general de la ecuación diferencial es
\[
  v(x) = c_1 \cos(kx) + c_2 \sin(kx)
\]
Notamos que $\alpha_1=\beta_1=1$ y $\alpha_2=\beta_2=0$, entonces buscamos la solución $v_1$ que satisface las siguientes condiciones iniciales en $x = 0$
\[
v_1(0) = \alpha_2 = 0, \quad v_1'(0) = -\alpha_1 = -1.
\]
Lo cual implica que $c_1=0$ y $kc_2= -1$. As'i $v_{1}$ es facil de calcular
\[
v_1(x) = -\frac{1}{k} \sin(kx).
\]
Ahora, la solución $v_2$ satisface las condiciones
\[
v_2(1) = \beta_2 = 0, \quad v_2'(1) = -\beta_1 = -1,
\]
se encuentra resolviendo el sistema
\begin{align*}
  v_{2}(x)=\gamma_{1}\cos(kx)+\gamma_{2}\sin(kx)\implies\gamma_{1}\cos(k)+\gamma_{2}\sin(k)=0\\
  v'_{2}(x)=-\gamma_{1}k\sin(kx)+\gamma_{2}k\cos(kx)\implies\gamma_{1}k\sin(k)-\gamma_{2}k\cos(k)=1.
\end{align*}
es decir, invirtiendo la matriz definida por el sistema tenemos que
\begin{equation}
  \begin{pmatrix}
    \gamma_{1}\\
    \gamma_{2}
  \end{pmatrix} = \frac{1}{k}
  \begin{pmatrix}
    k\cos(k) && \sin(k)\\
    k\sin(k) && -\cos(k)
  \end{pmatrix}
  \begin{pmatrix}
    0\\
    1
  \end{pmatrix}
\end{equation}
Por lo tanto
\[
  v_2(x) = \frac{\sin k}{k} \cos(kx) - \frac{\cos k}{k} \sin(kx).
\]
Entonces la funcion de Green es
\begin{equation}
G(x, \xi) =
\begin{cases}
c^{-1} v_1(\xi)v_2(x), & 0 \leq \xi \leq x \leq 1, \\
c^{-1} v_1(x)v_2(\xi), & 0 \leq x \leq \xi \leq 1.
\end{cases}
\end{equation}
Donde la constante $c$ está dada por 
\begin{align*}
  &\begin{vmatrix}
    -\sin(kx)/k && -\cos(kx)\\
    -\sin(k)/k\cos(kx)-\cos(k)/k\sin(kx) && -(\sin(k)\sin(kx)+\cos(k)\cos(kx))
  \end{vmatrix}\\
  &=\frac{\sin(k)}{k}
\end{align*}
Las otras propiedades de la funcion de Green son f'aciles de verificar.

\obs El oprador de tipo Fredholm $T$ definido por una funci'on de Green $G$ es un operador acotado en $\mathcal{L}^{2}([a,b],\mu)$ y en $(\mathcal{C},\|\cdot\|_{\infty})$ pues $G$ es continua en $[a,b]\times[a,b]$ como hemos visto previamente, donde $\|T\|\leq\sqrt{b-a}\|G\|_{\infty}$ en $\mathcal{L}^{2}$ y $\|T\|\leq(b-a)\|G\|_{\infty}$ con la norma infinito.

\subsubsection*{Calculo del espectro:}
\noindent Al principio de esta secci'on usamos el hecho de que not todo n'umero real va a ser parte del espectro de un operador de Strum-Liouville, el siguiente lema lo demuestra
\begin{lema}
  Los valores propio de un operador de Strum-Liouville $-\mathcal{L}$ se encuentran acotados inferiormente.
  \end{lema}
  \dem Sea $\phi\in\mathcal{C}^{2}$ que cumplan las condiciones de frontera \ref{SL-fron}, etnonces si calculamos
  \begin{align*}  \braket{-\mathcal{L}\phi\phi}&=\int_{a}^{b}\left(-(p(z)\phi'(x))'\overline{\phi(x)}-q(x)|\phi(x)|^{2}\right)\\
                              &=\int_{a}^{b}\left((p(z)\phi'(x))|\phi(x)|^{2}-q(x)|\phi(x)|^{2}\right)
                                                +p(a)\phi'(a)\phi(a)-p(b)\phi'(b)\phi(b)\\
                              &=\int_{a}^{b}\left((p(z)\phi'(x))|\phi(x)|^{2}-q(x)|\phi(x)|^{2}\right)
                              +p(a)\frac{\beta_{1}}{\beta_{2}}\phi(a)^{2}-p(b)\frac{\alpha_{1}}{\alpha_{2}}\phi(b)^{2}.
  \end{align*}
  Entonces si $\phi$ es vector propio con valor propio $\lambda$ y suponemos que tenemos como condiciones de frontera $\phi(a)=\phi(b)=0.$ Entonces
  \[
    \lambda\|\phi\|_{2}^{2}=\int_{a}^{b}\left((p(z)\phi'(x))|\phi(x)|^{2}-q(x)|\phi(x)|^{2}\right)\geq-\|\phi\|^{2}_{2}\|q\|_{\infty}.
  \]
  Por lo tanto $-\|q\|_{\infty}$ es una cota inferior para $\lambda$. Por otro lado, si $\phi$ satisface las condiciones de forntera \ref{SL-fron},
entonces el siguiente argumento dimensional muestra que no puede haber más de dos funciones propias linealmente independientes de $-\mathcal{L}$ con valores propios menores que $\|q\|_{\infty}$.

Buscando una contradicción, supongamos que $-\mathcal{L}$ tiene tres funciones propias linealmente independientes $\phi_1$, $\phi_2$ y $\phi_3$ con sus correspondientes valores propios $\lambda_1$, $\lambda_2$ y $\lambda_3$ todos menores que $\|q\|$. Podemos asumir, sin pérdida de generalidad, que las funciones propias son ortonormales. Dado que
\[
\alpha_1 \phi_i(a) + \alpha_2 \phi_i'(a) = 0, \quad \beta_1 \phi_i(b) + \beta_2 \phi_i'(b) = 0, \quad i = 1,2,3,
\]
emos que cada uno de los seis vectores $(\phi_i(a), \phi_i'(a))$ y $(\phi_i(b), \phi_i'(b))$ yace en un subespacio unidimensional de $\mathbb{R}^2$. Por lo tanto, los tres vectores $\mathbf{u}_i = (\phi_i(a), \phi_i'(a), \phi_i(b), \phi_i'(b))$ yacen en un subespacio bidimensional de $\mathbb{R}^4$. Podemos entonces formar una combinación lineal $c_1 \mathbf{u}_1 + c_2 \mathbf{u}_2 + c_3 \mathbf{u}_3$, donde no todos los coeficientes son cero, tal que
\[
c_1 \mathbf{u}_1 + c_2 \mathbf{u}_2 + c_3 \mathbf{u}_3 = 0.
\]
Esto implica que:
\[
c_1 \phi_1(a) + c_2 \phi_2(a) + c_3 \phi_3(a) = 0,
\]
\[
c_1 \phi_1(b) + c_2 \phi_2(b) + c_3 \phi_3(b) = 0.
\]

La función
\[
v(x) = c_1 \phi_1(x) + c_2 \phi_2(x) + c_3 \phi_3(x)
\]
es por tanto una función propia de $-L$ que satisface $v(a) = v(b) = 0$, y en consecuencia su valor propio está acotado inferiormente por $\|q\|$. Sin embargo, esto contradice la desigualdad
\[
\langle -L v, v \rangle = \lambda_1 |c_1|^2 + \lambda_2 |c_2|^2 + \lambda_3 |c_3|^2 < \|q\|(|c_1|^2 + |c_2|^2 + |c_3|^2) = \|q\| \|v\|^2.
\]
\QED.

Ahora lo que se busca es resolver un problema de Sturm-Liouville \ref{SL}, es decir resolver la ecuacion $\mathcal{L}\phi+\lambda\phi=0$ con su correspondientes condiciones de frontera \ref{SL-fron}. Entonces como el operador $T$ es el operador inverso a $\mathcal{L}$ tenemos
\[
\mathcal{L}\phi+\lambda\phi=0\implies\phi+\lambda T\phi=0\implies T\phi=\frac{-1}{\lambda}\phi.
\]
Es decir, obtenemos una ecuacion de valores propios/vectores propios de $T$, el cual es un operador más fácil de estudiar.
\end{document}

\documentclass[main.tex]{subfiles}
\begin{document}
\chapter{Series de Fourier}
\noindent Comenzaremos considerando el problema que originalmente motivó el estudio de las series de Fourier, el problema de la ecuación de onda en una dimensión para una cuerda fija, o más bien una ecuación equivalente a ésta, \emph{la ecuación de Schrödinger pozo infinito} ¿Por qué la ecuación de Schródinger y no la ecuación para una cuerda fija? Dos razones, una porque creemos que la matemática de la mecánica cuántica es más interesante y es más fácil explicar tópicos importantes del análisis funcional con ésta y dos porque la manipulación algebraica de las series de Fourier complejas es más consisa que de las reales.

La ecuación de Schrödinger en una dimensión es una ecuación diferencial parcial que describe la evolución temporal y espacial de la función de onda \(\psi(x, t)\) de una partícula cuántica. Para esolverla utilizaremos  un métido que exploraremos a mayor profundidad posteriormente, el \emph{método de separación de variables o método de Bernoulli}, el cual en términos de mecánica cuántica, es válido para el caso de un potencial independiente del tiempo. Luego, aplicaremos este método al caso particular del \emph{pozo infinito} y mostraremos cómo este problema está relacionado con las \emph{series Fourier}.

\section{Ecuación de Schrödinger en una Dimensión}

La ecuación de Schrödinger en una dimensión para una partícula de masa \(m\) sometida a un potencial \(V(x,t)\) está dada por
\[
i\hbar \frac{\partial \psi(x, t)}{\partial t} = -\frac{\hbar^2}{2m} \frac{\partial^2 \psi(x, t)}{\partial x^2} + V(x,t) \psi(x, t),
\]
donde \(\hbar\) es la constante de Planck reducida y \(V(x,t)\) es el potencial, el cual suponemos independiente del tiempo, es decir \(V(x,t)=V(x)\) para poder utilizar el \emph{Método de Separación de Variables o método de Bernoulli}, donde suponemos que la solulción \(\psi(x, t)\) (también conocida como la función de onda) se expresa de la forma
\[
\psi(x, t) = \varphi(x) \phi(t).
\]
Sustituyendo esta expresión en la ecuación de Schrödinger, obtenemos
\[
i\hbar \varphi(x) \frac{d \phi(t)}{dt} = -\frac{\hbar^2}{2m} \phi(t) \frac{d^2 \varphi(x)}{dx^2} + V(x) \varphi(x) \phi(t).
\]
Dividiendo ambos lados por \(\psi(x, t) = \varphi(x) \phi(t)\), obtenemos
\[
i\hbar \frac{1}{\phi(t)} \frac{d \phi(t)}{dt} = -\frac{\hbar^2}{2m} \frac{1}{\varphi(x)} \frac{d^2 \varphi(x)}{dx^2} + V(x).
\]
Como el lado izquierdo depende solo de \(t\) y el lado derecho depende solo de \(x\), ambos lados deben ser iguales a una constante, que denotamos por \(E\). Esto nos da dos ecuaciones diferenciales ordinarias

\begin{enumerate}
    \item \textbf{Ecuación Temporal}:
    \[
    i\hbar \frac{d \phi(t)}{dt} = E \phi(t).
    \]
    La solución de esta ecuación es:
    \[
    \phi(t) = \phi(0) e^{-iEt/\hbar}.
    \]

    \item \textbf{Ecuación Espacial}:
    \[
    -\frac{\hbar^2}{2m} \frac{d^2 \varphi(x)}{dx^2} + V(x) \varphi(x) = E \varphi(x).
    \]
    Esta es la \emph{ecuación de Schrödinger independiente del tiempo}, la cual define el \emph{operador hamiltoniano cuántico}
    \[
    H =-\frac{\hbar^2}{2m} \frac{d^2 }{dx^2} + V(x)I,
    \]
    además \(E\) representa la energía del estado estacionario, es decir, la función de onda y la energía cumplen con una ecuación del tipo
    \[
    H\varphi=E\varphi,
    \]
    también conocida como ecuación de vectores propios o \emph{escuación espectral}.
\end{enumerate}

Entonces una solución particular es de la forma \(\psi(x,t)=\varphi(x)e^{-iEt/\hbar}\), observamos que \(|\psi|=|\phi(x)|\), esto en términos de las probabilidades cuánticas significa que éstas son independientes del tiempo.\\
\obs Dado un operador \(A\) en un espacio de Hilber, se define su valor esperado como \(\braket{A}=\braket{\psi| A \psi}\), donde \(\psi\) es cualquier vector de norma uno. En el caso particular del hammiltoniano, usamos \(\psi=\varphi\) vector propio de norma uno, entonces
\[
    \braket{H}=\braket{\varphi|H\varphi}=\int\overline{\varphi}H\varphi\,d\mu=\int\overline{\varphi}E\varphi\,d\mu=E.
\]
\noindent Más aún se define su desviación estandard de un operador \(A\), se define la desviación estandard de \(A\) como
\[
    \sigma_A=\sqrt{\braket{A^2}}=\sqrt{\braket{\varphi|A^2\varphi}}=\quad\|\varphi\|=1.
\]
\exe Demuestre que \(\sigma_H=0\).\\
Ahora si \(\{\phi_1,\phi_2,\dots\}\) con energías \(\{E_1,E_2,\dots\}\) son vectores propios del hammiltoniano, entonces por linealidad
\[
\psi(x,t)=\sum_{k=1}^nc_k\varphi_k(x)e^{-iE_kt/\hbar},\quad c_k\in\co,
\]
es solución, enttonces naturalmente uno se pregunta si
\[
\psi(x,t)=\sum_{k=1}^{\infty}c_k\varphi_k(x)e^{-iE_kt/\hbar},\quad c_k\in\co,
\]
es solución y si son todas de esta forma.
\subsubsection{Caso Particular: Pozo Infinito}
\noindent Consideremos el caso de un \emph{pozo infinito}, donde el potencial \(V(x)\) está dado por:
\[
V(x) =
\begin{cases}
0, & \text{si } 0 < x < L, \\
\infty, & \text{en otro caso}.
\end{cases}
\]
Este potencial describe una partícula confinada en una región de longitud \(L\), con paredes impenetrables en \(x = 0\) y \(x = L\). El cual establece las condiciones de frontera, donde la función de onda \(\varphi(x)\) debe anularse en \(x = 0\) y \(x = L\).
Ahora para resolver la ecuación espacial, sentro del pozo (\(0 < x < L\)), el potencial es cero, por lo que la ecuación de Schrödinger independiente del tiempo se reduce a
\[
-\frac{\hbar^2}{2m} \frac{d^2 \varphi(x)}{dx^2} = E \varphi(x).
\]
Reescribiendo, obtenemos
\[
\frac{d^2 \varphi(x)}{dx^2} + k^2 \varphi(x) = 0,
\]
donde \(k^2 = \frac{2mE}{\hbar^2}\), así conocemos la solución general de esta ecuación,
\[
\varphi(x) = A \sin(kx) + B \cos(kx).
\]
Aplicando las condiciones de frontera \(\varphi(0) = 0\) y \(\varphi(L) = 0\), obtenemos
\[
\varphi(0) = B = 0 \quad \Rightarrow \quad \varphi(x) = A \sin(kx).
\]
Además, \(\varphi(L) = A \sin(kL) = 0\). Para que esta ecuación se satisfaga, \(kL\) debe ser un múltiplo entero de \(\pi\), es decir
\[
kL = n\pi \quad \Rightarrow \quad k = \frac{n\pi}{L}, \quad n = 1, 2, 3, \dots
\]
Por lo tanto, las funciones de onda normalizadas son
\[
\varphi_n(x) = \sqrt{\frac{2}{L}} \sin\left(\frac{n\pi x}{L}\right),
\]
y las energías correspondientes son
\[
E_n = \frac{\hbar^2 k^2}{2m} = \frac{\hbar^2 n^2 \pi^2}{2mL^2}.
\]
\obs Nótese que el conjunto de valores propios del hammiltoniano es un \emph{conjunto discreto,} ésta es una propiedad de ciertos operadores conocidos como \emph{operadores Hilbert-Schmidt} o \emph{operadores de traza}.
\subsection{Soluciónes Completas y Series de Fourier}
\noindent De la discusión anterior, tenemos que una solución de la función de onda \(\psi(x, t)\) para el pozo infinito es
\[
\psi_n(x, t) = \varphi_n(x) \phi_n(t) = \sqrt{\frac{2}{L}} \sin\left(\frac{n\pi x}{L}\right) e^{-iE_n t/\hbar}.
\]

La soluciones del pozo infinito, nos indican naturalmente a las series de Fourier, de forma análoga que al desarollo hecho por Fourier y Cauchy en el siglo XIX, donde en sus casos, estudiaron la ecuación de una cuerda fija y la ecuación de calor en una barra. Se demostrará que las funciones de onda \(\varphi_n(x)\) forman una base ortonormal en el espacio de funciones definidas en el intervalo \([0, L]\), es decir, vamos a demostrar que cualquier función \(f(x)\) cuadrado integrable en este intervalo puede expresarse como una combinación lineal de estas funciones de onda
\[
f(x) = \sum_{n=1}^\infty c_n \varphi_n(x),
\]
donde los coeficientes \(c_n\) están dados por
\[
c_n = \int_0^L f(x) \varphi_n(x) \, d\lambda.
\]
Esta es precisamente la idea detrás de las \emph{series de Fourier}, donde una función se expresa como una suma infinita de los conocidos como \emph{polinomios trigonométricos}
\[
s_n(f)(x)=\frac{a_0}{2}+\sum_{k=1}^n\left(a_k\cos\left(\frac{k\pi x}{L}\right)+ b_k\sin\left(\frac{k\pi x}{L}\right)\right),
\]
donde los coeficientes \(a_n\) y \(b_n\) son
\[
  a_n = \frac{2}{L} \int_0^L f(x) \cos\left(\frac{n\pi x}{L}\right) \, d\lambda\quad b_n = \frac{2}{L} \int_0^L f(x) \sin\left(\frac{n\pi x}{L}\right) \, d\lambda.
\]
\obs \(\)
En el caso del pozo infinito, las funciones de onda \(\varphi_n(x)\) son funciones seno, lo que corresponde a una serie de Fourier en senos
\[
f(x) = \sum_{n=1}^\infty b_n \sin\left(\frac{n\pi x}{L}\right),
\]

\subsection{Ortogonalidad de los monomios triginométicos}
\noindent Demostraremos que los monomios trigonométicos son un conjunto ortogonal en el intervalo \([-\pi,\pi]\), en general esto se puede ajustar a cualquier intervalo \([a,b]\) utilizando una transformación afín. En concreto demostraremos que si \(n\) y \(m\) son enteros no negativos, entonces
\[
\braket{ \cos(nx)| \cos(mx)} =
\begin{cases}
0 & \text{si } n \neq m, \\
\pi & \text{si } n = m \neq 0, \\
2\pi & \text{si } n = m = 0.
\end{cases}
\]
Analogamente con la funcón seno
\[
\braket{\sin(nx)| \sin(mx)} =
\begin{cases}
0 & \text{si } n \neq m, \\
\pi & \text{si } n = m.
\end{cases}
\]
y neturalmente también se tiene que
\[
\braket{\cos(nx) | \sin(mx) }= 0 \quad \text{para todo } n, m.
\]
Lo cual establece que el conjunto \(\{\cos(nx),\sin(nx)\}_{n\in\nat}\) es un sistema ortogonal de funciones, asimismo lo anterior establece que los monmios \(\{e^{inx}\}_{n\in\nat}\) es un sistema ortogonal de funciones. Sin embargo mostrar que es una base ortonormal de fucniones requerirá de más trabajo. Comenzamos  estableciendo la ortogonalidad entre seno y coseno.
Usamos la identidad trigonométrica
\[
\cos(nx) \sin(mx) = \frac{1}{2} \left( \sin((n+m)x) + \sin((n-m)x) \right).
\]
Integrando sobre \([-\pi, \pi]\)
\[
\int_{-\pi}^{\pi} \cos(nx) \sin(mx) \, dx = \frac{1}{2} \int_{-\pi}^{\pi} \sin((n+m)x) \, dx + \frac{1}{2} \int_{-\pi}^{\pi} \sin((n-m)x) \, dx.
\]
Ambas integrales son cero porque las integrales de \(\sin(kx)\) sobre \([-\pi, \pi]\) son cero para cualquier \(k\) y  por lo tanto
\[
\langle \cos(nx), \sin(mx) \rangle = 0 \quad \text{para todo } n, m.
\]
El resto de los cáclculos son similares y se dividen en los siguientes casos
\subsubsection{Caso \(n \neq m\)}
Comenzamos con los cosenos, si usamos la identidad trigonométrica
\[
\cos(nx) \cos(mx) = \frac{1}{2} \left( \cos((n+m)x) + \cos((n-m)x) \right).
\]
Integramos sobre \([-\pi, \pi]\)
\[
\int_{-\pi}^{\pi} \cos(nx) \cos(mx) \, dx = \frac{1}{2} \int_{-\pi}^{\pi} \cos((n+m)x) \, dx + \frac{1}{2} \int_{-\pi}^{\pi} \cos((n-m)x) \, dx.
\]
Ambas integrales son cero porque las integrales de \(\cos(kx)\) sobre \([-\pi, \pi]\) son cero para \(k \neq 0\) y por lo tanto
\[
\langle \cos(nx), \cos(mx) \rangle = 0 \quad \text{si } n \neq m.
\]
Similarmente para los senos
\[
\sin(nx) \sin(mx) = \frac{1}{2} \left( \cos((n-m)x) - \cos((n+m)x) \right).
\]
Entonces igualmente ambas integrales son cero y por lo tanto
\[
\langle \sin(nx), \sin(mx) \rangle = 0 \quad \text{si } n \neq m.
\]
\subsubsection{Caso \(n = m \neq 0\)}
Usamos la identidad trigonométrica
\[
\cos^2(nx) = \frac{1 + \cos(2nx)}{2}.
\]
Integramos sobre \([-\pi, \pi]\)
\[
\int_{-\pi}^{\pi} \cos^2(nx) \, dx = \frac{1}{2} \int_{-\pi}^{\pi} 1 \, dx + \frac{1}{2} \int_{-\pi}^{\pi} \cos(2nx) \, dx.
\]
La primera integral es \(\pi\) y la segunda es cero y por lo tanto
\[
\langle \cos(nx), \cos(nx) \rangle = \pi \quad \text{si } n = m \neq 0.
\]
Para el seno es basicamente lo mismo
\[
\sin^2(nx) = \frac{1 - \cos(2nx)}{2}.
\]
entonces
\[
\int_{-\pi}^{\pi} \sin^2(nx) \, dx = \frac{1}{2} \int_{-\pi}^{\pi} 1 \, dx - \frac{1}{2} \int_{-\pi}^{\pi} \cos(2nx) \, dx.
\]
Por lo tanto
\[
\langle \sin(nx), \sin(nx) \rangle = \pi \quad \text{si } n = m.
\]
Ahora existe un caso ``especial'', cuando \(n = m = 0\), \(\cos(0x) = 1\) y \(\sin(0x)=0\), entonces sólo tenemos la integral de 1
\[
\int_{-\pi}^{\pi} 1 \, dx = 2\pi.
\]

\subsection{Convergencia y Fórmula de Dirichlet}
\noindent Usando que \(\{\cos(nx),\sin(nx)\}_{n\in\nat}\) es un sistema ortogonal de funciones podemos garantizar la convergencia de los coeficientes de las series de Fourier. Para demostrar esto usaremos la \emph{desigualdad de Bessel}
\begin{teorema}
  Sea \(\mathcal{H},\braket{\cdot|\cdot}\) un espacio de hilbert y \(\ket{e_i}\}_{i\in\Lambda}\) un conjunto ortonormal, entonces para todo \(\ket{\psi}\in\mathcal{H}\)
  \[
  \sum_{i\in\Lambda}|\braket{e_i|\psi}|^2\leq \|\psi\|^2.
  \]
\end{teorema}
\dem Sea
\[
        \textrm{Proy}=\sum_{i\in\Lambda} \ket{e_i}\bra{e_i}\quad\textrm{Proy}\ket{\psi}=\ket{\phi},
\]
entonces \(\|\psi-\phi\|^2=\braket{\psi-\phi|\psi-\phi}\geq0\). Por lo que calculamos
\begin{align*}
  \braket{\psi-\phi|\psi-\phi}&=\braket{\psi-\sum_{i\in\Lambda}\braket{e_i|\psi}\ket{e_i}|\psi-\sum_{j\in\Lambda}\braket{e_j|\psi}\ket{e_j}}\\
  &=\braket{\psi|\psi}-\sum_{j\in\Lambda}\overline{\braket{e_j|\psi}}\braket{\psi|e_j}-\sum_{i\in\Lambda}\braket{e_i|\psi}\braket{e_i|\psi}+\sum_{i,j\in\Lambda}\braket{e_i|\psi}\overline{\braket{e_j|\psi}}\braket{e_i|e_j}\\
  &=\braket{\psi|\psi}-\sum_{i\in\Lambda}|\braket{e_i|\psi}|^2\geq0
  \end{align*}
\end{document}

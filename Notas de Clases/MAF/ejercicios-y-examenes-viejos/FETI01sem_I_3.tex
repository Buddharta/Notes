
\input exfeti.sty

% 20 de oct. 1989
\newbox\Ancha
\def\gros#1{{\setbox\Ancha=\hbox{$#1$}
   \kern-.025em\copy\Ancha\kern-\wd\Ancha
   \kern.05em\copy\Ancha\kern-\wd\Ancha
   \kern-.025em\raise.0433em\box\Ancha}}
%\def\gros#1{\vec #1}
\font\titulo=cmbxti10 scaled 5000
\font\biggfnt=cmr10 scaled\magstep 3

\font\bigfnt=cmr10 scaled\magstep 1
\def\ni{\noindent}
\def\esp{\par \vskip .2 in }
\baselineskip 18pt \nopagenumbers

\def\ni{\noindent}
\baselineskip 15pt \magnification 1100
\def\Par{\par\vskip .15in}
\def\esp{\Par \vskip .2 in }
%----------------------------------empieza--------------------------------------

 \centerline{\biggfnt Funciones
Especiales}\par\centerline{\slll y}\Par
 \centerline{\biggfnt Transformadas Integrales}\Par

{\baselineskip 8pt \centerline{\it Examen 3}\par
\centerline{\smalll \hoy}
\Par

%-----------------------------------figura---------------------------------------------

%\newbox\z
\gdef\fetifig#1#2#3{\setbox1=\vbox{
%%Empieza figura
\let\picnaturalsize=N
\def\picsize{#3}
\def\picfilename{#1}
%If you do not have the picture file add:
%\let\nopictures=Y
%to the beginning of the file.
\ifx\nopictures Y\else{\ifx\epsfloaded Y\else\input epsf \fi
\global\let\epsfloaded=Y \hskip 2cm{\ifx\picnaturalsize
N\epsfxsize \picsize\fi \epsfbox{\picfilename}}}\fi
%%Termina figura
 \setbox\z=\hbox{}
\copy\z}
 \setbox2=\vbox{\vskip
4pt\splittopskip=\baselineskip\hsize 5cm {\hskip -60pt
{\titulo#2}}} \centerline{$\hskip -3cm\vcenter{\box1}\hskip
-4.8cm\vcenter{\box2}\hfill$} }
%\input epsf.sty
\vskip -160pt 
\fetifig{unamN3.eps}{\titulo}{2.5cm}


\vskip 70pt
\libro
\ni 1) Deduce la f\'ormula de  expansi\'on:

$$ \int_0^z J_\nu(t)\, dt = 2\sum_{k=0}^\infty J_{\nu + 2k +1}\, (z)\qquad {\hbox{Re} }\,(\nu)> -1. $$

\ni 2) Deduce el valor de la integral definida siguiente:

$$ \int_0^1 J_n (zt) \, t^{n +1}\, dt = \left[ {J_{n + 1} (z) \over z}\right]  .$$



\ni 3) {\bf Integral de Bessel.}  El contorno ${\sl C}$ de la figura, est\'a formado por la linea que va por debajo de la l\'{\i}nea de corte desde $-\infty$ hasta -1,  despu\'es pasa por  el c\'{\i}rculo unitario desde $e^{-i\pi}$ hasta $e^{i\pi}$  y finalmente va de nuevo desde  $-\infty$ hasta  -1 pero ahora  por encima de la l\'{\i}nea de corte. 

\input epsf

%%Begin InstantTeX Picture
\let\picnaturalsize=N
\def\picsize{8cm}
\def\picfilename{FETI01sem_I_3.eps}
%If you do not have the picture file add:
%\let\nopictures=Y
%to the beginning of the file.
\ifx\nopictures Y\else{\ifx\epsfloaded Y\else\input epsf \fi
\global\let\epsfloaded=Y
\centerline{\ifx\picnaturalsize N\epsfxsize \picsize\fi \epsfbox{\picfilename}}}\fi
%%End InstantTeX Picture


\ni Prueba que la expresi\'on

$$ J_\nu (x) = {1\over 2\pi i } \int_{\sl C} e^{(x/2)(t-1/t) } t^{-\nu -1}\, dt. $$

\ni Se convierte mediante el cambio de variable $u= te^{\pm i\pi} $en. 
 

$$ J_\nu (x) = {1\over \pi } \int_{0}^{2\pi} \hbox{cos} (\nu \theta - x \hbox{sen} \theta)\, d\theta - {\hbox{sen} \nu \pi \over \pi} \int_0^\infty e^{(-\nu\theta - x\,\hbox{\small senh}\theta) }d\theta. $$

\ni 4) Prueba que

$$ \delta(x-x_0) = \sum_{n=-\infty}^\infty {2n + 1\over \pi} j_n(x)\, j_n(x_0) $$

\ni en donde la $j_n(x)$ es una funci\'on esf\'erica de Bessel.\Par


\ni {\bf F\'ormulas, f\'ormulas....} \par

\ni Serie de potencias de la funci\'on {\it Bessel}

$$J_\nu(x) =\sum_{k=0}^\infty {(-1)^k\over k!\,(k + \nu)! } \left({x\over 2}\right)^{\nu + 2k}.$$

\ni Relaciones de recurrencia

$$  J_{\nu -1}(x) + J_{\nu + 1}(x) = \left({2\nu\over x}\right) J_\nu (x),\qquad J_{\nu -1}(x) - J_{\nu + 1}(x) = 2\, J_\nu'(x).$$

\ni Ortonormalidad de las funciones de Bessel esf\'ericas

$$  \int_{-\infty}^\infty j_m(x) j_n(x) \, dx = {\pi \over 2n +1 }\, \delta_{n m}$$



\vfill \eject
\end

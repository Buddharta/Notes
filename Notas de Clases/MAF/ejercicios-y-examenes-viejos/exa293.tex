\newbox\Ancha
\def\gros#1{{\setbox\Ancha=\hbox{$#1$}
   \kern-.025em\copy\Ancha\kern-\wd\Ancha
   \kern.05em\copy\Ancha\kern-\wd\Ancha
   \kern-.025em\raise.0433em\box\Ancha}}
%\def\gros#1{\vec #1}
\font\biggfnt=cmr10 scaled\magstep 3
\font\bigfnt=cmr10 scaled\magstep1
\def\ni{\noindent}
\def\esp{\par \vskip .2 in }
\baselineskip 22pt
\magnification 1200
\centerline{\biggfnt EXAMEN 3}\esp
\centerline{\bigfnt F\'{\i}sica Moderna III}\esp

\ni 1) Calcular la corriente para una onda plana de un electr\'on 
libre de energ\'ia positiva y de impulso ${\bf k}$
%
$$ u({\bf k}) = \sqrt{{E+m_{{}_0}\over 2E}}\pmatrix{ \chi \cr {}\cr
{{\gros\sigma}\cdot {\bf k}\over E+m_{{}_0}}\chi \cr}$$
%
en donde $\chi$ es un bi-espinor ($\hbar=c=1$).\par

\ni 2) Calcular el conmutador con el hamiltoniano $H={\gros\alpha}\cdot 
{\bf p}+{\beta}m_{{}_0}$ del operador $ {\gros \sigma}\cdot \hat n$, 
en donde ${\hat n}$ es un vector unidad. Para simplificar los c\'alculos 
se puede tomar $\hat n$ paralelo a $ {\bf k}$.\par
\ni 3) Misma pregunta que en 2 para el operador
$\beta {\gros \sigma}\cdot\hat n$. \par

\ni 4) Definiremos el vector de polarizaci\'on ${\bf P}$ como:
%
$${\bf P}= {\gros \sigma}_{\|}+\beta{\gros \sigma}_{\bot}$$
en donde
%
$${\gros \sigma}_{\|}={({\gros \sigma}\cdot{\bf k}){\bf k}\over |{\bf k}|^2}$$
y
$${\gros \sigma}_{\bot}= {\gros \sigma}-{\gros \sigma}_{\|}.$$
%
Mostrar que ${\bf P}$  conmuta con el hamiltoniano $H.$\par 
\ni 5) Calcular $ {\bf P}\cdot{\hat n}$ y ${\bf P}^2.$ 
?`Se pueden diagonalizar simult\'aneamente $ H,{\bf P}_{{}_x},
{\bf P}_{{}_y}$ y ${\bf P}_{{}_z}$?\par
\vfill
\eject
\end



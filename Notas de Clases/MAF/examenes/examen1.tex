% Created 2025-02-06 Thu 20:56
% Intended LaTeX compiler: pdflatex
\documentclass[letterpaper]{article}
\usepackage[utf8]{inputenc}
\usepackage[T1]{fontenc}
\usepackage{graphicx}
\usepackage{longtable}
\usepackage{wrapfig}
\usepackage{rotating}
\usepackage[normalem]{ulem}
\usepackage{amsmath}
\usepackage{amssymb}
\usepackage{capt-of}
\usepackage{hyperref}
\usepackage{lmodern} % Ensures we have the right font
\usepackage[utf8]{inputenc}
\usepackage{graphicx}
\usepackage{amsmath, amsthm, amssymb, amsfonts, amssymb, amscd}
\usepackage[table, xcdraw]{xcolor}
%\usepackage{mdsymbol}
\usepackage{tikz-cd}
\usepackage{float}
\usepackage[spanish, activeacute, ]{babel}
\usepackage{color}
\usepackage{transparent}
\graphicspath{{./figs/}}
\usepackage{makeidx}
\usepackage{afterpage}
\usepackage{array}
\usepackage{braket}
\usepackage{pst-node}
\newtheorem{teorema}{Teorema}[section]
\newtheorem{prop}[teorema]{Proposici\'on}
\newtheorem{cor}[teorema]{Corolario}
\newtheorem{lema}[teorema]{Lema}
\newtheorem{def.}{Definici\'on}[section]
\newtheorem{afir}{Afirmaci\'on}
\newtheorem{conjetura}{Conjetura}
\renewcommand{\figurename}{Figura}
\renewcommand{\indexname}{\'{I}ndice anal\'{\i}tico}
\newcommand{\zah}{\ensuremath{ \mathbb Z }}
\newcommand{\rac}{\ensuremath{ \mathbb Q }}
\newcommand{\nat}{\ensuremath{ \mathbb N }}
\newcommand{\prob}{\textbf{P}}
\newcommand{\esp}{\mathbb E}
\newcommand{\eje}{{\newline \noindent \sc \textbf{Ejemplo. }}}
\newcommand{\obs}{{\newline \noindent \sc \textbf{Observación. }}}
\newcommand{\dem}{{\noindent \sc Demostraci\'on. }}
\newcommand{\bg}{\ensuremath{\overline \Gamma}}
\newcommand{\ga}{\ensuremath{\gamma}}
\newcommand{\fb}{\ensuremath{\overline F}}
\newcommand{\la}{\ensuremath{\Lambda}}
\newcommand{\om}{\ensuremath{\Omega}}
\newcommand{\sig}{\ensuremath{\Sigma}}
\newcommand{\bt}{\ensuremath{\overline T}}
\newcommand{\li}{\ensuremath{\mathbb{L}}}
\newcommand{\ord}{\ensuremath{\mathbb{O}}}
\newcommand{\bs}{\ensuremath{\mathbb{S}^1}}
\newcommand{\co}{\ensuremath{\mathbb C }}
\newcommand{\con}{\ensuremath{\mathbb{C}^n}}
\newcommand{\cp}{\ensuremath{\mathbb{CP}}}
\newcommand{\rp}{\ensuremath{\mathbb{RP}}}
\newcommand{\re}{\ensuremath{\mathbb R }}
\newcommand{\hc}{\ensuremath{\widehat{\mathbb C} }}
\newcommand{\pslz}{\ensuremath{\mathrm{PSL}(2,\mathbb Z) }}
\newcommand{\pslr}{\ensuremath{\mathrm{PSL}(2,\mathbb R) }}
\newcommand{\pslc}{\ensuremath{\mathrm{PSL}(2,\mathbb C) }}
\newcommand{\hd}{\ensuremath{\mathbb H^2}}
\newcommand{\slz}{\ensuremath{\mathrm{SL}(2,\mathbb Z) }}
\newcommand{\slr}{\ensuremath{\mathrm{SL}(2,\mathbb R) }}
\newcommand{\slc}{\ensuremath{\mathrm{SL}(2,\mathbb C) }}
\newcommand{\mdlr}{\ensuremath{\mathrm{M}}}
\date{\today}
\title{Examen 1 de MAF}
\begin{document}

\maketitle

\noindent\textbf{Instrucciones: Tienen 2 horas y media para resolver el examen. Cada problema vale 2.5 puntos. TODAS sus hojas de examen deben tener su nombre SIN EXCEPCION.}

\begin{enumerate}
\item Sea \(J\in\textrm{M}_{2\times 2}(\co)\) la matriz,
\[
      J=\begin{pmatrix} 0 & 1\\
                       -1 & 0
        \end{pmatrix}.
\]
Calcule la matriz exponencial y demuestre que
\[
        A(\theta)=\exp(\theta J)=\cos\theta I_{2} + \sin\theta J.
   \]
Donde \(I_{2}\) es la matriz identidad, además muestre que \(\det(A(\theta))=\|A(\theta)\|=1\).
\item Utilice el m'etodo de Bernoulli o m'etodo de separaci'on de variables para resolver el problema de la cuerda vibrante de longitud \(L\) con extremos fijos, es decir, utilice el m'etodo para reducir la ecuaci'on
  \[
  \dfrac{\partial^{2} u}{\partial t^{2}}=\dfrac{\partial^{2}u}{\partial x^{2}}
  \]
  a dos ecuaciones ordinarias. Utilize las condiciones de frontera \(u(0,t)=u(L,t)=0\) para demostrar que las frecuencias en dicha cuerda son \emph{discretas}.
\noindent\emph{Punto extra: Suponga que las series de Fourier son densas determinar la soluci'on con condici'on inicial
  \[
  u(x,0)=\phi(x),\,\phi(0)=\phi(L).\text{ donde $\phi$ es diferenciable.}
  \]
 Calcule la soluci'on para $\phi(x)=x(x-L)$}
\item Demuestre que dada \(X,\sig,\mu\) un espacio de medida, y \(\mathbb{B}_{\re}\) la \(\sigma\)-álgebra de borel en \(\re\) y \(f:X\to\re\) una funci'on no negativa. La funci'on
\[
\nu:\sig\to\re\quad\nu(E)=\int_{E}fd\,\mu
\]
es una medida tal que todo conjunto nulo de \((\sig,\mu)\) es nulo para \((\sig,\nu)\).
\item La función Gamma para los reales positivos, \(\Gamma:\re^{+}\to\re\)
  \begin{itemize}
    \item Sea \(\alpha>0\), pruebe que \(\int_0^{\infty}e^{-x}x^{\alpha-1}\,dx=\Gamma(\alpha)\) existe.\\
    \emph{Sugerencia: Examine por separado la integral en \((0,1)\) y \([1,\infty)\) y acote adecuadamente}.

    \item Demuestre que \((\Gamma)(z+1)=z\Gamma(z)\), calcule \(\Gamma(n)\) para toda \(n\in\nat\)
    \end{itemize}

\end{enumerate}
\end{document}

\input exfeti.sty
% 20 de oct. 1989
\newbox\Ancha
\def\gros#1{{\setbox\Ancha=\hbox{$#1$}
   \kern-.025em\copy\Ancha\kern-\wd\Ancha
   \kern.05em\copy\Ancha\kern-\wd\Ancha
   \kern-.025em\raise.0433em\box\Ancha}}
%\def\gros#1{\vec #1}
\font\biggfnt=cmr10 scaled\magstep 3
\font\bigfnt=cmr10 scaled\magstep 1
\def\ni{\noindent}
\def\esp{\par \vskip .2 in }
\baselineskip 18pt
\nopagenumbers

\def\ni{\noindent}  
\baselineskip 15pt    
\magnification 1100 
\def\Par{\par\vskip .15in} 
\def\esp{\Par \vskip .2 in }
 
%%Begin InstantTeX Picture
\let\picnaturalsize=N
\def\picsize{3cm}
\def\picfilename{unamN4.eps}
%If you do not have the picture file add:
%\let\nopictures=Y
%to the beginning of the file.
\ifx\nopictures Y\else{\ifx\epsfloaded Y\else\input epsf \fi
\let\epsfloaded=Y
{\ifx\picnaturalsize N\epsfxsize \picsize\fi \epsfbox{\picfilename}}}\fi
%%End InstantTeX Picture

\centerline{\bigfnt MAF}\par

\centerline{\it Examen 1}\par\centerline{\small\hoy}\Par
\ni 1) Calcular las integrales siguientes\par
\ni  $$ \int_{-1}^1 (1-x^2)^{1/2}x^{2n}dx = \cases{{\pi\over 2}& Si $n=0$\cr
\cr
\pi {(2n-1)!!\over (2n +2)!!}  & Si $n= 1,2,3...$}$$


\ni 2) La aproximaci\'on de {\it Bloch--Gruneissen} para el c\'alculo de la resistencia el\'ectrica $\rho$ en un metal monovalente es 

 $$\rho= C{T^5\over \Theta^6} \int_0^{\Theta/T}{x^5\over(e^x-1)\,(1-e^{-x})}dx.$$

\ni donde $C$ es una constante y $\Theta$ es la llamada temperatura de Debye, la cual depende del material. Demostrar que en el l\'{\i}mite cuando $T\to0$, $\rho$ vale
$$  \rho = 5!\zeta(5) C{T^5\over \Theta^6}.$$

\ni 3) Probar que la $\zeta(s)$ de Riemann se puede poner como  

$$   \zeta(s)=-{(-s)!\over 2\pi i} \int_{\cal C} {(-z)^{s-1} \over e^z -1} \,dz,$$

\end

% Created 2025-02-06 Thu 20:56
% Intended LaTeX compiler: pdflatex
\documentclass[letterpaper]{article}
\usepackage[utf8]{inputenc}
\usepackage[T1]{fontenc}
\usepackage{graphicx}
\usepackage{longtable}
\usepackage{wrapfig}
\usepackage{rotating}
\usepackage[normalem]{ulem}
\usepackage{amsmath}
\usepackage{amssymb}
\usepackage{capt-of}
\usepackage{hyperref}
\usepackage{lmodern} % Ensures we have the right font
\usepackage[utf8]{inputenc}
\usepackage{graphicx}
\usepackage{amsmath, amsthm, amssymb, amsfonts, amssymb, amscd}
\usepackage[table, xcdraw]{xcolor}
%\usepackage{mdsymbol}
\usepackage{tikz-cd}
\usepackage{float}
\usepackage[spanish, activeacute, ]{babel}
\usepackage{color}
\usepackage{transparent}
\graphicspath{{./figs/}}
\usepackage{makeidx}
\usepackage{afterpage}
\usepackage{array}
\usepackage{braket}
\usepackage{pst-node}
\newtheorem{teorema}{Teorema}[section]
\newtheorem{prop}[teorema]{Proposici\'on}
\newtheorem{cor}[teorema]{Corolario}
\newtheorem{lema}[teorema]{Lema}
\newtheorem{def.}{Definici\'on}[section]
\newtheorem{afir}{Afirmaci\'on}
\newtheorem{conjetura}{Conjetura}
\renewcommand{\figurename}{Figura}
\renewcommand{\indexname}{\'{I}ndice anal\'{\i}tico}
\newcommand{\zah}{\ensuremath{ \mathbb Z }}
\newcommand{\rac}{\ensuremath{ \mathbb Q }}
\newcommand{\nat}{\ensuremath{ \mathbb N }}
\newcommand{\prob}{\textbf{P}}
\newcommand{\esp}{\mathbb E}
\newcommand{\eje}{{\newline \noindent \sc \textbf{Ejemplo. }}}
\newcommand{\obs}{{\newline \noindent \sc \textbf{Observación. }}}
\newcommand{\dem}{{\noindent \sc Demostraci\'on. }}
\newcommand{\bg}{\ensuremath{\overline \Gamma}}
\newcommand{\ga}{\ensuremath{\gamma}}
\newcommand{\fb}{\ensuremath{\overline F}}
\newcommand{\la}{\ensuremath{\Lambda}}
\newcommand{\om}{\ensuremath{\Omega}}
\newcommand{\sig}{\ensuremath{\Sigma}}
\newcommand{\bt}{\ensuremath{\overline T}}
\newcommand{\li}{\ensuremath{\mathbb{L}}}
\newcommand{\ord}{\ensuremath{\mathbb{O}}}
\newcommand{\bs}{\ensuremath{\mathbb{S}^1}}
\newcommand{\co}{\ensuremath{\mathbb C }}
\newcommand{\con}{\ensuremath{\mathbb{C}^n}}
\newcommand{\cp}{\ensuremath{\mathbb{CP}}}
\newcommand{\rp}{\ensuremath{\mathbb{RP}}}
\newcommand{\re}{\ensuremath{\mathbb R }}
\newcommand{\hc}{\ensuremath{\widehat{\mathbb C} }}
\newcommand{\pslz}{\ensuremath{\mathrm{PSL}(2,\mathbb Z) }}
\newcommand{\pslr}{\ensuremath{\mathrm{PSL}(2,\mathbb R) }}
\newcommand{\pslc}{\ensuremath{\mathrm{PSL}(2,\mathbb C) }}
\newcommand{\hd}{\ensuremath{\mathbb H^2}}
\newcommand{\slz}{\ensuremath{\mathrm{SL}(2,\mathbb Z) }}
\newcommand{\slr}{\ensuremath{\mathrm{SL}(2,\mathbb R) }}
\newcommand{\slc}{\ensuremath{\mathrm{SL}(2,\mathbb C) }}
\newcommand{\mdlr}{\ensuremath{\mathrm{M}}}
\date{\today}
\title{Examen 1 de MAF}
\begin{document}

\maketitle
\noindent\textbf{Instrucciones: Tienen hasta el Mi'ercoles a las 11:59 pm para resolver el examen. Cada problema vale 2.5 puntos.}

\begin{enumerate}

  \item Sean $y_{1}(x)$, $y_{2}(x)$ soluciones linealmente independientes de la ecuaci'on homogenea
        \[
            y''(x)+p(x)y'(x)+q(x)y=0,\quad x\in I.
        \]
        Muestre que
        \[
        p(x)=\frac{y_{2}(x)y''_{1}(x)-y_{1}(x)y''_{2}(x)}{W(y_{1},y_{2})(x)}\quad q(x)=\frac{y'_{1}(x)y''_{2}(x)-y'_{2}(x)y''_{1}(x)}{W(y_{1},y_{2})(x)}
        \]

  \item Resuelva la ecuaci'on do \emph{Cauchy-Euler} con valores de fontera
        \begin{align*}
          x^{2}y''+xy'+\lambda y=0,\quad x\in(1,b)\\
          y(1)=y(b)=0
        \end{align*}
        \begin{itemize}
          \item Encuentre el espectro.
          \item Calcule la funci'on peso $\rho(x)$ que la hace autoadjunta y calcule \emph{producto interno modificado} para mostrar que las soluciones son en efecto son ortogonales.
        \end{itemize}

  \item El prop'osito de este problema es calcular la funci'on generadora de los polinomio de Hermite a partir de las ecuaciones recursivas.
        \begin{itemize}
          \item[a)] En clase vimos que los polinomio de Hermite son solucion a la ecuaci'on $H_{n}''(x)-2xH_{n}(x)+2nH_{n}(x)=0,$ para cada $n\in\nat$ y se calculan como
\[
      H_{n}=\sum_{k=0}^{\lfloor n/2\rfloor}\frac{(-1)^{k}n!}{k!(n-2k)!}(2x)^{n-2k},\quad\lfloor\cdot\rfloor\text{ es la funci'on piso.}
\]
          A partir de la ecuaci'on de $H_{n}$ y su definici'on, demuestre que se cumplen las relaciones de recurrencia
          \begin{align*}
            H'_{n}(x)&=2nH_{n-1}(x)\\
            H_{n+1}(x)&=2xH_{}(x)-2nH_{n-1}(x).
          \end{align*}
          \item[c)] Demuestre inductivamente que se cumple la f'ormula de Rodrigues
                \[
                H_{n}(x)=(-1)^{n}e^{x^{2}}\dfrac{d^{n}e^{-x^{2}}}{dx^{n}}
                \]
          \item[d)] Utilice la f'ormula de Rodrigues para demostrar que
                \[
                \braket{H_{n}|H_{m}}_{e^{-x^{2}}}=\int_{\infty}^{\infty}e^{-x^{2}}H_{n}(x)H_{m}(x)=2^{n}\sqrt{\pi}n!\delta_{nm}
                \]
        \end{itemize}
\noindent\emph{Punto extra: Calcule la funcion generadora de los polinomios de Hermite y demuestre que en efecto es funci'on generadora}

  \item Sean
\begin{equation}
  y''+\frac{1}{x}y'+\left(1-\frac{\nu^{2}}{x^{2}}\right)y=0,\quad x\in\re^{+}
\end{equation}
las \emph{ecuaciones de Bessel de orden \(\nu\).} Encuentre \(v(x)>0\) tal que \hbox{\(y(x)=u(x)v(x)\)}, donde $u$ es solucion a la ecuaci'on
\[
  u''+\left(1+\frac{1-4\nu^{2}}{4x^{2}}\right)u=0.
\]
Compare con \(u''+u=0\) para demostrar que si \(\nu\in[0,1/2]\), entonces las soluciones son ondulatorias.
\end{enumerate}
\end{document}

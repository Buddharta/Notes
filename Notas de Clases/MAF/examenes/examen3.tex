% Created 2025-02-06 Thu 20:56
% Intended LaTeX compiler: pdflatex
\documentclass[letterpaper]{article}
\usepackage[utf8]{inputenc}
\usepackage[T1]{fontenc}
\usepackage{graphicx}
\usepackage{longtable}
\usepackage{wrapfig}
\usepackage{rotating}
\usepackage[normalem]{ulem}
\usepackage{amsmath}
\usepackage{amssymb}
\usepackage{capt-of}
\usepackage{hyperref}
\usepackage{lmodern} % Ensures we have the right font
\usepackage[utf8]{inputenc}
\usepackage{graphicx}
\usepackage{amsmath, amsthm, amssymb, amsfonts, amssymb, amscd}
\usepackage[table, xcdraw]{xcolor}
%\usepackage{mdsymbol}
\usepackage{tikz-cd}
\usepackage{float}
\usepackage[spanish, activeacute, ]{babel}
\usepackage{color}
\usepackage{transparent}
\graphicspath{{./figs/}}
\usepackage{makeidx}
\usepackage{afterpage}
\usepackage{array}
\usepackage{braket}
\usepackage{pst-node}
\newtheorem{teorema}{Teorema}[section]
\newtheorem{prop}[teorema]{Proposici\'on}
\newtheorem{cor}[teorema]{Corolario}
\newtheorem{lema}[teorema]{Lema}
\newtheorem{def.}{Definici\'on}[section]
\newtheorem{afir}{Afirmaci\'on}
\newtheorem{conjetura}{Conjetura}
\renewcommand{\figurename}{Figura}
\renewcommand{\indexname}{\'{I}ndice anal\'{\i}tico}
\newcommand{\zah}{\ensuremath{ \mathbb Z }}
\newcommand{\rac}{\ensuremath{ \mathbb Q }}
\newcommand{\nat}{\ensuremath{ \mathbb N }}
\newcommand{\prob}{\textbf{P}}
\newcommand{\esp}{\mathbb E}
\newcommand{\eje}{{\newline \noindent \sc \textbf{Ejemplo. }}}
\newcommand{\obs}{{\newline \noindent \sc \textbf{Observación. }}}
\newcommand{\dem}{{\noindent \sc Demostraci\'on. }}
\newcommand{\bg}{\ensuremath{\overline \Gamma}}
\newcommand{\ga}{\ensuremath{\gamma}}
\newcommand{\fb}{\ensuremath{\overline F}}
\newcommand{\la}{\ensuremath{\Lambda}}
\newcommand{\om}{\ensuremath{\Omega}}
\newcommand{\sig}{\ensuremath{\Sigma}}
\newcommand{\bt}{\ensuremath{\overline T}}
\newcommand{\li}{\ensuremath{\mathbb{L}}}
\newcommand{\ord}{\ensuremath{\mathbb{O}}}
\newcommand{\bs}{\ensuremath{\mathbb{S}^1}}
\newcommand{\co}{\ensuremath{\mathbb C }}
\newcommand{\con}{\ensuremath{\mathbb{C}^n}}
\newcommand{\cp}{\ensuremath{\mathbb{CP}}}
\newcommand{\rp}{\ensuremath{\mathbb{RP}}}
\newcommand{\re}{\ensuremath{\mathbb R }}
\newcommand{\hc}{\ensuremath{\widehat{\mathbb C} }}
\newcommand{\pslz}{\ensuremath{\mathrm{PSL}(2,\mathbb Z) }}
\newcommand{\pslr}{\ensuremath{\mathrm{PSL}(2,\mathbb R) }}
\newcommand{\pslc}{\ensuremath{\mathrm{PSL}(2,\mathbb C) }}
\newcommand{\hd}{\ensuremath{\mathbb H^2}}
\newcommand{\slz}{\ensuremath{\mathrm{SL}(2,\mathbb Z) }}
\newcommand{\slr}{\ensuremath{\mathrm{SL}(2,\mathbb R) }}
\newcommand{\slc}{\ensuremath{\mathrm{SL}(2,\mathbb C) }}
\newcommand{\mdlr}{\ensuremath{\mathrm{M}}}
\date{\today}
\title{Examen 3 de MAF}
\begin{document}

\maketitle
\noindent\textbf{Instrucciones: Tienen hasta el Mi'ercoles a las 11:59 pm para resolver el examen. Cada problema vale 2.5 puntos.}

\begin{enumerate}

  \item Calcule el Laplaciano tres dimencional
        \[
        \nabla^{2}=\dfrac{\partial^{2}}{\partial x^{2}}+\dfrac{\partial^{2}}{\partial y^{2}}+\dfrac{\partial^{2}}{\partial z^{2}}
        \]
        en coordenadas esfericas y cilindricas, las cuales se definen como
        \begin{align*}
          \text{Esfericas: }(x,y,z)&=r(\sin\theta\cos\phi,\sin\theta\sin\phi,\cos\theta),\\ &r=\sqrt{x^{2}+y^{2}+z^{2}}\quad\phi=\arctan\left(\frac{y}{x}\right)\,\theta=\arccos\left(\frac{z}{r}\right)\\
          \text{Polares: }(x,y,z)&=(r\cos\theta,r\sin\theta,z),\\
          &r=\sqrt{x^{2}+y^{2}}\quad\theta=\arctan\left(\frac{y}{x}\right)
        \end{align*}
  \item Utilice el \emph{m'etodo de Bernoulli} para encontrar los sistemas de ecuaciones relacionados a la \emph{ecuaci'on de Laplace} en coordenadas cartesianas, esfericas y cilindricas (tres dimensiones).
  \item \emph{Resuelva} los sistemas obtenidos para la ecuaci'on de laplace $\nabla^{2}\psi=0$ en el conjunto $\Omega=\{(x,y,z)\,:\,0\leq x\leq a,\,0\leq y\leq b, \,0\leq z\}$ en coordenadas cartesianas con condiciones de Dirichlet
        \begin{align*}
          &\psi(0,y,z)=0\quad\psi(a,y,z)=0\\
          &\psi(x,0,z)=0\quad\psi(x,b,z)=0\\
          &\psi(x,y,0)=f(x,y)\quad\psi(x,y,z)\to0\,\text{ cuando }z\to\infty\\
          \end{align*}
  \item Sean
\begin{equation}
  \dfrac{d}{dx}\left[(1-x^{2})y'(x)\right]+n(n+1)y=0,\quad x\in [-1,1]
\end{equation}
        las \emph{ecuaciones de Legengre de orden \(n\),} y sean $P_{n}(x)$ sus soluciones polinomiales tales que $P_{n}(1)=1$, demuestre que se cumplen las ecuaciones de recurrencia
        \begin{align*}
          P'_{n+1}(x)-P'_{n-1}(x)&=(2n+1)P_{n}(x)\\
          (n+1)P_{n+1}(x)-nP_{n-1}(x)&=(2n+1)xP_{n}(x)\\
        \end{align*}
\end{enumerate}
\end{document}

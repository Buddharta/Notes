\input exfeti.sty
% 20 de oct. 1989
\newbox\Ancha
\def\gros#1{{\setbox\Ancha=\hbox{$#1$}
   \kern-.025em\copy\Ancha\kern-\wd\Ancha
   \kern.05em\copy\Ancha\kern-\wd\Ancha
   \kern-.025em\raise.0433em\box\Ancha}}
%\def\gros#1{\vec #1}
\font\biggfnt=cmr10 scaled\magstep 3
\font\bigfnt=cmr10 scaled\magstep 1
\font\titulob  = cmbx10 scaled\magstep 5

\def\ni{\noindent}
\def\esp{\par \vskip .2 in }
\baselineskip 18pt
\nopagenumbers

\def\ni{\noindent}  
\baselineskip 15pt    
\magnification 1100 
\def\Par{\par\vskip .15in} 
\def\esp{\Par \vskip .2 in }

%\hskip28pt{.}
\hskip20pt{.}\hskip6pt{.}
\vskip -70pt{
%%Begin InstantTeX Picture
\let\picnaturalsize=N
\def\picsize{10cm}
\def\picfilename{FETI00_4.eps}
%If you do not have the picture file add:
%\let\nopictures=Y
%to the beginning of the file.
\ifx\nopictures Y\else{\ifx\epsfloaded Y\else\input epsf \fi
\global\let\epsfloaded=Y
\hskip -30pt{\ifx\picnaturalsize N\epsfxsize \picsize\fi \epsfbox{\picfilename}}}\fi
%%End InstantTeX Picture
}

{\baselineskip10pt \centerline{\it Examen 4}\par
\centerline{\small\hoy}\par}
\Par
\ni 1.a) Mostrar que la expansi\'on de Fourier de $\hbox{cos}\,x$ es 

  $$ \hbox{cos}\,ax={2a\hbox{sen}\,a\pi \over \pi}\left[{1\over 2a^2} + \sum_{m=1}^\infty{(-1)^m\hbox{cos}\,mx\over (a^2 - m^2)}\right].$$

\ni b) A partir de este resultado, prueben que

$$  a\pi \hbox{cot}\, a\pi = 1-2\sum_{p=1}^\infty \zeta(2p)a^{2p}.$$


\ni 2) Verifiquen la siguiente  identidad para la $\delta(x)$ de Dirac. 

$$  \delta(\phi_1 - \phi_2) = {1\over 2\pi} \sum_{m=-\infty}^\infty e^{im(\phi_1 -\phi_2)}.$$ 

 
\ni 3) Prueben que de la  expansi\'on de Fourier de $f(x)= x$ en el intervalo $- \pi<x<\pi$ se obtiene, al integrar, el resultado

$$  {\pi^2\over 12} = \sum_{n=1} ^\infty {(-1)^{n+1}\over n^2}.$$ 

 
\ni 4.a) Probar que el desarrollo de Fourier para $x^2$ en el intervalo $-\pi\leq x \leq \pi$ es:

$$  x^2 = {\pi^2\over 3} + 4 \sum_{n=1}^\infty{(-1)^n\hbox{cos} \, nx\over n^2} .$$

\ni b) Usen este resultado y la identidad de Parseval para obtener $\zeta(4)$ y verificar el resultado para $\zeta(2)$.
\Par
\ni {\bf F\'ormulas, f\'ormulas...} \Par
\ni Serie de Fourier en el intervalo $-\pi\leq x \leq \pi$

$$f(x) = {a_0\over 2} + \sum_{m=1}^\infty a_m\hbox{cos}\,mx \, +  \sum_{m=1}^\infty b_m\hbox{sen}\,mx,$$

\ni en donde 

$$ a_0 = {1\over \pi}\int_{-\pi}^\pi f(x)\, dx,\quad a_m = {1\over \pi} \int_{-\pi}^\pi f(x)\,\hbox{cos}\,mx\, dx,\quad  b_m = {1\over \pi}
\int_{-\pi}^\pi f(x)\,\hbox{sen}\,mx\, .$$

\ni La identidad de {\it Parseval, } otrora caballero de la mesa redonda.

$$ {1\over \pi}\int_{-\pi}^\pi [f(x)]^2\, dx  = {a_0^2\over 2}  + \sum_{m=1}^\infty \left(a_m^2 + b_m^2  \right).$$

\ni La funci\'on {\it Zeta de Riemann:}
$$  \zeta(s) = \sum_{k=1}^\infty{1\over k^s},\qquad \zeta(2)= {\pi^2\over 6}.$$
\vfill
\eject
\end
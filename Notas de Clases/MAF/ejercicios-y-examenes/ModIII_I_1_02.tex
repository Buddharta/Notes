\input exfeti.sty

% 20 de oct. 1989
\newbox\Ancha
\def\gros#1{{\setbox\Ancha=\hbox{$#1$}
   \kern-.025em\copy\Ancha\kern-\wd\Ancha
   \kern.05em\copy\Ancha\kern-\wd\Ancha
   \kern-.025em\raise.0433em\box\Ancha}}
%\def\gros#1{\vec #1}
\font\titulo=cmbxti10 scaled 5000
\font\biggfnt=cmr10 scaled\magstep 3

\font\bigfnt=cmr10 scaled\magstep 1
\def\ni{\noindent}
\def\esp{\par \vskip .2 in }
\baselineskip 18pt \nopagenumbers

\def\ni{\noindent}
\baselineskip 15pt \magnification 1200
\def\Par{\par\vskip .15in}
\def\esp{\Par \vskip .2 in }
%----------------------------------empieza--------------------------------------

 \centerline{\bigggfnt F\'{\i}sica Moderna III}\Par
\centerline{{\bf Profesor:}  {\it Rodolfo P. Mart\'{\i}nez y Romero}}\Par

{\baselineskip 8pt \centerline{\it Examen 1}\par
\centerline{\small \hoy}
\Par

%-----------------------------------figura---------------------------------------------

%\newbox\z
\gdef\fetifig#1#2#3{\setbox1=\vbox{
%%Empieza figura
\let\picnaturalsize=N
\def\picsize{#3}
\def\picfilename{#1}
%If you do not have the picture file add:
%\let\nopictures=Y
%to the beginning of the file.
\ifx\nopictures Y\else{\ifx\epsfloaded Y\else\input epsf \fi
\global\let\epsfloaded=Y \hskip 2cm{\ifx\picnaturalsize
N\epsfxsize \picsize\fi \epsfbox{\picfilename}}}\fi
%%Termina figura
 \setbox\z=\hbox{}
\copy\z}
 \setbox2=\vbox{\vskip
4pt\splittopskip=\baselineskip\hsize 5cm {\hskip -60pt
{\titulo#2}}} \centerline{$\hskip -3cm\vcenter{\box1}\hskip
-4.8cm\vcenter{\box2}\hfill$} }
%\input epsf.sty
\vskip -125pt 
\fetifig{unamN4.eps}{\titulo}{2.5cm}


\vskip 70pt

%\n
\ni 1) Calcular el impulso m\'{\i}nimo del mes\'on $\pi^{-}$  necesario
para producir el mes\'on $\rho$ en  la reacci\'on $\pi^- +P\to\rho+n$, sobre protones en reposo. Las masas de cada part\'{\i}cula son aproximadamente:

$$\matrix{m_\pi =& 0.14 \quad\hbox{Gev.}\cr
          m_\rho =& 0.76 \quad\hbox{Gev.}\cr
          m_p = & 0.94 \quad\hbox{Gev.}\cr}
$$

\input epsf.tex
%%Begin InstantTeX Picture
\let\picnaturalsize=N
\def\picsize{10cm}
\def\picfilename{ex1.eps}
%If you do not have the picture file add:
%\let\nopictures=Y
%to the beginning of the file.
\ifx\nopictures Y\else{\ifx\epsfloaded Y\else\input epsf \fi
\let\epsfloaded=Y
\centerline{\ifx\picnaturalsize N\epsfxsize \picsize\fi \epsfbox{\picfilename}}}\fi
%%End InstantTeX Picture


\ni 2). Prueben que para una colisi\'on el\'astica, 

$$  s + t+ u = 2(m_1^2 + m_2^2),$$

\ni en donde $ s= (\tilde{p}_1 + \tilde{p}) ,\quad  t= (\tilde{p}_1 - \tilde{p}_3) ,\quad u= (\tilde{p}_1 - \tilde{p}_4 ).$ Recordemos que se define $ \tilde{p} $  como $\tilde{p}\equiv(E,\hbox{\bf p}). $

\vfill
\eject
\end









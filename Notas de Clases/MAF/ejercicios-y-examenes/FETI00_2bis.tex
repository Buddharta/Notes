\input exfeti.sty
% 20 de oct. 1989
\newbox\Ancha
\def\gros#1{{\setbox\Ancha=\hbox{$#1$}
   \kern-.025em\copy\Ancha\kern-\wd\Ancha
   \kern.05em\copy\Ancha\kern-\wd\Ancha
   \kern-.025em\raise.0433em\box\Ancha}}
%\def\gros#1{\vec #1}
\font\biggfnt=cmr10 scaled\magstep 3
\font\bigfnt=cmr10 scaled\magstep 1
\def\ni{\noindent}
\def\esp{\par \vskip .2 in }
\baselineskip 18pt
\nopagenumbers

\def\ni{\noindent}  
\baselineskip 15pt    
\magnification 1050 
\def\Par{\par\vskip .15in} 
\def\esp{\Par \vskip .2 in }
 


\vskip -100pt{
%%Begin InstantTeX Picture
\let\picnaturalsize=N
\def\picsize{15cm}
\def\picfilename{FETI00_2bis.eps}
%If you do not have the picture file add:
%\let\nopictures=Y
%to the beginning of the file.
\ifx\nopictures Y\else{\ifx\epsfloaded Y\else\input epsf \fi
\global\let\epsfloaded=Y
{\ifx\picnaturalsize N\epsfxsize \picsize\fi \epsfbox{\picfilename}}}\fi
%%End InstantTeX Picture
}

\centerline{\it Examen 2 {\it bis (de Halloween o d\'{\i}a de muertos)}}\par

\centerline{\small\hoy}\par
\centerline{\it El examen consta de 4 preguntitas muy faciles, quien conteste todas recibir\'a su calaverita.}\par
\ni 1) Probar que

$$  {d^m\over dx^m }{}_1\hskip -1ptF_1(a;c;x) ={ (a)_m\over (c)_m} \,{}_1\hskip -1ptF_1(a +m  ;c +m ;x). $$

\ni 2) La seguna soluci\'on linealmente independiente de la ecuaci\'on diferencial para la hipergeom\'etrica  $U(a;c;x)= $ se puede escribir como

$$ U(a;c;x)={\pi\over \hbox{\rm sen} \pi c}\left[{ {}_1\hskip -1ptF_1(a;c;x)\over (a-c)!\,(c-1)!} -x^{1-c} \,\,{ {}_1\hskip -1ptF_1(a+1-c;2-c;x)\over (a-1)!\,(1-c)!} \right],$$

\ni prueben que 

$$   U(a;c;x)={1\over \Gamma(a)}\int_{0}^\infty  e^{-xt}\,t^{a-1} (1+t)^{c-a-1}dt; \quad Re\,(a)>0\,\,\hbox{y}\,\,Re\,(c-a)>0. $$

{\it Sugerencia}  Prueben que la integral arriba definida es en realidad una soluci\'on de la ecuaci\'on diferencial para la hipergeom\'etrica confluente. Al meter la integral en la ecuaci\'on diferencial debe dar algo de la forma: 

$$  -{1\over \Gamma(a)}\int_{0}^\infty {d\over dt }\left[ e^{-xt}\,t^{a} (1+t)^{c-a}\right]dt .$$

\ni Prueben luego que la potencia m\'as baja en $x$ es $x^{1-c}$. Noten que este t\'ermino {\it no}  se obtiene desarrollando en serie la exponencial (no se trata de  un desarrollo en potencias enteras  alrededor del origen), sino jugando con el termino $  (1+t)^{c-a-1} =t^{c-a-1}\, (1 + 1/t)^{c-a-1} $.  
  
\ni 3) Prueben que

$$ H_n(x)= \left( 2x -{d\over dx}\right)^n 1. $$
\ni (As\'{\i} es, es el operador $\left( 2x -{d\over dx}\right)^n  $ actuando sobre el n\'umero uno). Prueben que vale para los dos primeros casos y luego prueben el resultado por inducci\'on. 

\ni 4) Probar que 

$$ \int_{-\infty }^\infty x e^{-x^2} H_n(x) H_m(x) dx = \sqrt{\pi} \left[  2^m(m+1)!\delta_{n,m+1} + 2^{m-1}  m! \delta_{n,m-1}\right]$$





\ni {\bf F\'ormulas, f\'ormulas....} \Par
\ni La ecuaci\'on diferencial para la hipergeom\'etrica confluente:
$$xy'' + (c-x)\,y' -ay=0.$$
\ni Relaciones de recurrencia para los {\it Polinomios de Hermite} 
$$ \eqalign{H_{n+1} (x) &= 2x H_n(x) -2n H_{n-1}(x),\cr
H_n'(x)&= 2nH_{n-1}(x).} $$
\ni Funci\'on generadora de los polinomios de Hermite:
$$g(x,t)=e^{-t^2 +2tx} =\sum_{m=0}^{\infty }{H_m(x) {t^m\over m!}} \,\,  . $$
Ortonormalizaci\'on de los polinomios de Hermite
$$  \int_{-\infty}^\infty e^{-x^2}  H_n(x) H_m(x) \,dx = 2^n \sqrt{\pi} n! \delta_{nm}.$$
\ni La funci\'on {\it Gama}
$$z! = \int_0^\infty t^z e^{-t}dt .$$
\ni Tambi\'en puede necesitarse
$$  z!(-z)! ={\pi z\over \hbox{sen} \pi z}$$

\vfill
\eject
\end
\input exfeti.sty

% 20 de oct. 1989
\newbox\Ancha
\def\gros#1{{\setbox\Ancha=\hbox{$#1$}
   \kern-.025em\copy\Ancha\kern-\wd\Ancha
   \kern.05em\copy\Ancha\kern-\wd\Ancha
   \kern-.025em\raise.0433em\box\Ancha}}
%\def\gros#1{\vec #1}
\font\titulo=cmbxti10 scaled 5000
\font\biggfnt=cmr10 scaled\magstep 3

\font\bigfnt=cmr10 scaled\magstep 1
\def\ni{\noindent}
\def\esp{\par \vskip .2 in }
\baselineskip 18pt \nopagenumbers

\def\ni{\noindent}
\baselineskip 15pt \magnification 1200
\def\Par{\par\vskip .15in}
\def\esp{\Par \vskip .2 in }
%----------------------------------empieza--------------------------------------

 \centerline{\bigggfnt F\'{\i}sica  Moderna III}\Par
\centerline{{\bf Profesor:}  {\it Rodolfo P. Mart\'{\i}nez y Romero}}\Par

{\baselineskip 8pt \centerline{\it Examen 2}\par
\centerline{\small \hoy}
\Par

%-----------------------------------figura---------------------------------------------

%\newbox\z
\gdef\fetifig#1#2#3{\setbox1=\vbox{
%%Empieza figura
\let\picnaturalsize=N
\def\picsize{#3}
\def\picfilename{#1}
%If you do not have the picture file add:
%\let\nopictures=Y
%to the beginning of the file.
\ifx\nopictures Y\else{\ifx\epsfloaded Y\else\input epsf \fi
\global\let\epsfloaded=Y \hskip 2cm{\ifx\picnaturalsize
N\epsfxsize \picsize\fi \epsfbox{\picfilename}}}\fi
%%Termina figura
 \setbox\z=\hbox{}
\copy\z}
 \setbox2=\vbox{\vskip
4pt\splittopskip=\baselineskip\hsize 5cm {\hskip -60pt
{\titulo#2}}} \centerline{$\hskip -3cm\vcenter{\box1}\hskip
-4.8cm\vcenter{\box2}\hfill$} }
%\input epsf.sty
\vskip -125pt 
\fetifig{unamN4.eps}{\titulo}{2.5cm}


\vskip 30pt

\libro

\ni 1).  Calcular la velocidad para una onda plana de la forma

$$ u({\bf k}) = \sqrt{{E+m_{{}_0}\over 2E}}\pmatrix{ \chi \cr {}\cr
{{\gros\sigma}\cdot {\bf k}\over E+m_{{}_0}}\chi \cr },$$

\ni en donde $\chi$ es un bi-espinor ($\hbar=c=1$).\Par

\ni 2). {\negrit Momento infinito.} En este problema estudiaremos el comportamiento ultrarrelativista de una soluci\'on libre de la ecuaci\'on de Dirac. Pondremos una representaci\'on en donde $\alpha^3$ sea diagonal: 

$$ \beta = \pmatrix{
0 & \sigma^3 \cr
\sigma^3 & 0 \cr
},\quad \alpha^3=\pmatrix{I & 0 \cr0 & -I \cr}, \quad\alpha^2 = \pmatrix{0 & \sigma^2 \cr
\sigma^2 & 0 \cr},\quad \alpha^1= \pmatrix{0 & \sigma^1 \cr\sigma^1 & 0 \cr}.$$\Par

\ni a) Tomen $\vec{\hbox{\bf p}}  \parallel  \hat{\hbox{\bf k}}.$ Prueben entonces que en el l\'{\i}mite ultrarrelativista ($v\rightarrow  1)$ , la ecuaci\'on de Dirac se reescribe como 

$$\left( 1 \mp \alpha^3\right) \Psi =0, $$

\ni cuya soluci\'on es $\Psi = \psi_\pm = 1/2\left( 1 \pm \alpha^3\right) \psi . $ En donde $\psi $ es un espinor arbitrario.\par
\ni b) Probar que para los componentes de abajo de la funci\'on de onda, o sea para $\psi_-$, en el l\'{\i}mite ultrarrelativista se obtiene

$$ \psi_- \rightarrow 0.  $$

\ni Es decir, ahora los componentes de abajo de la funcion de onda se comportan como {\it  peque\~nos componentes}, pero para el caso ultrarrelativista. Para probarlo, pongan  su sistema de referencia en el  CM de la part\'{\i}cula y apliquen una transformacion de Lorentz $ \Lambda(- \hat{\hbox{\bf k}}) $ para pasar al laboratorio. Recuerden que $ |\hat{\hbox{\bf k}}| = \gamma m_0 v $ y que entonces tenemos

$$ \Lambda(- \hat{\hbox{\bf k}}) = {1\over \sqrt{2(1 +\gamma)}}\left[ 1 + \gamma  + \alpha_3 v\gamma \right]  $$


\Par
\ni 3). Supongamos que la funci\'on de onda del \'atomo de hidr\'ogeno se encuentra  en el nivel exitado n=2.  Supongamos ahora que se realiza una medici\'on del momento angular sobre el sistema y se obtiene que $ j_x = 3/2. $ ?`Cu\'al es la probabilidad de que una segunda medici\'on (realizada un poco despu\'es) d\'e como resultado $j_z = 1/2$?

\ni Me parece que solamente necesitan que

$$ J_\pm {\cal Y}_{jm}(\theta,\phi)= \sqrt{j(j+1) - m(m\pm 1)} \, {\cal Y}_{jm\pm 1}(\theta,\phi).$$
\ni para el tercer problema, en tanto que para el primero es \'util acordarse de
$$ \left(\gros\sigma\cdot \hbox{\bf A}\right) \left(\gros\sigma\cdot \hbox{\bf B}\right)  = \hbox{\bf A}\cdot \hbox{\bf B} + i \gros\sigma\cdot(\hbox{\bf A}\times\hbox{\bf B} )$$

 \vfill
\eject
\end









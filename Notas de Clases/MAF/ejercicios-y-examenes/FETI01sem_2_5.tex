
\input exfeti.sty

% 20 de oct. 1989
\newbox\Ancha
\def\gros#1{{\setbox\Ancha=\hbox{$#1$}
   \kern-.025em\copy\Ancha\kern-\wd\Ancha
   \kern.05em\copy\Ancha\kern-\wd\Ancha
   \kern-.025em\raise.0433em\box\Ancha}}
%\def\gros#1{\vec #1}
\font\titulo=cmbxti10 scaled 5000
\font\biggfnt=cmr10 scaled\magstep 3

\font\bigfnt=cmr10 scaled\magstep 1
\def\ni{\noindent}
\def\esp{\par \vskip .2 in }
\baselineskip 18pt \nopagenumbers

\def\ni{\noindent}
\baselineskip 15pt \magnification 1100
\def\Par{\par\vskip .15in}
\def\esp{\Par \vskip .2 in }
%----------------------------------empieza--------------------------------------

 \centerline{\biggfnt Funciones
Especiales}\par\centerline{\slll y}\par
 \centerline{\biggfnt Transformadas Integrales}\Par

{\baselineskip 8pt \centerline{\it Examen 5}\par
\centerline{\small \hoy}
\Par

%-----------------------------------figura---------------------------------------------

%\newbox\z
\gdef\fetifig#1#2#3{\setbox1=\vbox{
%%Empieza figura
\let\picnaturalsize=N
\def\picsize{#3}
\def\picfilename{#1}
%If you do not have the picture file add:
%\let\nopictures=Y
%to the beginning of the file.
\ifx\nopictures Y\else{\ifx\epsfloaded Y\else\input epsf \fi
\global\let\epsfloaded=Y \hskip 2cm{\ifx\picnaturalsize
N\epsfxsize \picsize\fi \epsfbox{\picfilename}}}\fi
%%Termina figura
 \setbox\z=\hbox{}
\copy\z}
 \setbox2=\vbox{\vskip
4pt\splittopskip=\baselineskip\hsize 5cm {\hskip -60pt
{\titulo#2}}} \centerline{$\hskip -3cm\vcenter{\box1}\hskip
-4.8cm\vcenter{\box2}\hfill$} }
%\input epsf.sty
\vskip -150pt 
\fetifig{unamN3.eps}{\titulo}{2.5cm}


\vskip 70pt
\libro

\ni 1.a) Mostrar que la expansi\'on de Fourier de $\hbox{cos}\,x$ es: 

  $$ \hbox{cos}\,ax={2a\hbox{sen}\,a\pi \over \pi}\left[{1\over 2a^2} + \sum_{m=1}^\infty{(-1)^m\hbox{cos}\,mx\over (a^2 - m^2)}\right].$$

\ni b) A partir de este resultado, prueben que,

$$  a\pi \hbox{cot}\, a\pi = 1-2\sum_{p=1}^\infty \zeta(2p)a^{2p}.$$


\ni 2) Muestren que la $\delta(x-a)$ de Dirac desarrollada como una serie {\it seno} de Fourier en el  medio intervalo $[0,L]$ se escribe como

$$  \delta(x-a) = {2\over L} \sum_{n=1}^\infty \hbox{\rm sen} \left( {n\pi a\over L}\right)\hbox{\rm sen} \left( {n\pi x\over L}\right).$$

\ni Notar que en realidad esta serie describe a $-\delta(x+a) + \delta(x-a)$ en el intervalo completo $[-L,L]$. 

\ni 3) A partir de la transformada de Fourier y de su inversa,
prueben que el cambio de variables,

$$ \eqalign{x\rightarrow \,&\hbox{ln}\,t \cr ik\rightarrow\, &
\alpha -\gamma} $$

 \ni lleva a las expresiones

 $$\eqalign{G(\alpha)= & \int_{0}^\infty F(x) \, x^{\alpha -1}\, dx\cr
 F(x) = & {1\over 2\pi i}\int_{\gamma -i\infty}^{\gamma + i\infty}
  G(\alpha) x^{-\alpha}\, d\alpha  }$$

  \ni Estas integrales se llaman transformaciones de {\it Mellin.} (La segunda integral es en realidad una integral de {\it
Bromwich}, como la que se usa en la inversa de la transformada de
 Laplace)\par



\ni 4) La transformada seno de Fourier se define como

$$\eqalign{\tilde f_s(k) = &\sqrt{2\over \pi}\int_{0}^\infty f_s(t)\,\hbox{sen}\,
kt\, dt\cr
 f_s(x) = &\sqrt{2\over \pi}\int_{0}^\infty \tilde f_s(k)\,\hbox{sen}\,
kx\, dk }$$

\ni y una expresi\'on equivalente para la transformada coseno.
Muestren que la transformada de Fourier seno y coseno de $e^{-at}$
son, respectivamente,

$$ \eqalign{\tilde f_s(k) = &\sqrt{2\over \pi}{k\over k^2 +a^2}\cr
\tilde f_c(k) = &\sqrt{2\over \pi}{a\over k^2 +a^2}} $$

\ni Usando el teorema del residuo, encuentren las relaciones
inversas

$$  \eqalign{\int_{0}^\infty {k\,\hbox{sen}\,(kx)\over k^2+a^2}\,dk =
&{\pi\over 2}\,e^{-ax}\cr
 {}\cr
 \int_{0}^\infty
{\hbox{cos}\,(kx)\over k^2+a^2}\,dk = &{\pi\over 2 a}\,e^{-ax}}$$





\ni {\bf Algunas f\'ormulas \'utiles.} \Par

\ni Serie de Fourier en el intervalo $-\pi\leq x \leq \pi$

$$f(x) = {a_0\over 2} + \sum_{m=1}^\infty a_m\hbox{cos}\,mx \, +  \sum_{m=1}^\infty b_m\hbox{sen}\,mx,$$

\ni en donde 

$$ a_0 = {1\over \pi}\int_{-\pi}^\pi f(x)\, dx,\quad a_m = {1\over \pi} \int_{-\pi}^\pi f(x)\,\hbox{cos}\,mx\, dx,\quad  b_m = {1\over \pi}
\int_{-\pi}^\pi f(x)\,\hbox{sen}\,mx\, .$$


\ni La funci\'on {\it Zeta de Riemann:}
$$  \zeta(s) = \sum_{k=1}^\infty{1\over k^s},\qquad \zeta(2)= {\pi^2\over 6}.$$
\vfill \eject
\end

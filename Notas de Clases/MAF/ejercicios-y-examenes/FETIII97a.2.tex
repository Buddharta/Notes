\input exfeti.sty
% 20 de oct. 1989
\newbox\Ancha
\def\gros#1{{\setbox\Ancha=\hbox{$#1$}
   \kern-.025em\copy\Ancha\kern-\wd\Ancha
   \kern.05em\copy\Ancha\kern-\wd\Ancha
   \kern-.025em\raise.0433em\box\Ancha}}
%\def\gros#1{\vec #1}
\font\biggfnt=cmr10 scaled\magstep 3
\font\bigfnt=cmr10 scaled\magstep 1
\def\ni{\noindent}
\def\esp{\par \vskip .2 in }
\baselineskip 18pt
\magnification 1200
\nopagenumbers

\def\ni{\noindent}  
\baselineskip 15pt    
\magnification 1200 
\def\Par{\par\vskip .15in} 
\def\esp{\Par \vskip .2 in }
 
%%Begin InstantTeX Picture
\let\picnaturalsize=N
\def\picsize{12.5cm}
\def\picfilename{FETIII97.1.eps}
%If you do not have the picture file add:
%\let\nopictures=Y
%to the beginning of the file.
\ifx\nopictures Y\else{\ifx\epsfloaded Y\else\input epsf \fi
\let\epsfloaded=Y
{\ifx\picnaturalsize N\epsfxsize \picsize\fi \epsfbox{\picfilename}}}\fi
%%End InstantTeX Picture

%\centerline{\bigfnt FETI}\par

\centerline{\negrit Examen 2a}\Par
\centerline{\sll Integrales de Lommel.}\par
\ni 1) Calcular la integral siguiente. A esta identidad se le conoce como una de las {\it Integrales de Lommel.}\par

\ecmarcocen{}{\int dx \,x [\alpha J_0(\alpha x)J_0(\beta x) - \beta J_1(\alpha x)J_1(\beta x)] = xJ_1(\alpha x)J_0(\beta x).}

\ni 2) Demostrar la siguiente relaci\'on  

$$ \int_{0}^1 J_n(zt)\,t^{n + 1}dt = \left[{J_{n + 1}(z)\over z}\right].$$

\ni 3) Probar que las dos f\'ormulas siguientes son iguales
 
$$ \eqalign{T =& \,1 -{1\over 2ka}\int_0^{2ka}J_0(x)\, dx,\cr
T=& \,1- {1\over ka}\sum_{n=0}^\infty J_{2n + 1}(2ka).}  $$


\ni {\negrit Algunas f\'ormulas \'utiles.} \Par
\ni Serie de potencias de la funci\'on {\it Bessel}

$$J_\nu(x) =\sum_{k=0}^\infty {(-1)^k\over k!\,(k + \nu)! } \left({x\over 2}\right)^{\nu + 2k}.$$

\ni Relaciones de recurrencia

$$  J_{\nu -1}(x) + J_{\nu + 1}(x) = \left({2\nu\over x}\right) J_\nu (x),\qquad J_{\nu -1}(x) - J_{\nu + 1}(x) = 2\, J_\nu'(x).$$


\vfill
\eject
\end
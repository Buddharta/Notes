
%\documentstyle[aps,preprint]{revtex}
\documentstyle[12pt]{article}
\hoffset=-1.5cm
\textwidth=17cm


\begin{document}

\section{Problema \# 2}


Dos part\'\i culas no-id\'enticas, cada una de masa $m,\;$se\ mueven libremente 
en direccion $x$\ dentro de una caja impenetrable de longitud $L$. El
problema es unidimensional y denotamos por $x_{1},$ $x_{2}\;\;$las
coordenadas de las particulas $1$ y $2$ respectivamente.

(a) Escriba las energ\'\i as y funciones de onda  para los primeros tres estados
del sistema: aqu\'ellos donde a lo m\'as una part\'\i cula est\'a excitada fuera de su
estado base.

(b) Se agrega el potencial de interacci\'on%
\[
V=\lambda \delta (x_{1}-x_{2}).
\]
Calcule las energ\'\i as de estos estados a orden $\lambda $ y las
correspondientes funciones propias a orden $\lambda ^{0}.$

Las siguientes integrales pueden ser de utilidad%
\begin{eqnarray*}
\int_{0}^{\pi }dy\;\sin ^{4}y &=&\frac{3\pi }{8}. \\
\int_{0}^{\pi }dy\;\sin ^{2}2y\;\sin ^{2}y &=&\frac{\pi }{4}.
\end{eqnarray*}

\bigskip 

\subsection{SOLUCION}

\bigskip 

En una caja impenetrable las funciones y energias propias de cada particula
son%
\[
\Psi _{n}(x)=\sqrt{\frac{2}{L}}\sin \frac{n\pi x}{L},\;\;\;E_{n}=\frac{%
n^{2}\pi ^{2}\hbar ^{2}}{2mL^{2}},\;\;n=1,2,3,....
\]%
El sistema de dos particulas tendra%
\[
\Psi _{n_{1}n_{2}}(x_{1},x_{2})=\frac{2}{L}\sin \frac{n_{1}\pi x_{1}}{L}\sin 
\frac{n_{2}\pi x_{2}}{L},\;\;E_{n_{1}n_{2}}\;=\frac{\pi ^{2}\hbar ^{2}}{%
2mL^{2}}\left( n_{1}^{2}+n_{2}^{2}\right) \;
\]

\bigskip (a) Los primeros tres estados son%
\begin{eqnarray*}
\Psi _{11}(x_{1},x_{2}) &=&\frac{2}{L}\sin \frac{\pi x_{1}}{L}\sin \frac{\pi
x_{2}}{L},\;\;E_{11}\;=\frac{\pi ^{2}\hbar ^{2}}{mL^{2}}\; \\
\Psi _{21}(x_{1},x_{2}) &=&\frac{2}{L}\sin \frac{2\pi x_{1}}{L}\sin \frac{%
\pi x_{2}}{L},\;\;E_{21}\;=\frac{\pi ^{2}\hbar ^{2}}{mL^{2}}\frac{5}{2} \\
\Psi _{12}(x_{1},x_{2}) &=&\frac{2}{L}\sin \frac{\pi x_{1}}{L}\sin \frac{%
2\pi x_{2}}{L},\;\;E_{12}\;=\frac{\pi ^{2}\hbar ^{2}}{mL^{2}}\frac{5}{2}
\end{eqnarray*}%
El nivel $E=\frac{\pi ^{2}\hbar ^{2}}{mL^{2}}\frac{5}{2}$\ es degenerado.

(b) Las correcciones son:

(b-1)Para el estado fundamental tenemos%
\begin{eqnarray*}
\Delta E &=&<\Psi _{11}|V|\Psi _{11}> \\
&=&\int_{0}^{L}dx_{1}\int_{0}^{L}dx_{2}\left( \frac{2}{L}\sin \frac{\pi x_{1}%
}{L}\sin \frac{\pi x_{2}}{L}\right) ^{2}\lambda \delta (x_{1}-x_{2}) \\
&=&\lambda \frac{4}{L^{2}}\int_{0}^{L}dx_{1}\left( \sin \frac{\pi x_{1}}{L}%
\right) ^{4} \\
\Delta E &=&\lambda \frac{4}{L^{2}}\frac{L}{\pi }\int_{0}^{\pi }dy\;\sin
^{4}y
\end{eqnarray*}%
\begin{eqnarray*}
\int_{0}^{\pi }dy\;\sin ^{4}y &=&2\int_{0}^{\frac{\pi }{2}}dy\;\sin ^{4}y \\
\int_{0}^{\pi }dy\;\sin ^{4}y &=&2\frac{3!!}{4!!}\frac{\pi }{2}=\pi \frac{3}{%
8}
\end{eqnarray*}%
\[
\Delta E=\lambda \frac{4}{L^{2}}\frac{L}{\pi }\pi \frac{3}{8}=\frac{3}{2}%
\frac{\lambda }{L}
\]%
y la funcion de onda a orden cero es%
\[
\Psi _{11}(x_{1},x_{2})=\frac{2}{L}\sin \frac{\pi x_{1}}{L}\sin \frac{\pi
x_{2}}{L}
\]

(b-2) Para el siguiente nivel tenemos degeneracion por lo que debemos
calcular la matrix de la perturbacion en el subespacio $(\Psi _{21},\Psi
_{12})$

\[
W=\left[ 
\begin{array}{cc}
<\Psi _{21}|V|\Psi _{21}> & <\Psi _{21}|V|\Psi _{12}> \\ 
<\Psi _{12}|V|\Psi _{21}> & <\Psi _{12}|V|\Psi _{12}>%
\end{array}%
\right] 
\]%
Por simetria tenemos%
\[
<\Psi _{21}|V|\Psi _{21}>=<\Psi _{12}|V|\Psi _{12}>
\]%
y ademas%
\[
<\Psi _{12}|V|\Psi _{21}>=<\Psi _{21}|V|\Psi _{12}>^{\ast }
\]%
Tenemos%
\begin{eqnarray*}
&<&\Psi _{21}|V|\Psi _{21}>\;=\lambda \int_{0}^{L}dx\left( \frac{2}{L}\sin 
\frac{2\pi x}{L}\sin \frac{\pi x}{L}\right) ^{2} \\
&=&\lambda \frac{L}{\pi }\frac{4}{L^{2}}\int_{0}^{\pi }dy\;\sin ^{2}2y\;\sin
^{2}y=\lambda \frac{L}{\pi }\frac{4}{L^{2}}\frac{\pi }{4}=\frac{\lambda }{L}
\end{eqnarray*}%
\begin{eqnarray*}
&<&\Psi _{21}|V|\Psi _{12}>=\lambda \int_{0}^{L}dx\left( \frac{2}{L}\sin 
\frac{2\pi x}{L}\sin \frac{\pi x}{L}\right) \left( \frac{2}{L}\sin \frac{\pi
x}{L}\sin \frac{2\pi x}{L}\right)  \\
&=&\lambda \frac{4}{L^{2}}\int_{0}^{L}dx\;\sin ^{2}\frac{2\pi x}{L}\sin ^{2}%
\frac{\pi x}{L}=\lambda \frac{4}{L^{2}}\frac{L}{\pi }\int_{0}^{\pi }dy\;\sin
^{2}2y\sin ^{2}y=\frac{\lambda }{L}
\end{eqnarray*}%
Tenemos%
\[
W=\frac{\lambda }{L}\left[ 
\begin{array}{cc}
1 & 1 \\ 
1 & 1%
\end{array}%
\right] 
\]%
por lo que $\Delta E$ esta determinado por 
\[
\det \left[ 
\begin{array}{cc}
a-\Delta E & a \\ 
a & a-\Delta E%
\end{array}%
\right] =0,\;\;a=\frac{\lambda }{L}
\]%
Esto implica%
\begin{eqnarray*}
\left( a-\Delta E\right) ^{2}-a^{2} &=&0 \\
\Delta E^{2}-2a\Delta E &=&0 \\
\Delta E_{1} &=&0,\;\;\Delta E_{2}=2a
\end{eqnarray*}%
The corresponding zero order functions are%
\[
\Psi _{1}=\frac{1}{\sqrt{2}}\left( \Psi _{21}+\Psi _{12}\right) ,\;\Psi _{2}=%
\frac{1}{\sqrt{2}}\left( \Psi _{21}-\Psi _{12}\right) \;
\]%
respectivamente.

\end{document}

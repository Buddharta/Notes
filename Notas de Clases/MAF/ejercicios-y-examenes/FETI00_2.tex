\input exfeti.sty
% 20 de oct. 1989
\newbox\Ancha
\def\gros#1{{\setbox\Ancha=\hbox{$#1$}
   \kern-.025em\copy\Ancha\kern-\wd\Ancha
   \kern.05em\copy\Ancha\kern-\wd\Ancha
   \kern-.025em\raise.0433em\box\Ancha}}
%\def\gros#1{\vec #1}
\font\biggfnt=cmr10 scaled\magstep 3
\font\bigfnt=cmr10 scaled\magstep 1
\def\ni{\noindent}
\def\esp{\par \vskip .2 in }
\baselineskip 18pt
\nopagenumbers

\def\ni{\noindent}  
\baselineskip 15pt    
\magnification 1100 
\def\Par{\par\vskip .15in} 
\def\esp{\Par \vskip .2 in }
 
%%Begin InstantTeX Picture
\let\picnaturalsize=N
\def\picsize{10cm}
\def\picfilename{unamN2.eps}
%If you do not have the picture file add:
%\let\nopictures=Y
%to the beginning of the file.
\ifx\nopictures Y\else{\ifx\epsfloaded Y\else\input epsf \fi
\let\epsfloaded=Y
{\ifx\picnaturalsize N\epsfxsize \picsize\fi \epsfbox{\picfilename}}}\fi
%%End InstantTeX Picture

%\centerline{\bigfnt FETI}\par

\centerline{\it Examen 2}\par\centerline{\small\hoy}\Par
\ni 1) Probar que \par
\ni  $${}_2\hskip -1ptF_1(a,b;c;z) ={\Gamma(c)\over \Gamma(b)\,\Gamma(c-b)}\int_{0}^1 t^{b-1} (1-t)^{c-b-1}(1-tz)^{-a}dt  $$

\ni Cuando $ Re(c)>Re(b)>0$ y $Re(a)>1.$  Estas condiciones permiten que la integral converja. Tambi\'en hay que notar que cuando $a$ no es entera  (que es la parte de la integral que contiene a $z$), el eje real desde $1$ hasta $\infty$ es una l\'{\i}nea de corte. Esta f\'ormula es muy \'util para calcular propiedades de la funci\'on hipergeom\'etrica, por ejemplo  

$$  {d\over dz }{}_2\hskip -1ptF_1(a,b;c;z) = a{}_2\hskip -1ptF_1(a +1 ,b +1 ;c;z)$$ 


\ni 2) Probar la transformaci\'on de {\it Kummer:} 

 $$ {}_1\hskip -1ptF_1(a;c;x)= e^x {}_1\hskip -1ptF_1(c-a;c;-x)\quad U(a;c;x)=x^{1-c} \,U(a-c+1;2-c;x) .$$

\ni donde $U(a;c;x)= $ (la segunda soluci\'on independiente  de la ecuaci\'on diferencial para la hipergeom\'etrica), es igual a

$$ U(a;c;x)={\pi\over \hbox{\rm sen} \pi c}\left[{ {}_1\hskip -1ptF_1(a;c;x)\over (a-c)!\,(c-1)!} -x^{1-c} \,\,{ {}_1\hskip -1ptF_1(a+1-c;2-c;x)\over (a-1)!\,(1-c)!} \right]$$

\ni 3) Probar que   

$$    {}_1\hskip -1ptF_1(a;c;x)={\Gamma(c)\over \Gamma(a)\,\Gamma(c-a)}\int^1_0e^{xt} t^{a-1} (1-t)^{c-a-1} dt$$
\ni Aqu\'{\i} $Re\,(c)>Re\,(a)>0$

 
\ni 4) Desarrollar $x^{2r}$ en una serie de polnomios de Hermite de orden par y $x^{2r +1}$ en una serie de polinomios de Hermite impares.
 
$$  \eqalign{x^{2r} &= {(2r)!\over 2^{2r}} \sum_{n=0}^{r}{H_{2n} (x)\over (2n) ! \, (r-n)!}\cr
x^{2r+1} &= {(2r +1)!\over 2^{2r+1}} \sum_{n=0}^{r}{H_{2n+1} (x)\over (2n+1) ! \, (r-n)!}}$$

\ni 5) Probar que 
$$\eqalign{ \hbox{\rm a)}&\qquad\int_{-\infty}^\infty H_n(x)\,e^{-x^2/2}dx = \cases{\sqrt{2\pi}{(2m)! \over m!}& Si $n$ es par: $n=2m$ \cr
\cr
0  & Si $n$ es impar: $n= 2m+1$}\cr
\hbox{\rm b)}&\qquad \int_{-\infty}^\infty x H_n(x)\,e^{-x^2/2}dx = \cases{ 0& Si $n$ es par: $n=2m $ \cr
\cr
2\sqrt{2\pi}{(2m +1)! \over m!}  & Si $n$ es impar: $n= 2m+1 $}  }$$

\ni {\bf Algunas f\'ormulas \'utiles. (Y creo que ninguna in\'util).} \Par
\ni La ecuaci\'on diferencial para la hipergeom\'etrica:
$$xy'' + (c-x)\,y' -ay=0.$$
\ni La funci\'on {\it Beta:}
$$  B(m + 1, n +1) = 2 \int_0^{\pi/2}\cos^{2m + 1}\theta\,\sen^{2n+1}\theta\,d\theta= {m!\,n!\over (m+n+1)!}.$$
\ni Funci\'on generadora de los polinomios de Hermite:
$$g(x,t)=e^{-t^2 +2tx} =\sum_{m=0}^{\infty }{H_m(x) {t^m\over m!}}  $$
Ortonormalizaci\'on de los polinomios de Hermite
$$  \int_{-\infty}^\infty e^{-x^2}  H_n(x) H_m(x) \,dx = 2^n \sqrt{\pi} n! \delta_{nm}$$
F\'ormula de {\it Rodrigues}
$$  H_m(x)=(-1)^m e^{x^2}{d^m\over dx^m}  e^{-x^2}$$
\ni F\'ormula de {\smali duplicaci\'on de Legendre}
$$z!(z-{1\over 2})\,! = \sqrt{\pi}\, 2^{-2z} (2z)\,!$$

\vfill
\eject
\end
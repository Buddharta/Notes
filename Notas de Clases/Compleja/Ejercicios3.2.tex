% Created 2024-11-04 Mon 21:41
% Intended LaTeX compiler: pdflatex
\documentclass[letterpaper]{article}
\usepackage[utf8]{inputenc}
\usepackage[T1]{fontenc}
\usepackage{graphicx}
\usepackage{longtable}
\usepackage{wrapfig}
\usepackage{rotating}
\usepackage[normalem]{ulem}
\usepackage{amsmath}
\usepackage{amssymb}
\usepackage{capt-of}
\usepackage{hyperref}
\usepackage{lmodern} % Ensures we have the right font
\usepackage[utf8]{inputenc}
\usepackage{graphicx}
\usepackage{amsmath, amsthm, amssymb, amsfonts, amssymb, amscd}
\usepackage[table, xcdraw]{xcolor}
%\usepackage{mdsymbol}
\usepackage{tikz-cd}
\usepackage{float}
\usepackage[spanish, activeacute, ]{babel}
\usepackage{color}
\usepackage{transparent}
\graphicspath{{./figs/}}
\usepackage{makeidx}
\usepackage{afterpage}
\usepackage{array}
\usepackage{pst-node}
\newtheorem{teorema}{Teorema}[section]
\newtheorem{prop}[teorema]{Proposici\'on}
\newtheorem{cor}[teorema]{Corolario}
\newtheorem{lema}[teorema]{Lema}
\newtheorem{def.}{Definici\'on}[section]
\newtheorem{afir}{Afirmaci\'on}
\newtheorem{conjetura}{Conjetura}
\renewcommand{\figurename}{Figura}
\renewcommand{\indexname}{\'{I}ndice anal\'{\i}tico}
\newcommand{\zah}{\ensuremath{ \mathbb Z }}
\newcommand{\rac}{\ensuremath{ \mathbb Q }}
\newcommand{\nat}{\ensuremath{ \mathbb N }}
\newcommand{\prob}{\textbf{P}}
\newcommand{\esp}{\mathbb E}
\newcommand{\eje}{{\newline \noindent \sc \textbf{Ejemplo. }}}
\newcommand{\obs}{{\newline \noindent \sc \textbf{Observación. }}}
\newcommand{\dem}{{\noindent \sc Demostraci\'on. }}
\newcommand{\bg}{\ensuremath{\overline \Gamma}}
\newcommand{\ga}{\ensuremath{\gamma}}
\newcommand{\fb}{\ensuremath{\overline F}}
\newcommand{\la}{\ensuremath{\Lambda}}
\newcommand{\om}{\ensuremath{\Omega}}
\newcommand{\sig}{\ensuremath{\Sigma}}
\newcommand{\bt}{\ensuremath{\overline T}}
\newcommand{\li}{\ensuremath{\mathbb{L}}}
\newcommand{\ord}{\ensuremath{\mathbb{O}}}
\newcommand{\bs}{\ensuremath{\mathbb{S}^1}}
\newcommand{\co}{\ensuremath{\mathbb C }}
\newcommand{\con}{\ensuremath{\mathbb{C}^n}}
\newcommand{\cp}{\ensuremath{\mathbb{CP}}}
\newcommand{\rp}{\ensuremath{\mathbb{RP}}}
\newcommand{\re}{\ensuremath{\mathbb R }}
\newcommand{\hc}{\ensuremath{\widehat{\mathbb C} }}
\newcommand{\pslz}{\ensuremath{\mathrm{PSL}(2,\mathbb Z) }}
\newcommand{\pslr}{\ensuremath{\mathrm{PSL}(2,\mathbb R) }}
\newcommand{\pslc}{\ensuremath{\mathrm{PSL}(2,\mathbb C) }}
\newcommand{\hd}{\ensuremath{\mathbb H^2}}
\newcommand{\slz}{\ensuremath{\mathrm{SL}(2,\mathbb Z) }}
\newcommand{\slr}{\ensuremath{\mathrm{SL}(2,\mathbb R) }}
\newcommand{\slc}{\ensuremath{\mathrm{SL}(2,\mathbb C) }}
\newcommand{\mdlr}{\ensuremath{\mathrm{M}}}
\author{Carlos Eduardo Martínez Aguilar}
\date{\today}
\title{Dos ejemlos de la prueba M Latex Export}
\hypersetup{
 pdfauthor={Carlos Eduardo Martínez Aguilar},
 pdftitle={Dos ejemlos de la prueba M Latex Export},
 pdfkeywords={},
 pdfsubject={},
 pdfcreator={Emacs 29.4 (Org mode 9.7.11)}, 
 pdflang={Esp}}
\begin{document}

\maketitle
\tableofcontents

\section{Ejercicio 3.1-5}
\label{sec:org190ee27}
Se define la siguiente serie de funciones analíticas
\[
g(z)=\sum_{n=1}^{\infty}e^{-n}\cos(nz).
\]
Demuestre que \(g\) es holomorfa en \(\om=\{z\in\co\,|\,-1<\mathrm{Im}(z)<1\}\).

\dem Primero, para poder aplicar la prueba M de Weierstrass, buscamos una cota para los términos \(|\cos(nz)|\). Entonces por definición de coseno
\begin{align*}
|\cos(nz)|&=\big|\frac{e^{inz}-e^{-inz}}{2}\big|\\
&\leq\frac{|e^{inz}|+|e^{-inz}|}{2}\\
&=\frac{e^{-n\mathrm{Im}(z)}+e^{n\mathrm{Im}(z)}}{2}.
\end{align*}

\noindent Donde aplicamos la desigualdad del triángulo para la desigualdad y el hecho \(|e^{i(x+iy)}|=|e^{(ix-y)}|=e^{-y}\). Así para \(K\subset\om\) compacto, sea \(\delta\) la distancia de \(K\) a \(\partial\om=\{\mathrm{Im}(z)=1\}\cup\{\mathrm{Im}(z)=-1\}\), la cual existe pues \(K\) es compacto y \(\partial\om\) es cerrado. Por lo tanto si \(z\in K\), se tiene que \(\pm\mathrm{Im}(z)<1-\delta\). Por lo tanto tenemos que

\begin{align*}
\sum_{n=1}^{\infty}|e^{-n}\cos(nz)|&=\sum_{n=1}^{\infty}e^{-n}|\cos(nz)|\\
&\leq \sum_{n=1}^{\infty}e^{-n}e^{n(1-\delta)}\\
&=\sum_{n=1}^{\infty}e^{-n\delta}\\
&=\sum_{n=1}^{\infty}(e^{-\delta})^n=\frac{1}{1-e^{-\delta}}<\infty.
\end{align*}

\noindent Por lo tanto por la prueba M de Weierstrass y el teorema de Weierstrass, ls serie que define a \(g(z)\) converge absoluta y normalmente en \(\om\) y \(g(z)\) es holomorfa en \(\om\).
\qed
\section{Ejercicio 3.1-8}
\label{sec:orgd449b61}
Se define la siguiente serie de funciones analíticas
\[
g(z)=\sum_{n=1}^{\infty}2i\frac{z^n}{z^{2n}+1}.
\]
Demuestre que \(g\) es holomorfa en \(\om=\co\setminus\partial\Delta\), donde \(\partial\Delta=\{z\in\co\,|\,|z|=1\}\).

\dem Similar al problema anterior, buscamos una cota para los sumandos
\[
\Big|\frac{z^n}{z^{2n}+1}\Big|$.
\]

Entonces, sea \(D\subset\om\) un disco cerrado y sea \(\delta\) la distancia de \(D\) a \(\partial\Delta\), la cual existe pues ambos son conjuntos compactos. Ahora, existen dos posibilidades; \(|z|<1\) para toda \(z\in D\) o \(|z|>1\) para toda \(z\in D\), si \(|z|< 1\), entonces
\[
\vert z^{2n}+1\vert\geq 1-\vert z\vert^{2n}
\]
además se cumple que \(|z|^k<|z|\) para todo complejo de norma menor que uno y \(k\in\nat\). Más aún para todo \(z\in D\), se cumple que
\[
\vert z\vert<1+\delta\quad\implies\quad\vert z\vert^{2n}<1+\delta\quad\implies\quad-\vert z\vert^{2n}>-(1+\delta).
\]
Por lo tanto
\[
\vert z^{2n}+1\vert> 1-(1+\delta)=\delta
\quad\implies\quad\Big|\frac{z^n}{z^{2n}+1}\Big|\leq\frac{|z^n|}{\delta}.
\]

Ahora si \(|z|>1\), entonces

\[
\Big|\frac{z^n}{z^{2n}+1}\Big|=\frac{1}{|z^n+z^{-n}|}\leq\frac{1}{|z|^n-|z|^{-n}}\leq\frac{1}{|z|^n-1}\leq\frac{\kappa}{|z|^n}\leq\frac{\kappa}{(1-\delta)^n}.
\]
\noindent Donde \(\kappa\in\re^+\) es una cota para la sucesion funciónes en \(D\)
\[
f_n(z)=\frac{|z|^n}{|z|^n-1}.
\]
\noindent Notamos que \(\kappa\) existe pues el máximo de las \(f_n\) existen y \(f_n\rightarrow 1\) cuando \(n\rightarrow\infty\). Por lo tanto, al igual que en el ejercicio anterior, se puede acotar la serie que define a \(g(z)\) por una geométrica convergente, la convergencia de la serie geométrica en el caso \(|z|<1\) es más sencillo pues podemos suponer que \(|z|<r<1\),entonces el valor que acota la serie de los valores absolutos es \(1/\delta(1-r)\) y el caso \(|z|>1\), se puede verificar que se obtiene

$$\frac{2\kappa}{1-\frac{1}{(1-\delta)}}.$$

Por lo tanto por la prueba M de Weierstrass y el teorema de Weierstrass, ls serie que define a \(g(z)\) converge absoluta y normalmente en \(\om\) y \(g(z)\) es holomorfa en \(\om\).
\qed
\end{document}

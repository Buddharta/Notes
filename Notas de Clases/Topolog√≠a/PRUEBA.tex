\documentclass[10pt,letter,spanish]{article}
\usepackage{latexsym}
\usepackage[spanish,activeacute]{babel}
\usepackage[utf8]{inputenc}
\usepackage{graphicx}
\usepackage{anysize}
\usepackage{ amssymb }

\textwidth=18.5cm
\textheight=26cm
\marginsize{4cm}{4cm}{2cm}{2cm}
\begin{document}
\begin{center}

\large  \bf Espacios Cubrientes\\

\end{center} 

Como se vió en el capítulo anterior utilizando algunas propiedades de espacios cubrientes pudimos calcular el grupo fumdamental de la circunferencia. En este capítulo haremos exactamente lo contrario, es decir deduciremos propiedaes de los espacios cubrientes a travez del grupo fundamental.\\
Como sabemos una función (o proyección) $P:X \rightarrow Y$ es cubriente si para todo $y \in Y$ exite una vecindad $V$ tal que $P^{-1}(V)=\bigcup_{i \in I} U_i$ donde las $U_{i}$ son vecindades disjuntas y $P |_{U_{i}} :U_{i} \rightarrow V$ es un homeomorfismo, a las vecindades $V$ de $X$ que cumplen esta propiedad se les conoce como "admicibles", durate toda esta sección asumiremos que tanto $X$ como $Y$ son conexos y localmente conexos por trayectorias.
La teoría de proyecciones cubrientes es de gran importancia no sólo en la topología sino en diversas ramas de las matemáticas como el Análisis Complejo, la Geometría Diferencial y la Teoría de los Grupos de Lie entre otas.







\end{document}

\documentclass{article}
\usepackage[utf8]{inputenc}

\usepackage[T1]{fontenc}
\usepackage{enumitem}
\usepackage[top=3cm,bottom=2cm,left=3cm,right=3cm,marginparwidth=1.75cm]{geometry}


\usepackage{pifont}
\usepackage{amsmath}
\usepackage{graphicx}
\usepackage{xcolor}
%\usepackage[usenames]{color}
%\usepackage{footnote}
\usepackage{amssymb}
\usepackage{amsmath}
%\usepackage[colorinlistoftodos]{todonotes}
\usepackage{hyperref}
\usepackage{color}
\usepackage{multicol}


\title{Segundo Examen Parcial\\
Ecuaciones diferenciales I}
\date{Mayo 2023}

\begin{document}

\maketitle
\vspace{2cm}
\begin{enumerate}
\item \textbf{Punto Extra:} Considere el siguiente campo vectorial
        \[
            F(t,y)=\frac{-t+(t^{2}+4y)^{1/2}}{2},
        \]
        \noindent el cual claramente es continuo y está definido en todo $\mathbb{R}\times\mathbb{R}$.

Determine si todos los problemas de valores iniciales de este campo tienen soluciónes únicas y en caso de que no sea cierto, determine un problema de valores iniciales sin soluciones únicas y encuentre por lo menos dos soluciones distintas para éste. Además explique por qué el haber encontrado las soluciones distintas para el problema de valores iniciales dado, no viola el teorema de existencia y unicidad.
\end{enumerate}
\end{document}
